\documentclass[12pt]{book}

\usepackage[T1]{fontenc}
\usepackage[light,condensed]{iwona}

\usepackage{fancyhdr}
\pagestyle{fancy}
\fancyhf{}
\renewcommand{\headrulewidth}{0.5pt}
\renewcommand{\footrulewidth}{0pt}
%
\renewcommand{\chaptermark}[1]{\markboth{#1}{}}
\renewcommand{\sectionmark}[1]{\markright{\thesection.\ #1}}
%
\fancyhead[LE,RO]{\bfseries\thepage}
\fancyhead[LO]{\bfseries\rightmark}
\fancyhead[RE]{\bfseries\leftmark}

\usepackage[usenames]{color}
\usepackage{amssymb}
\usepackage{amsmath}
\usepackage{amscd}
\usepackage{amsthm}
\usepackage[utf8]{inputenc}
\usepackage[spanish,mexico]{babel}
\usepackage{enumerate}
\usepackage{pgf,tikz}
\usepackage{makeidx}
\usetikzlibrary{arrows}

\usepackage[colorlinks=true, linkcolor=blue, urlcolor=blue, citecolor=blue]{hyperref}

\usepackage[Glenn]{fncychap}
\ChNameVar{\bfseries\Large}
\ChNumVar{\Huge}
\ChTitleVar{\bfseries\Large}
\ChRuleWidth{0.5 pt}
\ChNameUpperCase
\ChTitleUpperCase

\theoremstyle{definition}
\newtheorem{df}{\textcolor{blue}{Definición}}[chapter]
\newtheorem{teo}{\textcolor{red}{Teorema}}[chapter]
\newtheorem{ej}{\textcolor{orange}{Ejemplo}}[chapter]
\newtheorem{prop}{\textcolor{blue}{Proposición}}[chapter]
\newtheorem{lema}{\textcolor{blue}{Lema}}[chapter]
\newtheorem{cor}{\textcolor{blue}{Corolario}}[chapter]
\newtheorem{obs}{\textcolor{red}{Observación}}[chapter]

\newenvironment{pba}{\noindent\textbf{Prueba:}}{\begin{flushright} $\square$ \end{flushright}}
\newenvironment{dem}{\noindent\textbf{Demostración:}}{\begin{flushright} \rule{1ex}{1ex} \end{flushright}}

\newcommand{\R}{\mathbb R}
\newcommand{\s}{\mathbb S}
\newcommand{\D}{\mathbb D}
\newcommand{\B}{\mathbb B}
\newcommand{\C}{\mathbb C}
\newcommand{\E}{\mathbb E}
\newcommand{\N}{\mathbb N}
\newcommand{\A}{{\bf \acute{A}rea}}

\makeindex

\begin{document}

\title{Notas de geometría}
\author{}
\date{}
\maketitle

\pagenumbering{roman}
\tableofcontents



%\chapter*{Libro I de Euclides}
\pagenumbering{arabic}


\chapter{Preliminares}
Funciones.\index{funciones}


- DENOTAR CRITERIOS DE CONGRUENCIA COMO \textbf{cc(???)}

\section{Área de objetos en el plano.}
En esta sección nos daremos a la tarea de contestar la siguiente pregunta: ¿Cómo calcular el área de objetos en el plano? Es probable que nos venga a la mente alguna fórmula que conocemos desde la educación básica para el cálculo de áreas, pero buscaremos una manera de deducir dichas expresiones.

Para comenzar, daremos la siguiente definición:

\begin{df}\label{paralelogramo}
Sea $\{A, B, C, D\}$ un conjunto de puntos distintos en el plano, no colineales de tres en tres. Decimos que  el conjunto $\{A, B, C, D\}$ es un \textcolor{red}{\bf paralelogramo}\index{paralelogramo}, que denotaremos como $\square ABCD$, si y solamente si se cumplen las siguientes condiciones:
\begin{itemize}
\item $\overline{AB}$ es paralela a $\overline{CD}$
\item $\overline{CB}$ es paralela a $\overline{DA}$
\end{itemize}
\end{df}

Como consecuencia de esto, daremos otras definiciones:

\begin{df}.
\begin{itemize}
\item Un paralelogramo $\square ABCD$ es un \textcolor{red}{\textbf{rectángulo}}\index{rectángulo} si y solamente si $\square ABCD$ es un paralelogramo con un ángulo interno recto.
\item Un paralelogramo $\square ABCD$ es un \textcolor{red}{{\bf cuadrado}}\index{cuadrado} si y solamente si $\square ABCD$ es un rectángulo en el que $|AB|=|BC|=|CD|=|DA|$.
\end{itemize}
\end{df}

El \index{área}área de un objeto es una medida que asignamos a dicho objeto. Por ejemplo, otras medidas que se pueden establecer entre los objetos son la longitud, el peso, la temperatura, etcétera. \\

Ahora, ¿cómo determinamos las diferentes medidas de los diferentes objetos?

Toda medida requiere de una unidad de medida para compararla; por ejemplo:
\begin{itemize}
\item Para determinar la longitud de un objeto se tiene que calcular cuántas veces la unidad llamada \textit{metro} es necesaria para tener la longitud del mismo.

\item Para calcular el peso de un objeto, se debe calcular cuántas veces la unidad de medida llamada \textit{gramo} es necesaria para tener el peso del mismo.
\end{itemize}

Así, para calcular el área de un objeto, necesitamos tener una unidad de medida que nos permita calcularla.

\begin{df}
Definiremos \index{unidad cuadrada} \textcolor{red}{{\bf unidad cuadrada}} al cuadrado que por lado mide una unidad. Es decir, una unidad cuadrada es la cantidad de espacio del plano, contenidad en un cuadrado $\square ABCD$ tal que $|AB|=1$.

\begin{figure}[!h]
\begin{center}
\definecolor{zzttqq}{rgb}{0.6,0.2,0.}
\begin{tikzpicture}[x=0.5cm,y=0.5cm]
\fill[color=zzttqq,fill=zzttqq,fill opacity=0.1] (1.06,0.) -- (6.92,0.) -- (6.92,5.86) -- (1.06,5.86) -- cycle;
\fill[color=zzttqq,fill=zzttqq,fill opacity=0.1] (8.,0.) -- (10.,0.) -- (10.,2.) -- (8.,2.) -- cycle;
\fill[color=zzttqq,fill=zzttqq,fill opacity=0.1] (-8.,0.) -- (-1.,0.) -- (-1.,7.) -- (-8.,7.) -- cycle;
\fill[color=zzttqq,fill=zzttqq,fill opacity=0.1] (-10.,0.) -- (-9.,0.) -- (-9.,1.) -- (-10.,1.) -- cycle;
\draw [color=zzttqq] (1.06,0.)-- (6.92,0.);
\draw [color=zzttqq] (6.92,0.)-- (6.92,5.86);
\draw [color=zzttqq] (6.92,5.86)-- (1.06,5.86);
\draw [color=zzttqq] (1.06,5.86)-- (1.06,0.);
\draw [color=zzttqq] (8.,0.)-- (10.,0.);
\draw [color=zzttqq] (10.,0.)-- (10.,2.);
\draw [color=zzttqq] (10.,2.)-- (8.,2.);
\draw [color=zzttqq] (8.,2.)-- (8.,0.);
\draw [color=zzttqq] (-8.,0.)-- (-1.,0.);
\draw [color=zzttqq] (-1.,0.)-- (-1.,7.);
\draw [color=zzttqq] (-1.,7.)-- (-8.,7.);
\draw [color=zzttqq] (-8.,7.)-- (-8.,0.);
\draw [color=zzttqq] (-10.,0.)-- (-9.,0.);
\draw [color=zzttqq] (-9.,0.)-- (-9.,1.);
\draw [color=zzttqq] (-9.,1.)-- (-10.,1.);
\draw [color=zzttqq] (-10.,1.)-- (-10.,0.);
\end{tikzpicture}
\caption{Diferentes unidades cuadradas.}\label{unidadcuadrada}
\end{center}
\end{figure}
\end{df}

\begin{df}\label{dfárea}\index{área}
Sea $P$ un subconjunto del plano. $\alpha$ es el \textcolor{red}{\textbf{área}} de $P$ si y solamente sí caben $\alpha$ unidades cuadradas dentro de $P$. Al área la denotaremos como $\alpha = \A(P)$.
\end{df}

Con esta convención, veamos cómo calcular áreas de objetos planos.

\begin{ej}
¿Qué área tiene un cuadrado cuyos lados miden $\frac{9}{8}$ unidades?
\end{ej}
\begin{pba}

%%%%%%%%%
%%%%%%%%%
%%%%%%%%%

Primero veamos la figura~\ref{cuadrado97}, en donde representaremos un cuadrado cuyos lados miden $\frac {17}{16}$ unidades.

\begin{figure}[!h]
\begin{center}
\definecolor{ffqqqq}{rgb}{1.,0.,0.}
\definecolor{zzttqq}{rgb}{0.6,0.2,0.}
\begin{tikzpicture}[x=0.5cm,y=0.5cm]
\fill[color=zzttqq,fill=zzttqq,fill opacity=0.1] (-3,0) -- (3,0) -- (3,6) -- (-3,6) -- cycle;
\draw [color=ffqqqq] (-3,0) -- (3,0);
\draw [color=ffqqqq] (3,0) -- (3,6);
\draw [color=ffqqqq] (3,6) -- (-3,6);
\draw [color=ffqqqq] (-3,6) -- (-3,0);
\draw[color=ffqqqq] (0,-1) node {$\frac{9}{7}$ \text{unidades}};
\end{tikzpicture}
\caption{Cuadrado de lado $\frac{9}{7}$ unidades.}\label{cuadrado97}
\end{center}
\end{figure}

Para calcular el área del cuadrado, debemos determinar de cuántas unidades cuadradas está formado, es decir, calcular la cantidad de cuadrados (cuyo lado mide una unidad) que caben en un cuadrado. Dado que el cuadrado tiene por lado $\frac{9}{7}$ unidades, determinemos unidades en una de sus bases para construir cuadrados que midan por lado una unidad, como en la figura~\ref{cuadradocuadrado}. \\

\begin{figure}[!h]
\begin{center}
\definecolor{ffqqqq}{rgb}{1.,0.,0.}
\definecolor{zzttqq}{rgb}{0.6,0.2,0.}
\begin{tikzpicture}[x=0.5cm,y=0.5cm]
\fill[color=zzttqq,fill=zzttqq,fill opacity=0.1] (-5.,0.) -- (5.,0.) -- (0.,8.66025403784) -- cycle;
\draw [color=ffqqqq] (-5.,0.)-- (5.,0.);
\draw [color=ffqqqq] (5.,0.)-- (0.,8.66025403784);
\draw [color=ffqqqq] (0.,8.66025403784)-- (-5.,0.);

\draw [dash pattern=on 2pt off 2pt] (-5,-1) -- (-5,10);
\draw [dash pattern=on 2pt off 2pt] (-4,-1) -- (-4,10);
\draw [dash pattern=on 2pt off 2pt] (-3,-1) -- (-3,10);
\draw [dash pattern=on 2pt off 2pt] (-2,-1) -- (-2,10);
\draw [dash pattern=on 2pt off 2pt] (-1,-1) -- (-1,10);
\draw [dash pattern=on 2pt off 2pt] (0,-1) -- (0,10);
\draw [dash pattern=on 2pt off 2pt] (1,-1) -- (1,10);
\draw [dash pattern=on 2pt off 2pt] (2,-1) -- (2,10);
\draw [dash pattern=on 2pt off 2pt] (3,-1) -- (3,10);
\draw [dash pattern=on 2pt off 2pt] (4,-1) -- (4,10);
\draw [dash pattern=on 2pt off 2pt] (5,-1) -- (5,10);

\draw [dash pattern=on 2pt off 2pt] (-5,0) -- (5,0);
\draw [dash pattern=on 2pt off 2pt] (-5,1) -- (5,1);
\draw [dash pattern=on 2pt off 2pt] (-5,2) -- (5,2);
\draw [dash pattern=on 2pt off 2pt] (-5,3) -- (5,3);
\draw [dash pattern=on 2pt off 2pt] (-5,4) -- (5,4);
\draw [dash pattern=on 2pt off 2pt] (-5,5) -- (5,5);
\draw [dash pattern=on 2pt off 2pt] (-5,6) -- (5,6);
\draw [dash pattern=on 2pt off 2pt] (-5,7) -- (5,7);
\draw [dash pattern=on 2pt off 2pt] (-5,8) -- (5,8);
\draw [dash pattern=on 2pt off 2pt] (-5,9) -- (5,9);
\draw [dash pattern=on 2pt off 2pt] (5,-1) -- (5,10);

\draw[color=ffqqqq] (0,-2) node {\text{10 unidades}};
\end{tikzpicture}
\caption{Triángulo equilátero cuyo lado mide 10 unidades.}\label{cuadradocuadrado}
\end{center}
\end{figure}

Al contar la cantidad de unidades cuadradas que están contenidas en el triángulo, tenemos que solamente hay 30 (figura~\ref{triangulounidaddentro}). \\

\begin{figure}[!h]
\begin{center}
\definecolor{ffqqqq}{rgb}{1.,0.,0.}
\definecolor{zzttqq}{rgb}{0.6,0.2,0.}
\definecolor{qqqqff}{rgb}{0.,0.,1.}
\definecolor{ffqqqq}{rgb}{1.,0.,0.}
\definecolor{zzttqq}{rgb}{0.6,0.2,0.}
\definecolor{cqcqcq}{rgb}{0.752941176471,0.752941176471,0.752941176471}

\begin{tikzpicture}[x=0.5cm,y=0.5cm]
\fill[color=zzttqq,fill=zzttqq,fill opacity=0.1] (-5.,0.) -- (5.,0.) -- (0.,8.66025403784) -- cycle;
\draw [color=ffqqqq] (-5.,0.)-- (5.,0.);
\draw [color=ffqqqq] (5.,0.)-- (0.,8.66025403784);
\draw [color=ffqqqq] (0.,8.66025403784)-- (-5.,0.);

\draw [dash pattern=on 2pt off 2pt] (-5,-1) -- (-5,10);
\draw [dash pattern=on 2pt off 2pt] (-4,-1) -- (-4,10);
\draw [dash pattern=on 2pt off 2pt] (-3,-1) -- (-3,10);
\draw [dash pattern=on 2pt off 2pt] (-2,-1) -- (-2,10);
\draw [dash pattern=on 2pt off 2pt] (-1,-1) -- (-1,10);
\draw [dash pattern=on 2pt off 2pt] (0,-1) -- (0,10);
\draw [dash pattern=on 2pt off 2pt] (1,-1) -- (1,10);
\draw [dash pattern=on 2pt off 2pt] (2,-1) -- (2,10);
\draw [dash pattern=on 2pt off 2pt] (3,-1) -- (3,10);
\draw [dash pattern=on 2pt off 2pt] (4,-1) -- (4,10);
\draw [dash pattern=on 2pt off 2pt] (5,-1) -- (5,10);

\draw [dash pattern=on 2pt off 2pt] (-5,0) -- (5,0);
\draw [dash pattern=on 2pt off 2pt] (-5,1) -- (5,1);
\draw [dash pattern=on 2pt off 2pt] (-5,2) -- (5,2);
\draw [dash pattern=on 2pt off 2pt] (-5,3) -- (5,3);
\draw [dash pattern=on 2pt off 2pt] (-5,4) -- (5,4);
\draw [dash pattern=on 2pt off 2pt] (-5,5) -- (5,5);
\draw [dash pattern=on 2pt off 2pt] (-5,6) -- (5,6);
\draw [dash pattern=on 2pt off 2pt] (-5,7) -- (5,7);
\draw [dash pattern=on 2pt off 2pt] (-5,8) -- (5,8);
\draw [dash pattern=on 2pt off 2pt] (-5,9) -- (5,9);
\draw [dash pattern=on 2pt off 2pt] (5,-1) -- (5,10);


\fill[color=zzttqq,fill=zzttqq,fill opacity=0.1] (-5.,0.) -- (5.,0.) -- (0.,8.66025403784) -- cycle;
\fill[color=qqqqff,fill=qqqqff,fill opacity=0.1] (-4.,0.) -- (-3.,0.) -- (-3.,1.) -- (-4.,1.) -- cycle;
\fill[color=qqqqff,fill=qqqqff,fill opacity=0.1] (-3.,0.) -- (-2.,0.) -- (-2.,1.) -- (-3.,1.) -- cycle;
\fill[color=qqqqff,fill=qqqqff,fill opacity=0.1] (-3.,1.) -- (-2.,1.) -- (-2.,2.) -- (-3.,2.) -- cycle;
\fill[color=qqqqff,fill=qqqqff,fill opacity=0.1] (-3.,2.) -- (-2.,2.) -- (-2.,3.) -- (-3.,3.) -- cycle;
\fill[color=qqqqff,fill=qqqqff,fill opacity=0.1] (-2.,3.) -- (-1.,3.) -- (-1.,4.) -- (-2.,4.) -- cycle;
\fill[color=qqqqff,fill=qqqqff,fill opacity=0.1] (-2.,4.) -- (-1.,4.) -- (-1.,5.) -- (-2.,5.) -- cycle;
\fill[color=qqqqff,fill=qqqqff,fill opacity=0.1] (-1.,5.) -- (0.,5.) -- (0.,6.) -- (-1.,6.) -- cycle;
\fill[color=qqqqff,fill=qqqqff,fill opacity=0.1] (0.,5.) -- (1.,5.) -- (1.,6.) -- (0.,6.) -- cycle;
\fill[color=qqqqff,fill=qqqqff,fill opacity=0.1] (0.,4.) -- (1.,4.) -- (1.,5.) -- (0.,5.) -- cycle;
\fill[color=qqqqff,fill=qqqqff,fill opacity=0.1] (-1.,4.) -- (0.,4.) -- (0.,5.) -- (-1.,5.) -- cycle;
\fill[color=qqqqff,fill=qqqqff,fill opacity=0.1] (1.,4.) -- (2.,4.) -- (2.,5.) -- (1.,5.) -- cycle;
\fill[color=qqqqff,fill=qqqqff,fill opacity=0.1] (1.,3.) -- (2.,3.) -- (2.,4.) -- (1.,4.) -- cycle;
\fill[color=qqqqff,fill=qqqqff,fill opacity=0.1] (2.,2.) -- (3.,2.) -- (3.,3.) -- (2.,3.) -- cycle;
\fill[color=qqqqff,fill=qqqqff,fill opacity=0.1] (3.,0.) -- (4.,0.) -- (4.,1.) -- (3.,1.) -- cycle;
\fill[color=qqqqff,fill=qqqqff,fill opacity=0.1] (2.,1.) -- (3.,1.) -- (3.,2.) -- (2.,2.) -- cycle;
\fill[color=qqqqff,fill=qqqqff,fill opacity=0.1] (2.,0.) -- (3.,0.) -- (3.,1.) -- (2.,1.) -- cycle;
\fill[color=qqqqff,fill=qqqqff,fill opacity=0.1] (0.,3.) -- (1.,3.) -- (1.,4.) -- (0.,4.) -- cycle;
\fill[color=qqqqff,fill=qqqqff,fill opacity=0.1] (-1.,3.) -- (0.,3.) -- (0.,4.) -- (-1.,4.) -- cycle;
\fill[color=qqqqff,fill=qqqqff,fill opacity=0.1] (-2.,2.) -- (-1.,2.) -- (-1.,3.) -- (-2.,3.) -- cycle;
\fill[color=qqqqff,fill=qqqqff,fill opacity=0.1] (-1.,2.) -- (0.,2.) -- (0.,3.) -- (-1.,3.) -- cycle;
\fill[color=qqqqff,fill=qqqqff,fill opacity=0.1] (0.,2.) -- (1.,2.) -- (1.,3.) -- (0.,3.) -- cycle;
\fill[color=qqqqff,fill=qqqqff,fill opacity=0.1] (1.,2.) -- (2.,2.) -- (2.,3.) -- (1.,3.) -- cycle;
\fill[color=qqqqff,fill=qqqqff,fill opacity=0.1] (-2.,1.) -- (-1.,1.) -- (-1.,2.) -- (-2.,2.) -- cycle;
\fill[color=qqqqff,fill=qqqqff,fill opacity=0.1] (-1.,1.) -- (0.,1.) -- (0.,2.) -- (-1.,2.) -- cycle;
\fill[color=qqqqff,fill=qqqqff,fill opacity=0.1] (0.,1.) -- (1.,1.) -- (1.,2.) -- (0.,2.) -- cycle;
\fill[color=qqqqff,fill=qqqqff,fill opacity=0.1] (1.,1.) -- (2.,1.) -- (2.,2.) -- (1.,2.) -- cycle;
\fill[color=qqqqff,fill=qqqqff,fill opacity=0.1] (-2.,0.) -- (-1.,0.) -- (-1.,1.) -- (-2.,1.) -- cycle;
\fill[color=qqqqff,fill=qqqqff,fill opacity=0.1] (-1.,0.) -- (0.,0.) -- (0.,1.) -- (-1.,1.) -- cycle;
\fill[color=qqqqff,fill=qqqqff,fill opacity=0.1] (0.,0.) -- (1.,0.) -- (1.,1.) -- (0.,1.) -- cycle;
\fill[color=qqqqff,fill=qqqqff,fill opacity=0.1] (1.,0.) -- (2.,0.) -- (2.,1.) -- (1.,1.) -- cycle;
\draw [color=ffqqqq] (-5.,0.)-- (5.,0.);
\draw [color=ffqqqq] (5.,0.)-- (0.,8.66025403784);
\draw [color=ffqqqq] (0.,8.66025403784)-- (-5.,0.);
\draw [color=qqqqff] (-4.,0.)-- (-3.,0.);
\draw [color=qqqqff] (-3.,0.)-- (-3.,1.);
\draw [color=qqqqff] (-3.,1.)-- (-4.,1.);
\draw [color=qqqqff] (-4.,1.)-- (-4.,0.);
\draw [color=qqqqff] (-3.,0.)-- (-2.,0.);
\draw [color=qqqqff] (-2.,0.)-- (-2.,1.);
\draw [color=qqqqff] (-2.,1.)-- (-3.,1.);
\draw [color=qqqqff] (-3.,1.)-- (-3.,0.);
\draw [color=qqqqff] (-3.,1.)-- (-2.,1.);
\draw [color=qqqqff] (-2.,1.)-- (-2.,2.);
\draw [color=qqqqff] (-2.,2.)-- (-3.,2.);
\draw [color=qqqqff] (-3.,2.)-- (-3.,1.);
\draw [color=qqqqff] (-3.,2.)-- (-2.,2.);
\draw [color=qqqqff] (-2.,2.)-- (-2.,3.);
\draw [color=qqqqff] (-2.,3.)-- (-3.,3.);
\draw [color=qqqqff] (-3.,3.)-- (-3.,2.);
\draw [color=qqqqff] (-2.,3.)-- (-1.,3.);
\draw [color=qqqqff] (-1.,3.)-- (-1.,4.);
\draw [color=qqqqff] (-1.,4.)-- (-2.,4.);
\draw [color=qqqqff] (-2.,4.)-- (-2.,3.);
\draw [color=qqqqff] (-2.,4.)-- (-1.,4.);
\draw [color=qqqqff] (-1.,4.)-- (-1.,5.);
\draw [color=qqqqff] (-1.,5.)-- (-2.,5.);
\draw [color=qqqqff] (-2.,5.)-- (-2.,4.);
\draw [color=qqqqff] (-1.,5.)-- (0.,5.);
\draw [color=qqqqff] (0.,5.)-- (0.,6.);
\draw [color=qqqqff] (0.,6.)-- (-1.,6.);
\draw [color=qqqqff] (-1.,6.)-- (-1.,5.);
\draw [color=qqqqff] (0.,5.)-- (1.,5.);
\draw [color=qqqqff] (1.,5.)-- (1.,6.);
\draw [color=qqqqff] (1.,6.)-- (0.,6.);
\draw [color=qqqqff] (0.,6.)-- (0.,5.);
\draw [color=qqqqff] (0.,4.)-- (1.,4.);
\draw [color=qqqqff] (1.,4.)-- (1.,5.);
\draw [color=qqqqff] (1.,5.)-- (0.,5.);
\draw [color=qqqqff] (0.,5.)-- (0.,4.);
\draw [color=qqqqff] (-1.,4.)-- (0.,4.);
\draw [color=qqqqff] (0.,4.)-- (0.,5.);
\draw [color=qqqqff] (0.,5.)-- (-1.,5.);
\draw [color=qqqqff] (-1.,5.)-- (-1.,4.);
\draw [color=qqqqff] (1.,4.)-- (2.,4.);
\draw [color=qqqqff] (2.,4.)-- (2.,5.);
\draw [color=qqqqff] (2.,5.)-- (1.,5.);
\draw [color=qqqqff] (1.,5.)-- (1.,4.);
\draw [color=qqqqff] (1.,3.)-- (2.,3.);
\draw [color=qqqqff] (2.,3.)-- (2.,4.);
\draw [color=qqqqff] (2.,4.)-- (1.,4.);
\draw [color=qqqqff] (1.,4.)-- (1.,3.);
\draw [color=qqqqff] (2.,2.)-- (3.,2.);
\draw [color=qqqqff] (3.,2.)-- (3.,3.);
\draw [color=qqqqff] (3.,3.)-- (2.,3.);
\draw [color=qqqqff] (2.,3.)-- (2.,2.);
\draw [color=qqqqff] (3.,0.)-- (4.,0.);
\draw [color=qqqqff] (4.,0.)-- (4.,1.);
\draw [color=qqqqff] (4.,1.)-- (3.,1.);
\draw [color=qqqqff] (3.,1.)-- (3.,0.);
\draw [color=qqqqff] (2.,1.)-- (3.,1.);
\draw [color=qqqqff] (3.,1.)-- (3.,2.);
\draw [color=qqqqff] (3.,2.)-- (2.,2.);
\draw [color=qqqqff] (2.,2.)-- (2.,1.);
\draw [color=qqqqff] (2.,0.)-- (3.,0.);
\draw [color=qqqqff] (3.,0.)-- (3.,1.);
\draw [color=qqqqff] (3.,1.)-- (2.,1.);
\draw [color=qqqqff] (2.,1.)-- (2.,0.);
\draw [color=qqqqff] (0.,3.)-- (1.,3.);
\draw [color=qqqqff] (1.,3.)-- (1.,4.);
\draw [color=qqqqff] (1.,4.)-- (0.,4.);
\draw [color=qqqqff] (0.,4.)-- (0.,3.);
\draw [color=qqqqff] (-1.,3.)-- (0.,3.);
\draw [color=qqqqff] (0.,3.)-- (0.,4.);
\draw [color=qqqqff] (0.,4.)-- (-1.,4.);
\draw [color=qqqqff] (-1.,4.)-- (-1.,3.);
\draw [color=qqqqff] (-2.,2.)-- (-1.,2.);
\draw [color=qqqqff] (-1.,2.)-- (-1.,3.);
\draw [color=qqqqff] (-1.,3.)-- (-2.,3.);
\draw [color=qqqqff] (-2.,3.)-- (-2.,2.);
\draw [color=qqqqff] (-1.,2.)-- (0.,2.);
\draw [color=qqqqff] (0.,2.)-- (0.,3.);
\draw [color=qqqqff] (0.,3.)-- (-1.,3.);
\draw [color=qqqqff] (-1.,3.)-- (-1.,2.);
\draw [color=qqqqff] (0.,2.)-- (1.,2.);
\draw [color=qqqqff] (1.,2.)-- (1.,3.);
\draw [color=qqqqff] (1.,3.)-- (0.,3.);
\draw [color=qqqqff] (0.,3.)-- (0.,2.);
\draw [color=qqqqff] (1.,2.)-- (2.,2.);
\draw [color=qqqqff] (2.,2.)-- (2.,3.);
\draw [color=qqqqff] (2.,3.)-- (1.,3.);
\draw [color=qqqqff] (1.,3.)-- (1.,2.);
\draw [color=qqqqff] (-2.,1.)-- (-1.,1.);
\draw [color=qqqqff] (-1.,1.)-- (-1.,2.);
\draw [color=qqqqff] (-1.,2.)-- (-2.,2.);
\draw [color=qqqqff] (-2.,2.)-- (-2.,1.);
\draw [color=qqqqff] (-1.,1.)-- (0.,1.);
\draw [color=qqqqff] (0.,1.)-- (0.,2.);
\draw [color=qqqqff] (0.,2.)-- (-1.,2.);
\draw [color=qqqqff] (-1.,2.)-- (-1.,1.);
\draw [color=qqqqff] (0.,1.)-- (1.,1.);
\draw [color=qqqqff] (1.,1.)-- (1.,2.);
\draw [color=qqqqff] (1.,2.)-- (0.,2.);
\draw [color=qqqqff] (0.,2.)-- (0.,1.);
\draw [color=qqqqff] (1.,1.)-- (2.,1.);
\draw [color=qqqqff] (2.,1.)-- (2.,2.);
\draw [color=qqqqff] (2.,2.)-- (1.,2.);
\draw [color=qqqqff] (1.,2.)-- (1.,1.);
\draw [color=qqqqff] (-2.,0.)-- (-1.,0.);
\draw [color=qqqqff] (-1.,0.)-- (-1.,1.);
\draw [color=qqqqff] (-1.,1.)-- (-2.,1.);
\draw [color=qqqqff] (-2.,1.)-- (-2.,0.);
\draw [color=qqqqff] (-1.,0.)-- (0.,0.);
\draw [color=qqqqff] (0.,0.)-- (0.,1.);
\draw [color=qqqqff] (0.,1.)-- (-1.,1.);
\draw [color=qqqqff] (-1.,1.)-- (-1.,0.);
\draw [color=qqqqff] (0.,0.)-- (1.,0.);
\draw [color=qqqqff] (1.,0.)-- (1.,1.);
\draw [color=qqqqff] (1.,1.)-- (0.,1.);
\draw [color=qqqqff] (0.,1.)-- (0.,0.);
\draw [color=qqqqff] (1.,0.)-- (2.,0.);
\draw [color=qqqqff] (2.,0.)-- (2.,1.);
\draw [color=qqqqff] (2.,1.)-- (1.,1.);
\draw [color=qqqqff] (1.,1.)-- (1.,0.);

\draw[color=ffqqqq] (0,-2) node {\text{10 unidades}};
\end{tikzpicture}
\caption{Unidades cuadradas contenidas en el triángulo.}\label{triangulounidaddentro}
\end{center}
\end{figure}
 
También, podemos aproximarnos al área del triángulo al contar la cantidad de unidades cuadradas que contienen al triángulo; tenemos que solamente hay 58 unidades cuadradas (figura~\ref{triangulounidadfuera}).

\newpage

\begin{figure}[!h]
\begin{center}
\definecolor{ffqqqq}{rgb}{1.,0.,0.}
\definecolor{zzttqq}{rgb}{0.6,0.2,0.}
\definecolor{qqqqff}{rgb}{0.,0.,1.}
\definecolor{ffqqqq}{rgb}{1.,0.,0.}
\definecolor{zzttqq}{rgb}{0.6,0.2,0.}
\definecolor{cqcqcq}{rgb}{0.752941176471,0.752941176471,0.752941176471}

\begin{tikzpicture}[x=0.5cm,y=0.5cm]
\fill[color=zzttqq,fill=zzttqq,fill opacity=0.1] (-5.,0.) -- (5.,0.) -- (0.,8.66025403784) -- cycle;
\draw [color=ffqqqq] (-5.,0.)-- (5.,0.);
\draw [color=ffqqqq] (5.,0.)-- (0.,8.66025403784);
\draw [color=ffqqqq] (0.,8.66025403784)-- (-5.,0.);

\draw [dash pattern=on 2pt off 2pt] (-5,-1) -- (-5,10);
\draw [dash pattern=on 2pt off 2pt] (-4,-1) -- (-4,10);
\draw [dash pattern=on 2pt off 2pt] (-3,-1) -- (-3,10);
\draw [dash pattern=on 2pt off 2pt] (-2,-1) -- (-2,10);
\draw [dash pattern=on 2pt off 2pt] (-1,-1) -- (-1,10);
\draw [dash pattern=on 2pt off 2pt] (0,-1) -- (0,10);
\draw [dash pattern=on 2pt off 2pt] (1,-1) -- (1,10);
\draw [dash pattern=on 2pt off 2pt] (2,-1) -- (2,10);
\draw [dash pattern=on 2pt off 2pt] (3,-1) -- (3,10);
\draw [dash pattern=on 2pt off 2pt] (4,-1) -- (4,10);
\draw [dash pattern=on 2pt off 2pt] (5,-1) -- (5,10);

\draw [dash pattern=on 2pt off 2pt] (-5,0) -- (5,0);
\draw [dash pattern=on 2pt off 2pt] (-5,1) -- (5,1);
\draw [dash pattern=on 2pt off 2pt] (-5,2) -- (5,2);
\draw [dash pattern=on 2pt off 2pt] (-5,3) -- (5,3);
\draw [dash pattern=on 2pt off 2pt] (-5,4) -- (5,4);
\draw [dash pattern=on 2pt off 2pt] (-5,5) -- (5,5);
\draw [dash pattern=on 2pt off 2pt] (-5,6) -- (5,6);
\draw [dash pattern=on 2pt off 2pt] (-5,7) -- (5,7);
\draw [dash pattern=on 2pt off 2pt] (-5,8) -- (5,8);
\draw [dash pattern=on 2pt off 2pt] (-5,9) -- (5,9);
\draw [dash pattern=on 2pt off 2pt] (5,-1) -- (5,10);

\fill[color=zzttqq,fill=zzttqq,fill opacity=0.1] (-5.,0.) -- (5.,0.) -- (0.,8.66025403784) -- cycle;
\fill[color=qqqqff,fill=qqqqff,fill opacity=0.1] (-4.,0.) -- (-3.,0.) -- (-3.,1.) -- (-4.,1.) -- cycle;
\fill[color=qqqqff,fill=qqqqff,fill opacity=0.1] (-3.,0.) -- (-2.,0.) -- (-2.,1.) -- (-3.,1.) -- cycle;
\fill[color=qqqqff,fill=qqqqff,fill opacity=0.1] (-3.,1.) -- (-2.,1.) -- (-2.,2.) -- (-3.,2.) -- cycle;
\fill[color=qqqqff,fill=qqqqff,fill opacity=0.1] (-3.,2.) -- (-2.,2.) -- (-2.,3.) -- (-3.,3.) -- cycle;
\fill[color=qqqqff,fill=qqqqff,fill opacity=0.1] (-2.,3.) -- (-1.,3.) -- (-1.,4.) -- (-2.,4.) -- cycle;
\fill[color=qqqqff,fill=qqqqff,fill opacity=0.1] (-2.,4.) -- (-1.,4.) -- (-1.,5.) -- (-2.,5.) -- cycle;
\fill[color=qqqqff,fill=qqqqff,fill opacity=0.1] (-1.,5.) -- (0.,5.) -- (0.,6.) -- (-1.,6.) -- cycle;
\fill[color=qqqqff,fill=qqqqff,fill opacity=0.1] (0.,5.) -- (1.,5.) -- (1.,6.) -- (0.,6.) -- cycle;
\fill[color=qqqqff,fill=qqqqff,fill opacity=0.1] (0.,4.) -- (1.,4.) -- (1.,5.) -- (0.,5.) -- cycle;
\fill[color=qqqqff,fill=qqqqff,fill opacity=0.1] (-1.,4.) -- (0.,4.) -- (0.,5.) -- (-1.,5.) -- cycle;
\fill[color=qqqqff,fill=qqqqff,fill opacity=0.1] (1.,4.) -- (2.,4.) -- (2.,5.) -- (1.,5.) -- cycle;
\fill[color=qqqqff,fill=qqqqff,fill opacity=0.1] (1.,3.) -- (2.,3.) -- (2.,4.) -- (1.,4.) -- cycle;
\fill[color=qqqqff,fill=qqqqff,fill opacity=0.1] (2.,2.) -- (3.,2.) -- (3.,3.) -- (2.,3.) -- cycle;
\fill[color=qqqqff,fill=qqqqff,fill opacity=0.1] (3.,0.) -- (4.,0.) -- (4.,1.) -- (3.,1.) -- cycle;
\fill[color=qqqqff,fill=qqqqff,fill opacity=0.1] (2.,1.) -- (3.,1.) -- (3.,2.) -- (2.,2.) -- cycle;
\fill[color=qqqqff,fill=qqqqff,fill opacity=0.1] (2.,0.) -- (3.,0.) -- (3.,1.) -- (2.,1.) -- cycle;
\fill[color=qqqqff,fill=qqqqff,fill opacity=0.1] (0.,3.) -- (1.,3.) -- (1.,4.) -- (0.,4.) -- cycle;
\fill[color=qqqqff,fill=qqqqff,fill opacity=0.1] (-1.,3.) -- (0.,3.) -- (0.,4.) -- (-1.,4.) -- cycle;
\fill[color=qqqqff,fill=qqqqff,fill opacity=0.1] (-2.,2.) -- (-1.,2.) -- (-1.,3.) -- (-2.,3.) -- cycle;
\fill[color=qqqqff,fill=qqqqff,fill opacity=0.1] (-1.,2.) -- (0.,2.) -- (0.,3.) -- (-1.,3.) -- cycle;
\fill[color=qqqqff,fill=qqqqff,fill opacity=0.1] (0.,2.) -- (1.,2.) -- (1.,3.) -- (0.,3.) -- cycle;
\fill[color=qqqqff,fill=qqqqff,fill opacity=0.1] (1.,2.) -- (2.,2.) -- (2.,3.) -- (1.,3.) -- cycle;
\fill[color=qqqqff,fill=qqqqff,fill opacity=0.1] (-2.,1.) -- (-1.,1.) -- (-1.,2.) -- (-2.,2.) -- cycle;
\fill[color=qqqqff,fill=qqqqff,fill opacity=0.1] (-1.,1.) -- (0.,1.) -- (0.,2.) -- (-1.,2.) -- cycle;
\fill[color=qqqqff,fill=qqqqff,fill opacity=0.1] (0.,1.) -- (1.,1.) -- (1.,2.) -- (0.,2.) -- cycle;
\fill[color=qqqqff,fill=qqqqff,fill opacity=0.1] (1.,1.) -- (2.,1.) -- (2.,2.) -- (1.,2.) -- cycle;
\fill[color=qqqqff,fill=qqqqff,fill opacity=0.1] (-2.,0.) -- (-1.,0.) -- (-1.,1.) -- (-2.,1.) -- cycle;
\fill[color=qqqqff,fill=qqqqff,fill opacity=0.1] (-1.,0.) -- (0.,0.) -- (0.,1.) -- (-1.,1.) -- cycle;
\fill[color=qqqqff,fill=qqqqff,fill opacity=0.1] (0.,0.) -- (1.,0.) -- (1.,1.) -- (0.,1.) -- cycle;
\fill[color=qqqqff,fill=qqqqff,fill opacity=0.1] (1.,0.) -- (2.,0.) -- (2.,1.) -- (1.,1.) -- cycle;
\fill[color=qqqqff,fill=qqqqff,fill opacity=0.1] (-5.,0.) -- (-4.,0.) -- (-4.,1.) -- (-5.,1.) -- cycle;
\fill[color=qqqqff,fill=qqqqff,fill opacity=0.1] (-4.,1.) -- (-3.,1.) -- (-3.,2.) -- (-4.,2.) -- cycle;
\fill[color=qqqqff,fill=qqqqff,fill opacity=0.1] (-5.,1.) -- (-4.,1.) -- (-4.,2.) -- (-5.,2.) -- cycle;
\fill[color=qqqqff,fill=qqqqff,fill opacity=0.1] (-4.,2.) -- (-3.,2.) -- (-3.,3.) -- (-4.,3.) -- cycle;
\fill[color=qqqqff,fill=qqqqff,fill opacity=0.1] (-3.,3.) -- (-2.,3.) -- (-2.,4.) -- (-3.,4.) -- cycle;
\fill[color=qqqqff,fill=qqqqff,fill opacity=0.1] (-4.,3.) -- (-3.,3.) -- (-3.,4.) -- (-4.,4.) -- cycle;
\fill[color=qqqqff,fill=qqqqff,fill opacity=0.1] (-3.,4.) -- (-2.,4.) -- (-2.,5.) -- (-3.,5.) -- cycle;
\fill[color=qqqqff,fill=qqqqff,fill opacity=0.1] (-2.,5.) -- (-1.,5.) -- (-1.,6.) -- (-2.,6.) -- cycle;
\fill[color=qqqqff,fill=qqqqff,fill opacity=0.1] (-2.,6.) -- (-1.,6.) -- (-1.,7.) -- (-2.,7.) -- cycle;
\fill[color=qqqqff,fill=qqqqff,fill opacity=0.1] (-2.,7.) -- (-1.,7.) -- (-1.,8.) -- (-2.,8.) -- cycle;
\fill[color=qqqqff,fill=qqqqff,fill opacity=0.1] (-1.,6.) -- (0.,6.) -- (0.,7.) -- (-1.,7.) -- cycle;
\fill[color=qqqqff,fill=qqqqff,fill opacity=0.1] (-1.,7.) -- (0.,7.) -- (0.,8.) -- (-1.,8.) -- cycle;
\fill[color=qqqqff,fill=qqqqff,fill opacity=0.1] (-1.,8.) -- (0.,8.) -- (0.,9.) -- (-1.,9.) -- cycle;
\fill[color=qqqqff,fill=qqqqff,fill opacity=0.1] (0.,8.) -- (1.,8.) -- (1.,9.) -- (0.,9.) -- cycle;
\fill[color=qqqqff,fill=qqqqff,fill opacity=0.1] (0.,7.) -- (1.,7.) -- (1.,8.) -- (0.,8.) -- cycle;
\fill[color=qqqqff,fill=qqqqff,fill opacity=0.1] (0.,6.) -- (1.,6.) -- (1.,7.) -- (0.,7.) -- cycle;
\fill[color=qqqqff,fill=qqqqff,fill opacity=0.1] (1.,7.) -- (2.,7.) -- (2.,8.) -- (1.,8.) -- cycle;
\fill[color=qqqqff,fill=qqqqff,fill opacity=0.1] (1.,6.) -- (2.,6.) -- (2.,7.) -- (1.,7.) -- cycle;
\fill[color=qqqqff,fill=qqqqff,fill opacity=0.1] (1.,5.) -- (2.,5.) -- (2.,6.) -- (1.,6.) -- cycle;
\fill[color=qqqqff,fill=qqqqff,fill opacity=0.1] (2.,5.) -- (3.,5.) -- (3.,6.) -- (2.,6.) -- cycle;
\fill[color=qqqqff,fill=qqqqff,fill opacity=0.1] (2.,4.) -- (3.,4.) -- (3.,5.) -- (2.,5.) -- cycle;
\fill[color=qqqqff,fill=qqqqff,fill opacity=0.1] (2.,3.) -- (3.,3.) -- (3.,4.) -- (2.,4.) -- cycle;
\fill[color=qqqqff,fill=qqqqff,fill opacity=0.1] (3.,3.) -- (4.,3.) -- (4.,4.) -- (3.,4.) -- cycle;
\fill[color=qqqqff,fill=qqqqff,fill opacity=0.1] (3.,2.) -- (4.,2.) -- (4.,3.) -- (3.,3.) -- cycle;
\fill[color=qqqqff,fill=qqqqff,fill opacity=0.1] (3.,1.) -- (4.,1.) -- (4.,2.) -- (3.,2.) -- cycle;
\fill[color=qqqqff,fill=qqqqff,fill opacity=0.1] (4.,1.) -- (5.,1.) -- (5.,2.) -- (4.,2.) -- cycle;
\fill[color=qqqqff,fill=qqqqff,fill opacity=0.1] (4.,0.) -- (5.,0.) -- (5.,1.) -- (4.,1.) -- cycle;
\fill[color=qqqqff,fill=qqqqff,fill opacity=0.1] (-3.,5.) -- (-2.,5.) -- (-2.,6.) -- (-3.,6.) -- cycle;
\draw [color=ffqqqq] (-5.,0.)-- (5.,0.);
\draw [color=ffqqqq] (5.,0.)-- (0.,8.66025403784);
\draw [color=ffqqqq] (0.,8.66025403784)-- (-5.,0.);
\draw [color=qqqqff] (-4.,0.)-- (-3.,0.);
\draw [color=qqqqff] (-3.,0.)-- (-3.,1.);
\draw [color=qqqqff] (-3.,1.)-- (-4.,1.);
\draw [color=qqqqff] (-4.,1.)-- (-4.,0.);
\draw [color=qqqqff] (-3.,0.)-- (-2.,0.);
\draw [color=qqqqff] (-2.,0.)-- (-2.,1.);
\draw [color=qqqqff] (-2.,1.)-- (-3.,1.);
\draw [color=qqqqff] (-3.,1.)-- (-3.,0.);
\draw [color=qqqqff] (-3.,1.)-- (-2.,1.);
\draw [color=qqqqff] (-2.,1.)-- (-2.,2.);
\draw [color=qqqqff] (-2.,2.)-- (-3.,2.);
\draw [color=qqqqff] (-3.,2.)-- (-3.,1.);
\draw [color=qqqqff] (-3.,2.)-- (-2.,2.);
\draw [color=qqqqff] (-2.,2.)-- (-2.,3.);
\draw [color=qqqqff] (-2.,3.)-- (-3.,3.);
\draw [color=qqqqff] (-3.,3.)-- (-3.,2.);
\draw [color=qqqqff] (-2.,3.)-- (-1.,3.);
\draw [color=qqqqff] (-1.,3.)-- (-1.,4.);
\draw [color=qqqqff] (-1.,4.)-- (-2.,4.);
\draw [color=qqqqff] (-2.,4.)-- (-2.,3.);
\draw [color=qqqqff] (-2.,4.)-- (-1.,4.);
\draw [color=qqqqff] (-1.,4.)-- (-1.,5.);
\draw [color=qqqqff] (-1.,5.)-- (-2.,5.);
\draw [color=qqqqff] (-2.,5.)-- (-2.,4.);
\draw [color=qqqqff] (-1.,5.)-- (0.,5.);
\draw [color=qqqqff] (0.,5.)-- (0.,6.);
\draw [color=qqqqff] (0.,6.)-- (-1.,6.);
\draw [color=qqqqff] (-1.,6.)-- (-1.,5.);
\draw [color=qqqqff] (0.,5.)-- (1.,5.);
\draw [color=qqqqff] (1.,5.)-- (1.,6.);
\draw [color=qqqqff] (1.,6.)-- (0.,6.);
\draw [color=qqqqff] (0.,6.)-- (0.,5.);
\draw [color=qqqqff] (0.,4.)-- (1.,4.);
\draw [color=qqqqff] (1.,4.)-- (1.,5.);
\draw [color=qqqqff] (1.,5.)-- (0.,5.);
\draw [color=qqqqff] (0.,5.)-- (0.,4.);
\draw [color=qqqqff] (-1.,4.)-- (0.,4.);
\draw [color=qqqqff] (0.,4.)-- (0.,5.);
\draw [color=qqqqff] (0.,5.)-- (-1.,5.);
\draw [color=qqqqff] (-1.,5.)-- (-1.,4.);
\draw [color=qqqqff] (1.,4.)-- (2.,4.);
\draw [color=qqqqff] (2.,4.)-- (2.,5.);
\draw [color=qqqqff] (2.,5.)-- (1.,5.);
\draw [color=qqqqff] (1.,5.)-- (1.,4.);
\draw [color=qqqqff] (1.,3.)-- (2.,3.);
\draw [color=qqqqff] (2.,3.)-- (2.,4.);
\draw [color=qqqqff] (2.,4.)-- (1.,4.);
\draw [color=qqqqff] (1.,4.)-- (1.,3.);
\draw [color=qqqqff] (2.,2.)-- (3.,2.);
\draw [color=qqqqff] (3.,2.)-- (3.,3.);
\draw [color=qqqqff] (3.,3.)-- (2.,3.);
\draw [color=qqqqff] (2.,3.)-- (2.,2.);
\draw [color=qqqqff] (3.,0.)-- (4.,0.);
\draw [color=qqqqff] (4.,0.)-- (4.,1.);
\draw [color=qqqqff] (4.,1.)-- (3.,1.);
\draw [color=qqqqff] (3.,1.)-- (3.,0.);
\draw [color=qqqqff] (2.,1.)-- (3.,1.);
\draw [color=qqqqff] (3.,1.)-- (3.,2.);
\draw [color=qqqqff] (3.,2.)-- (2.,2.);
\draw [color=qqqqff] (2.,2.)-- (2.,1.);
\draw [color=qqqqff] (2.,0.)-- (3.,0.);
\draw [color=qqqqff] (3.,0.)-- (3.,1.);
\draw [color=qqqqff] (3.,1.)-- (2.,1.);
\draw [color=qqqqff] (2.,1.)-- (2.,0.);
\draw [color=qqqqff] (0.,3.)-- (1.,3.);
\draw [color=qqqqff] (1.,3.)-- (1.,4.);
\draw [color=qqqqff] (1.,4.)-- (0.,4.);
\draw [color=qqqqff] (0.,4.)-- (0.,3.);
\draw [color=qqqqff] (-1.,3.)-- (0.,3.);
\draw [color=qqqqff] (0.,3.)-- (0.,4.);
\draw [color=qqqqff] (0.,4.)-- (-1.,4.);
\draw [color=qqqqff] (-1.,4.)-- (-1.,3.);
\draw [color=qqqqff] (-2.,2.)-- (-1.,2.);
\draw [color=qqqqff] (-1.,2.)-- (-1.,3.);
\draw [color=qqqqff] (-1.,3.)-- (-2.,3.);
\draw [color=qqqqff] (-2.,3.)-- (-2.,2.);
\draw [color=qqqqff] (-1.,2.)-- (0.,2.);
\draw [color=qqqqff] (0.,2.)-- (0.,3.);
\draw [color=qqqqff] (0.,3.)-- (-1.,3.);
\draw [color=qqqqff] (-1.,3.)-- (-1.,2.);
\draw [color=qqqqff] (0.,2.)-- (1.,2.);
\draw [color=qqqqff] (1.,2.)-- (1.,3.);
\draw [color=qqqqff] (1.,3.)-- (0.,3.);
\draw [color=qqqqff] (0.,3.)-- (0.,2.);
\draw [color=qqqqff] (1.,2.)-- (2.,2.);
\draw [color=qqqqff] (2.,2.)-- (2.,3.);
\draw [color=qqqqff] (2.,3.)-- (1.,3.);
\draw [color=qqqqff] (1.,3.)-- (1.,2.);
\draw [color=qqqqff] (-2.,1.)-- (-1.,1.);
\draw [color=qqqqff] (-1.,1.)-- (-1.,2.);
\draw [color=qqqqff] (-1.,2.)-- (-2.,2.);
\draw [color=qqqqff] (-2.,2.)-- (-2.,1.);
\draw [color=qqqqff] (-1.,1.)-- (0.,1.);
\draw [color=qqqqff] (0.,1.)-- (0.,2.);
\draw [color=qqqqff] (0.,2.)-- (-1.,2.);
\draw [color=qqqqff] (-1.,2.)-- (-1.,1.);
\draw [color=qqqqff] (0.,1.)-- (1.,1.);
\draw [color=qqqqff] (1.,1.)-- (1.,2.);
\draw [color=qqqqff] (1.,2.)-- (0.,2.);
\draw [color=qqqqff] (0.,2.)-- (0.,1.);
\draw [color=qqqqff] (1.,1.)-- (2.,1.);
\draw [color=qqqqff] (2.,1.)-- (2.,2.);
\draw [color=qqqqff] (2.,2.)-- (1.,2.);
\draw [color=qqqqff] (1.,2.)-- (1.,1.);
\draw [color=qqqqff] (-2.,0.)-- (-1.,0.);
\draw [color=qqqqff] (-1.,0.)-- (-1.,1.);
\draw [color=qqqqff] (-1.,1.)-- (-2.,1.);
\draw [color=qqqqff] (-2.,1.)-- (-2.,0.);
\draw [color=qqqqff] (-1.,0.)-- (0.,0.);
\draw [color=qqqqff] (0.,0.)-- (0.,1.);
\draw [color=qqqqff] (0.,1.)-- (-1.,1.);
\draw [color=qqqqff] (-1.,1.)-- (-1.,0.);
\draw [color=qqqqff] (0.,0.)-- (1.,0.);
\draw [color=qqqqff] (1.,0.)-- (1.,1.);
\draw [color=qqqqff] (1.,1.)-- (0.,1.);
\draw [color=qqqqff] (0.,1.)-- (0.,0.);
\draw [color=qqqqff] (1.,0.)-- (2.,0.);
\draw [color=qqqqff] (2.,0.)-- (2.,1.);
\draw [color=qqqqff] (2.,1.)-- (1.,1.);
\draw [color=qqqqff] (1.,1.)-- (1.,0.);
\draw [color=qqqqff] (-5.,0.)-- (-4.,0.);
\draw [color=qqqqff] (-4.,0.)-- (-4.,1.);
\draw [color=qqqqff] (-4.,1.)-- (-5.,1.);
\draw [color=qqqqff] (-5.,1.)-- (-5.,0.);
\draw [color=qqqqff] (-4.,1.)-- (-3.,1.);
\draw [color=qqqqff] (-3.,1.)-- (-3.,2.);
\draw [color=qqqqff] (-3.,2.)-- (-4.,2.);
\draw [color=qqqqff] (-4.,2.)-- (-4.,1.);
\draw [color=qqqqff] (-5.,1.)-- (-4.,1.);
\draw [color=qqqqff] (-4.,1.)-- (-4.,2.);
\draw [color=qqqqff] (-4.,2.)-- (-5.,2.);
\draw [color=qqqqff] (-5.,2.)-- (-5.,1.);
\draw [color=qqqqff] (-4.,2.)-- (-3.,2.);
\draw [color=qqqqff] (-3.,2.)-- (-3.,3.);
\draw [color=qqqqff] (-3.,3.)-- (-4.,3.);
\draw [color=qqqqff] (-4.,3.)-- (-4.,2.);
\draw [color=qqqqff] (-3.,3.)-- (-2.,3.);
\draw [color=qqqqff] (-2.,3.)-- (-2.,4.);
\draw [color=qqqqff] (-2.,4.)-- (-3.,4.);
\draw [color=qqqqff] (-3.,4.)-- (-3.,3.);
\draw [color=qqqqff] (-4.,3.)-- (-3.,3.);
\draw [color=qqqqff] (-3.,3.)-- (-3.,4.);
\draw [color=qqqqff] (-3.,4.)-- (-4.,4.);
\draw [color=qqqqff] (-4.,4.)-- (-4.,3.);
\draw [color=qqqqff] (-3.,4.)-- (-2.,4.);
\draw [color=qqqqff] (-2.,4.)-- (-2.,5.);
\draw [color=qqqqff] (-2.,5.)-- (-3.,5.);
\draw [color=qqqqff] (-3.,5.)-- (-3.,4.);
\draw [color=qqqqff] (-2.,5.)-- (-1.,5.);
\draw [color=qqqqff] (-1.,5.)-- (-1.,6.);
\draw [color=qqqqff] (-1.,6.)-- (-2.,6.);
\draw [color=qqqqff] (-2.,6.)-- (-2.,5.);
\draw [color=qqqqff] (-2.,6.)-- (-1.,6.);
\draw [color=qqqqff] (-1.,6.)-- (-1.,7.);
\draw [color=qqqqff] (-1.,7.)-- (-2.,7.);
\draw [color=qqqqff] (-2.,7.)-- (-2.,6.);
\draw [color=qqqqff] (-2.,7.)-- (-1.,7.);
\draw [color=qqqqff] (-1.,7.)-- (-1.,8.);
\draw [color=qqqqff] (-1.,8.)-- (-2.,8.);
\draw [color=qqqqff] (-2.,8.)-- (-2.,7.);
\draw [color=qqqqff] (-1.,6.)-- (0.,6.);
\draw [color=qqqqff] (0.,6.)-- (0.,7.);
\draw [color=qqqqff] (0.,7.)-- (-1.,7.);
\draw [color=qqqqff] (-1.,7.)-- (-1.,6.);
\draw [color=qqqqff] (-1.,7.)-- (0.,7.);
\draw [color=qqqqff] (0.,7.)-- (0.,8.);
\draw [color=qqqqff] (0.,8.)-- (-1.,8.);
\draw [color=qqqqff] (-1.,8.)-- (-1.,7.);
\draw [color=qqqqff] (-1.,8.)-- (0.,8.);
\draw [color=qqqqff] (0.,8.)-- (0.,9.);
\draw [color=qqqqff] (0.,9.)-- (-1.,9.);
\draw [color=qqqqff] (-1.,9.)-- (-1.,8.);
\draw [color=qqqqff] (0.,8.)-- (1.,8.);
\draw [color=qqqqff] (1.,8.)-- (1.,9.);
\draw [color=qqqqff] (1.,9.)-- (0.,9.);
\draw [color=qqqqff] (0.,9.)-- (0.,8.);
\draw [color=qqqqff] (0.,7.)-- (1.,7.);
\draw [color=qqqqff] (1.,7.)-- (1.,8.);
\draw [color=qqqqff] (1.,8.)-- (0.,8.);
\draw [color=qqqqff] (0.,8.)-- (0.,7.);
\draw [color=qqqqff] (0.,6.)-- (1.,6.);
\draw [color=qqqqff] (1.,6.)-- (1.,7.);
\draw [color=qqqqff] (1.,7.)-- (0.,7.);
\draw [color=qqqqff] (0.,7.)-- (0.,6.);
\draw [color=qqqqff] (1.,7.)-- (2.,7.);
\draw [color=qqqqff] (2.,7.)-- (2.,8.);
\draw [color=qqqqff] (2.,8.)-- (1.,8.);
\draw [color=qqqqff] (1.,8.)-- (1.,7.);
\draw [color=qqqqff] (1.,6.)-- (2.,6.);
\draw [color=qqqqff] (2.,6.)-- (2.,7.);
\draw [color=qqqqff] (2.,7.)-- (1.,7.);
\draw [color=qqqqff] (1.,7.)-- (1.,6.);
\draw [color=qqqqff] (1.,5.)-- (2.,5.);
\draw [color=qqqqff] (2.,5.)-- (2.,6.);
\draw [color=qqqqff] (2.,6.)-- (1.,6.);
\draw [color=qqqqff] (1.,6.)-- (1.,5.);
\draw [color=qqqqff] (2.,5.)-- (3.,5.);
\draw [color=qqqqff] (3.,5.)-- (3.,6.);
\draw [color=qqqqff] (3.,6.)-- (2.,6.);
\draw [color=qqqqff] (2.,6.)-- (2.,5.);
\draw [color=qqqqff] (2.,4.)-- (3.,4.);
\draw [color=qqqqff] (3.,4.)-- (3.,5.);
\draw [color=qqqqff] (3.,5.)-- (2.,5.);
\draw [color=qqqqff] (2.,5.)-- (2.,4.);
\draw [color=qqqqff] (2.,3.)-- (3.,3.);
\draw [color=qqqqff] (3.,3.)-- (3.,4.);
\draw [color=qqqqff] (3.,4.)-- (2.,4.);
\draw [color=qqqqff] (2.,4.)-- (2.,3.);
\draw [color=qqqqff] (3.,3.)-- (4.,3.);
\draw [color=qqqqff] (4.,3.)-- (4.,4.);
\draw [color=qqqqff] (4.,4.)-- (3.,4.);
\draw [color=qqqqff] (3.,4.)-- (3.,3.);
\draw [color=qqqqff] (3.,2.)-- (4.,2.);
\draw [color=qqqqff] (4.,2.)-- (4.,3.);
\draw [color=qqqqff] (4.,3.)-- (3.,3.);
\draw [color=qqqqff] (3.,3.)-- (3.,2.);
\draw [color=qqqqff] (3.,1.)-- (4.,1.);
\draw [color=qqqqff] (4.,1.)-- (4.,2.);
\draw [color=qqqqff] (4.,2.)-- (3.,2.);
\draw [color=qqqqff] (3.,2.)-- (3.,1.);
\draw [color=qqqqff] (4.,1.)-- (5.,1.);
\draw [color=qqqqff] (5.,1.)-- (5.,2.);
\draw [color=qqqqff] (5.,2.)-- (4.,2.);
\draw [color=qqqqff] (4.,2.)-- (4.,1.);
\draw [color=qqqqff] (4.,0.)-- (5.,0.);
\draw [color=qqqqff] (5.,0.)-- (5.,1.);
\draw [color=qqqqff] (5.,1.)-- (4.,1.);
\draw [color=qqqqff] (4.,1.)-- (4.,0.);
\draw [color=qqqqff] (-3.,5.)-- (-2.,5.);
\draw [color=qqqqff] (-2.,5.)-- (-2.,6.);
\draw [color=qqqqff] (-2.,6.)-- (-3.,6.);
\draw [color=qqqqff] (-3.,6.)-- (-3.,5.);

\draw[color=ffqqqq] (0,-2) node {\text{10 unidades}};
\end{tikzpicture}
%\caption{Unidades cuadradas contenidas en el triángulo.}\label{triangulounidadfuera}
\end{center}
\end{figure}


Es decir, nos hemos aproximado al área de la siguiente manera si $u^2$ representa una unidad cuadrada:

$$ 30 u^2 < \text{área del triángulo} < 58 u^2$$


¿Cómo podemos mejorar nuestra aproximación?\\

Una manera de hacerlo, es calcular si el triángulo contiene (y está contenido) en cuadros más pequeños; es decir, veamos qué pasa si tratamos de mejorar nuestro cálculo al mejorar con cuadrados de área un cuarto de unidad cuadrada: \\

En la figura~\ref{triangulocuartounidaddentro} tenemos los cuadrados de área un cuarto de unidad cuadrada contenidos en el triángulo. \\

\begin{figure}[!h]
\begin{center}
\definecolor{ffqqqq}{rgb}{1.,0.,0.}
\definecolor{zzttqq}{rgb}{0.6,0.2,0.}
\definecolor{qqqqff}{rgb}{0.,0.,1.}
\definecolor{ffqqqq}{rgb}{1.,0.,0.}
\definecolor{zzttqq}{rgb}{0.6,0.2,0.}
\definecolor{cqcqcq}{rgb}{0.752941176471,0.752941176471,0.752941176471}
\definecolor{qqzzqq}{rgb}{0.,0.6,0.}

\begin{tikzpicture}[x=0.5cm,y=0.5cm]
\fill[color=zzttqq,fill=zzttqq,fill opacity=0.1] (-5.,0.) -- (5.,0.) -- (0.,8.66025403784) -- cycle;
\draw [color=ffqqqq] (-5.,0.)-- (5.,0.);
\draw [color=ffqqqq] (5.,0.)-- (0.,8.66025403784);
\draw [color=ffqqqq] (0.,8.66025403784)-- (-5.,0.);

\draw [dash pattern=on 2pt off 2pt] (-5,-1) -- (-5,10);
\draw [dash pattern=on 2pt off 2pt] (-4,-1) -- (-4,10);
\draw [dash pattern=on 2pt off 2pt] (-3,-1) -- (-3,10);
\draw [dash pattern=on 2pt off 2pt] (-2,-1) -- (-2,10);
\draw [dash pattern=on 2pt off 2pt] (-1,-1) -- (-1,10);
\draw [dash pattern=on 2pt off 2pt] (0,-1) -- (0,10);
\draw [dash pattern=on 2pt off 2pt] (1,-1) -- (1,10);
\draw [dash pattern=on 2pt off 2pt] (2,-1) -- (2,10);
\draw [dash pattern=on 2pt off 2pt] (3,-1) -- (3,10);
\draw [dash pattern=on 2pt off 2pt] (4,-1) -- (4,10);
\draw [dash pattern=on 2pt off 2pt] (5,-1) -- (5,10);

\draw [dash pattern=on 2pt off 2pt] (-5,0) -- (5,0);
\draw [dash pattern=on 2pt off 2pt] (-5,1) -- (5,1);
\draw [dash pattern=on 2pt off 2pt] (-5,2) -- (5,2);
\draw [dash pattern=on 2pt off 2pt] (-5,3) -- (5,3);
\draw [dash pattern=on 2pt off 2pt] (-5,4) -- (5,4);
\draw [dash pattern=on 2pt off 2pt] (-5,5) -- (5,5);
\draw [dash pattern=on 2pt off 2pt] (-5,6) -- (5,6);
\draw [dash pattern=on 2pt off 2pt] (-5,7) -- (5,7);
\draw [dash pattern=on 2pt off 2pt] (-5,8) -- (5,8);
\draw [dash pattern=on 2pt off 2pt] (-5,9) -- (5,9);
\draw [dash pattern=on 2pt off 2pt] (5,-1) -- (5,10);

\fill[color=zzttqq,fill=zzttqq,fill opacity=0.1] (-5.,0.) -- (5.,0.) -- (0.,8.66025403784) -- cycle;

\fill[color=qqzzqq,fill=qqzzqq,fill opacity=.5] (-4.5,0) -- (-4.5,0.5) -- (-4,0.5) -- (-4,1.5) -- (-3.5,1.5) -- (-3.5,2.5) -- (-3,2.5) -- (-3,3) -- (-2.5,3) -- (-2.5,4) -- (-2,4) -- (-2,5) -- (-1.5,5) -- (-1.5,6) -- (-1,6) -- (-1,6.5) -- (-0.5,6.5) -- (-0.5,7.5) -- (0.5,7.5) -- (0.5,6.5) -- (1,6.5) -- (1,6) -- (1.5,6) -- (1.5,5) -- (2,5) -- (2,4) -- (2.5,4) -- (2.5,3) -- (3,3) -- (3,2.5) -- (3.5,2.5) -- (3.5,1.5) -- (4,1.5) -- (4,0.5) -- (4.5,0.5) -- (4.5,0) -- cycle;
\draw [dash pattern=on 2pt off 2pt] (-4.5,0.5) -- (4.5,0.5);
\draw [dash pattern=on 2pt off 2pt] (-4,1.5) -- (4,1.5);
\draw [dash pattern=on 2pt off 2pt] (-3.5,2.5) -- (3.5,2.5);
\draw [dash pattern=on 2pt off 2pt] (-2.5,3.5) -- (2.5,3.5);
\draw [dash pattern=on 2pt off 2pt] (-2,4.5) -- (2,4.5);
\draw [dash pattern=on 2pt off 2pt] (-1.5,5.5) -- (1.5,5.5);
\draw [dash pattern=on 2pt off 2pt] (-1,6.5) -- (1,6.5);
\draw [dash pattern=on 2pt off 2pt] (-0.5,7.5) -- (0.5,7.5);

\draw [dash pattern=on 2pt off 2pt] (-4.5,0) -- (-4.5,0.5);
\draw [dash pattern=on 2pt off 2pt] (-3.5,0) -- (-3.5,2.5);
\draw [dash pattern=on 2pt off 2pt] (-2.5,0) -- (-2.5,4);
\draw [dash pattern=on 2pt off 2pt] (-1.5,0) -- (-1.5,6);
\draw [dash pattern=on 2pt off 2pt] (-0.5,0) -- (-0.5,7.5);
\draw [dash pattern=on 2pt off 2pt] (0.5,0) -- (0.5,7.5);
\draw [dash pattern=on 2pt off 2pt] (1.5,0) -- (1.5,6);
\draw [dash pattern=on 2pt off 2pt] (2.5,0) -- (2.5,4);
\draw [dash pattern=on 2pt off 2pt] (3.5,0) -- (3.5,2.5);
\draw [dash pattern=on 2pt off 2pt] (4.5,0) -- (4.5,0.5);

\fill[color=qqqqff,fill=qqqqff,fill opacity=0.5] (8,2) -- (9,2) -- (9,3) -- (8,3) -- cycle;
\draw[color=qqqqff] (12,2.5) node {\text{1 unidad cuadrada}};
\fill[color=qqzzqq,fill=qqzzqq,fill opacity=0.5] (8,0) -- (8.5,0) -- (8.5,0.5) -- (8,0.5) -- cycle;
\draw[color=qqzzqq] (12,0) node {\text{1/4 unidad cuadrada}};

\draw[color=ffqqqq] (0,-2) node {\text{10 unidades}};
\end{tikzpicture}
\caption{Cuarto de unidades cuadradas contenidas en el triángulo.}\label{triangulocuartounidaddentro}
\end{center}
\end{figure}



En la figura~\ref{triangulocuartounidadfuera} tenemos los cuadrados de área un cuarto de unidad cuadrada que contienen al triángulo. \\

\begin{figure}[!h]
\begin{center}
\definecolor{ffqqqq}{rgb}{1.,0.,0.}
\definecolor{zzttqq}{rgb}{0.6,0.2,0.}
\definecolor{qqqqff}{rgb}{0.,0.,1.}
\definecolor{ffqqqq}{rgb}{1.,0.,0.}
\definecolor{zzttqq}{rgb}{0.6,0.2,0.}
\definecolor{cqcqcq}{rgb}{0.752941176471,0.752941176471,0.752941176471}
\definecolor{qqzzqq}{rgb}{0.,0.6,0.}

\begin{tikzpicture}[x=0.5cm,y=0.5cm]
\fill[color=zzttqq,fill=zzttqq,fill opacity=0.1] (-5.,0.) -- (5.,0.) -- (0.,8.66025403784) -- cycle;
\draw [color=ffqqqq] (-5.,0.)-- (5.,0.);
\draw [color=ffqqqq] (5.,0.)-- (0.,8.66025403784);
\draw [color=ffqqqq] (0.,8.66025403784)-- (-5.,0.);

\draw [dash pattern=on 2pt off 2pt] (-5,-1) -- (-5,10);
\draw [dash pattern=on 2pt off 2pt] (-4,-1) -- (-4,10);
\draw [dash pattern=on 2pt off 2pt] (-3,-1) -- (-3,10);
\draw [dash pattern=on 2pt off 2pt] (-2,-1) -- (-2,10);
\draw [dash pattern=on 2pt off 2pt] (-1,-1) -- (-1,10);
\draw [dash pattern=on 2pt off 2pt] (0,-1) -- (0,10);
\draw [dash pattern=on 2pt off 2pt] (1,-1) -- (1,10);
\draw [dash pattern=on 2pt off 2pt] (2,-1) -- (2,10);
\draw [dash pattern=on 2pt off 2pt] (3,-1) -- (3,10);
\draw [dash pattern=on 2pt off 2pt] (4,-1) -- (4,10);
\draw [dash pattern=on 2pt off 2pt] (5,-1) -- (5,10);

\draw [dash pattern=on 2pt off 2pt] (-5,0) -- (5,0);
\draw [dash pattern=on 2pt off 2pt] (-5,1) -- (5,1);
\draw [dash pattern=on 2pt off 2pt] (-5,2) -- (5,2);
\draw [dash pattern=on 2pt off 2pt] (-5,3) -- (5,3);
\draw [dash pattern=on 2pt off 2pt] (-5,4) -- (5,4);
\draw [dash pattern=on 2pt off 2pt] (-5,5) -- (5,5);
\draw [dash pattern=on 2pt off 2pt] (-5,6) -- (5,6);
\draw [dash pattern=on 2pt off 2pt] (-5,7) -- (5,7);
\draw [dash pattern=on 2pt off 2pt] (-5,8) -- (5,8);
\draw [dash pattern=on 2pt off 2pt] (-5,9) -- (5,9);
\draw [dash pattern=on 2pt off 2pt] (5,-1) -- (5,10);

\fill[color=zzttqq,fill=zzttqq,fill opacity=0.1] (-5.,0.) -- (5.,0.) -- (0.,8.66025403784) -- cycle;

\fill[color=qqzzqq,fill=qqzzqq,fill opacity=.5] (-5,0) -- (-5,1) -- (-4.5,1) -- (-4.5,2) -- (-4,2) -- (-4,3) -- (-3.5,3) -- (-3.5,3.5) -- (-3,3.5) -- (-3,4.5) -- (-2.5,4.5) -- (-2.5,5.5) -- (-2,5.5) -- (-2,6.5) -- (-1.5,6.5) -- (-1.5,7) -- (-1,7) -- (-1,8) -- (-0.5,8) -- (-0.5,9) -- (0.5,9) -- (0.5,8) -- (1,8) -- (1,7) -- (1.5,7) -- (1.5,6.5) -- (2,6.5) -- (2,5.5) -- (2.5,5.5) --  (2.5,4.5) -- (3,4.5) --(3,3.5) -- (3.5,3.5) --(3.5,3) --  (4,3) -- (4,2) -- (4.5,2) -- (4.5,1) -- (5,1) -- (5,0)-- cycle;

\draw [dash pattern=on 2pt off 2pt] (-5,0.5) -- (5,0.5);
\draw [dash pattern=on 2pt off 2pt] (-4.5,1.5) -- (4.5,1.5);
\draw [dash pattern=on 2pt off 2pt] (-4,2.5) -- (4,2.5);
\draw [dash pattern=on 2pt off 2pt] (-3.5,3.5) -- (3.5,3.5);
\draw [dash pattern=on 2pt off 2pt] (-3,4.5) -- (3,4.5);
\draw [dash pattern=on 2pt off 2pt] (-2.5,5.5) -- (2.5,5.5);
\draw [dash pattern=on 2pt off 2pt] (-2,6.5) -- (2,6.5);
\draw [dash pattern=on 2pt off 2pt] (-1,7.5) -- (1,7.5);
\draw [dash pattern=on 2pt off 2pt] (-0.5,8.5) -- (0.5,8.5);

\draw [dash pattern=on 2pt off 2pt] (-4.5,0) -- (-4.5,2);
\draw [dash pattern=on 2pt off 2pt] (-3.5,0) -- (-3.5,3.5);
\draw [dash pattern=on 2pt off 2pt] (-2.5,0) -- (-2.5,5.5);
\draw [dash pattern=on 2pt off 2pt] (-1.5,0) -- (-1.5,7);
\draw [dash pattern=on 2pt off 2pt] (-0.5,0) -- (-0.5,9);
\draw [dash pattern=on 2pt off 2pt] (0.5,0) -- (0.5,9);
\draw [dash pattern=on 2pt off 2pt] (1.5,0) -- (1.5,7);
\draw [dash pattern=on 2pt off 2pt] (2.5,0) -- (2.5,5.5);
\draw [dash pattern=on 2pt off 2pt] (3.5,0) -- (3.5,3.5);
\draw [dash pattern=on 2pt off 2pt] (4.5,0) -- (4.5,2);

\fill[color=qqqqff,fill=qqqqff,fill opacity=0.5] (8,2) -- (9,2) -- (9,3) -- (8,3) -- cycle;
\draw[color=qqqqff] (12,2.5) node {\text{1 unidad cuadrada}};
\fill[color=qqzzqq,fill=qqzzqq,fill opacity=0.5] (8,0) -- (8.5,0) -- (8.5,0.5) -- (8,0.5) -- cycle;
\draw[color=qqzzqq] (12,0) node {\text{1/4 unidad cuadrada}};

\draw[color=ffqqqq] (0,-2) node {\text{10 unidades}};
\end{tikzpicture}
\caption{Cuarto de unidades cuadradas que contienen al triángulo.}\label{triangulocuartounidadfuera}
\end{center}
\end{figure}

Así, podemos mejorar nuestra aproximación como sigue:

$$ 146 \cdot \frac{u^2}{4}< \text{área del triángulo} < 200 \cdot \frac{u^2}{4}$$

es decir, tenemos

$$ 36 u^2 + 2\cdot \frac{u^2}{4}< \text{área del triángulo} < 50 u^2$$

¡Nos hemos aproximado más a nuestro objetivo!
\end{pba}

Con este procedimiento, podemos determinar el área del triángulo equilátero de lado 10 unidades y de cualquier polígono que se nos presente con cualquier longitud en cada lado: rectángulos, triángulos rectángulos, rombos, pentágonos regulares, pentágonos irregulares, etcétera. \\

De hecho, podemos determinar el área de cualquier figura con este procedimiento (¡aunque no sea un subconjunto del plano!).\\


%%%%%%%%%
%%%%%%%%% CAMBIAR TRIÁNGULO POR CUADRADO DE LADOS RACIONALES
%%%%%%%%%


Nosotros nos enfocaremos a tratar de determinar el área de algunos subconjuntos específicos del plano: los triángulos.

\begin{obs}
El área de cualquier objeto siempre es un número real no negativo.
\end{obs}
\begin{obs}\label{ob2}
Si $\square ABCD$ es un rectángulo, el $\A(\square ABCD)= |AB||BC|$.
\end{obs}

\begin{df}
Diremos que $\triangle ABC$ es un \textcolor{red}{\bf triángulo rectángulo}\index{triángulo ! rectángulo} si uno de sus ángulos internos es $\perp$.
\end{df}

Ya estamos muy cerca de nuestro objetivo, antes de ello probaremos la siguiente proposición:

\begin{prop}\label{prop1}
Sea $ \triangle ABC$ un triángulo rectángulo tal que $|\angle CBA|= \perp $ entonces el $\A(\triangle ABC)= \frac{|AB||BC|}{2}$.
\end{prop}

\begin{pba}
Sean $l$ la recta paralela a $\overline{AB}$ por $C$, $m$ la recta paralela a $\overline{BC}$ por $A$ y $l \cap m$=$\{D\}$. Consideremos al rectángulo $\square ABCD$ (¿realmente es un rectángulo?, ver sección de ejercicios: Ejercicio~\ref{rectángulo}) y una diagonal que en este caso, sin pérdida de generalidad, elegiremos a $AC$. Entonces $\triangle ABC \equiv \triangle CDA$ \textbf{cc(ALA)} ya que comparten el lado $AC$ y como $l$ y $\overline{AB}$ son paralelas y $\overline{AC}$ es una recta transversal, se cumple que $|\angle CAD|=|\angle ACB|$ y $|\angle DCA|=|\angle BAC|$.
Ahora, notemos que si dos triángulos son congruentes, entonces su área es la misma\footnote{Dados $\triangle ABC$ y $\triangle DEF$ tales que $\triangle ABC \equiv \triangle DEF$, por la Definición~\ref{dfárea}, tenemos que si $\A(\triangle ABC)=\alpha$, entonces caben $\alpha$ unidades cuadradas en $\triangle ABC$, ahora como $\triangle ABC \equiv \triangle DEF$ la magnitud de los lados del $\triangle DEF$ es la misma que la de los lados del $\triangle ABC$ así que tanbién caben $\alpha$ unidades cuadradas en $\triangle DEF$ y por tanto tienen la misma área.}.
%%%
%%% triángulos congruentes tienen la misma área 
%%% REALIZADO :)
%%% VERIFICAR UNICIDAD DE LA PARALELA  (GASDE)

Así, tenemos que:
\begin{eqnarray*}
\A (\square ABCD) &=& \A (\triangle ABC)+\A(\triangle CDA)\\
&=&\A (\triangle ABC)+\A(\triangle ABC)\\
&=&2\cdot \A(\triangle ABC)
\end{eqnarray*}

Y, por la Observación~\ref{ob2}, se sigue que: $|AB||BC|= 2 \cdot \A (\triangle ABC)$.

Por lo tanto,
$$\A(\triangle ABC)=\frac{|AB||BC|}{2}$$
\end{pba}

\begin{df}\label{ADUT}
Sean $\triangle ABC$ y $X \in \{A, B, C\}$.

$h_{X}$ es la \textcolor{red}{{\bf altura por el vértice X}}\index{altura de un triángulo} si y solamente si $h_{X}$ es la recta ortogonal por $X$ a la recta determinada por $\{A, B, C\} \setminus \{X\}$. A la intersección de $h_{X}$ con la recta determinada por $\{A, B, C\} \setminus \{X\}$ se le llama \textcolor{red}{\bf pie de la altura por X}\index{pie de altura}.
\end{df}

Es decir, consideremos $\triangle ABC$, entonces tenemos lo siguiente:
\begin{itemize}
\item $h_{A}$ es la recta ortogonal al lado $BC$ por el vértice $A$ y sea $h_{A}\cap BC=\{D\}$.
\item $h_{B}$ es la recta ortogonal al lado $CA$ por el vértice $B$ y sea $h_{B}\cap CA=\{E\}$.
\item $h_{C}$ es la recta ortogonal al lado $AB$ por el vértice $C$ y sea $h_{C}\cap AB=\{F\}$.
\end{itemize}
De esta manera, se tiene que $D$, $E$ y $F$ son los pies de altura por $A$, $B$ y $C$ respectivamente.
%%%%%
%%%%% Ejemplos y notación de los pies de las alturas
%%%%% REALIZADO :)

\begin{prop} \label{prop2}
Sea $\triangle ABC$ y $X \in \{A,B,C\}$. Si $D$ es el pie de la altura por $A$, $E$ es el pie de la altura por $B$ y $F$ es el pie de la altura por $C$ entonces
$$\A(\triangle ABC) = \frac{|BC||AD|}{2}= \frac{|CA||BE|}{2} = \frac{|AB||CF|}{2} $$
\end{prop}

\begin{pba}
Daremos una prueba de la primer igualdad que no dependerá de la elección de los vértices, por lo que la misma prueba servirá para probar las otras dos igualdades (ver sección de ejercicios, Ejercicio~\ref{area}).

Haremos la prueba para $X=A$. Sea $h_A \cap \overline{BC} =\{D\}$. Notemos que tenemos tres casos:
\begin{enumerate}
\item $\frac{BD}{DC} < 0$. 
\begin{enumerate}
\item Supongamos que $0 < \frac{DB}{BC}$.

Consideremos a los triángulos $\triangle ADC$ y $\triangle ADB$. Notemos que ambos triángulos son rectángulos (pues $\overline{AD}$ es ortogonal a $\overline{BC}$) por lo que, debido a la Proposición~\ref{prop1}, se tiene que
$$\A(\triangle ADC)= \frac{|DC||AD|}{2} \quad \text{y} \quad \A(\triangle ADB) = \frac{|DB||AD|}{2}$$

Puesto que $\triangle ABC$ está contenido en $\triangle ADC$ tenemos que
\begin{eqnarray*}
\A(\triangle ABC) &=& \A(\triangle ADC)-\A(\triangle ADB)\\
&=& \frac{|DC||AD|}{2} -  \frac{|DB||AD|}{2}\\
&=& \left(|DC|-|DB|\right)\frac{|AD|}{2} 
\end{eqnarray*}

Por hipótesis, $\frac{BD}{DC}<0$ y $0< \frac{DB}{BC}$, por lo que
\begin{itemize}
\item $0<BD \Rightarrow DC<0$ y $BC <0$. Así,
\begin{eqnarray*}
\A(\triangle ABC) &=& \left(|DC|-|DB|\right)\frac{|AD|}{2}\\
&=&   \left(CD-(-DB)\right)\frac{|AD|}{2}\\
&=&   \left(CD +DB\right)\frac{|AD|}{2} = \frac{(CB)|AD|}{2} \\
&=& \frac{|BC||AD|}{2}\\
\end{eqnarray*}


%%%
%%%
%%%

\item $BD<0 \Rightarrow 0<DC$ y $0<BC$. Así, 
\begin{eqnarray*}
\A(\triangle ABC) &=& \left(|DC|-|DB|\right)\frac{|AD|}{2}\\
&=&   \left(DC-DB\right)\frac{|AD|}{2}\\
&=&   \left(DC+(-DB)\right)\frac{|AD|}{2}\\
&=&   \left(DC+BD\right)\frac{|AD|}{2}\\
&=&   \left(BD+DC\right)\frac{|AD|}{2}\\
&=& \frac{(BC)|AD|}{2}=\frac{|BC||AD|}{2}\\
\end{eqnarray*}
\end{itemize}

%%%
%%%REALIZADO
%%% :)
		 

\item Si $0 <\frac{BC}{CD}$ la prueba es análoga (ver sección de ejercicios, Ejercicio~\ref{cortafuera}).		 

\end{enumerate}

\item $0= \frac{BD}{DC}$

En este caso, estamos diciendo que $B=D$, por lo que $\overline{AB}$ es ortogonal a $\overline{BC}$; es decir, $\triangle ABC$ es un triángulo rectángulo y podemos hacer uso de la Proposición~\ref{prop1}.

\item $0< \frac{BD}{DC}$.

%%%
%%%
%%%

Consideremos los triángulos $\triangle ABD$ y $\triangle ADC$, notemos que ambos triángulos son rectángulos ya que AD es ortogonal a BC, entonces por la Proposición \ref{prop1} el $\A(\triangle ABD)$= $\frac{|BD||AD|}{2}$ y el $\A(\triangle ADC)$= $\frac{|DC||AD|}{2}$ así que como	el $\A(\triangle ABC)$=$\A(\triangle ABD)$+$\A(\triangle ADC)$ se tiene que:
		 
$\A(\triangle ABC)$=$\frac{|BD||AD|}{2}$ +  $\frac{|DC||AD|}{2}$= $\frac{|AD|(|BD|+|DC|)}{2}$=$\frac{|AD||BC|}{2}.$
		
Veamos como en el primer caso que el resultado no depende de la orientación de los segmentos. 
Por hipótesis $0< \frac{BD}{DC}$, entonces:
\begin{enumerate}
\item Si $0<BD \Rightarrow 0<DC.$
\begin{itemize}
\item Si $0<AD$, entonces:
\begin{eqnarray*}
\A(\triangle ABC)
&=&  \frac{|BD||AD|}{2} + \frac{|DC||AD|}{2}\\
&=&  \frac{(BD)(AD)}{2} + \frac{(DC)(AD)}{2}\\
&=&  \frac{(AD)(BD+DC)}{2}=\frac{(AD)(BC)}{2}\\
&=&  \frac{|AD||BC|}{2}.\\ 
\end{eqnarray*}
\item Si $AD<0$, entonces:
\begin{eqnarray*}
\A(\triangle ABC)
&=&  \frac{|BD||AD|}{2} + \frac{|DC||AD|}{2}\\
&=&  \frac{(BD)(DA)}{2} + \frac{(DC)(DA)}{2}\\
&=&  \frac{(DA)(BD+DC)}{2}=\frac{(DA)(BC)}{2}\\
&=&  \frac{|AD||BC|}{2}.\\ 
\end{eqnarray*}		
\end{itemize}		 
\item Si $BD<0 \Rightarrow DC<0$
\begin{itemize}
\item Si $0<AD$, entonces:
\begin{eqnarray*}
\A(\triangle ABC)
&=&  \frac{|BD||AD|}{2} + \frac{|DC||AD|}{2}\\
&=&  \frac{(DB)(AD)}{2} + \frac{(CD)(AD)}{2}\\
&=&  \frac{(AD)(DB+CD)}{2}=\frac{(AD)(CD+DB)}{2}\\
&=&  \frac{(AD)(CB)}{2}=\frac{|AD||BC|}{2}.\\ 
\end{eqnarray*}
\item Si $AD<0$, entonces:
\begin{eqnarray*}
\A(\triangle ABC)
&=&  \frac{|BD||AD|}{2} + \frac{|DC||AD|}{2}\\
&=&  \frac{(DB)(DA)}{2} + \frac{(CD)(DA)}{2}\\
&=&  \frac{(DA)(DB+CD)}{2}=\frac{(DA)(CD+DB)}{2}\\
&=&  \frac{(DA)(CB)}{2}=\frac{|AD||BC|}{2}.\\ 
\end{eqnarray*}		
\end{itemize}
\end{enumerate}
\end{enumerate}
	
\end{pba}


%%%% Hacer los subcasos
%%% CORRECCIÓN REALIZADA :)
%%%%DUDA: CASO 2 (REVISAR) 


%%%%%% Cuidar la longitud de la altura :)
Así, la proposición Proposición~\ref{prop2} nos da la forma de calcular el área de cualquier triángulo dado. 
Enseguida, se mencionan un par de resultados que relacionan el área de dos triángulos. 
		 
\begin{prop}\label{prop3}
Sean $\triangle ABC$ y $\triangle DEF$ tales que $|AB|=|DE|$ y consideremos $h_{C}\cap AB=\{G\}$ y $h_{F}\cap DE=\{H\}$, entonces:
$$\frac{\A(\triangle ABC)}{\A(\triangle DEF)}= \frac{|CG|}{|FH|}.$$
		
\end{prop}
\begin{pba} Por la Proposición~\ref{prop2}, tenemos que:
$$\frac{\A(\triangle ABC)}{\A(\triangle DEF)}= \frac{\frac{|CG||AB|}{2}}{\frac{|FH||DE|}{2}}$$
Como $|AB|=|DE|$, concluimos que:
$$\frac{\A(\triangle ABC)}{\A(\triangle DEF)}= \frac{|CG|}{|FH|}$$
\end{pba}
		 
\begin{prop}\label{prop4}
Sean $\triangle ABC$ y $\triangle DEF$ y consideremos $h_{C}\cap AB=\{G\}$ y $h_{F}\cap DE=\{H\}$, tales que $|CG|=|FH|$, entonces: 
$$\frac{\A(\triangle ABC)}{\A(\triangle DEF)}= \frac{|AB|}{|DE|}.$$
\end{prop}
\begin{pba}Por la Proposición~\ref{prop2}, tenemos que:
$$\frac{\A(\triangle ABC)}{\A(\triangle DEF)}= \frac{\frac{|CG||AB|}{2}}{\frac{|FH||DE|}{2}}$$ 
Como $|CG|=|FH|$ se sigue que:
$$\frac{\A(\triangle ABC)}{\A(\triangle DEF)}= \frac{|AB|}{|DE|}$$
\end{pba}
 
%%%%%%%
%%%%%%%Notación en las pruebas anteriores
%%%%%%%CORRECCIÓN REALIZADA :) 


\subsection*{Ejercicios}

\begin{enumerate}

\item Demostrar que si  $\square ABCD$ es un paralelogramo tal que un ángulo interno es recto entonces todos sus ángulos internos son rectos.

\item  Demostrar que si $\square ABCD$ es un cuadrado tal que $|AB|=\alpha \in \N$ entonces $\A(\square ABCD)=\alpha^{2}$.

\item Demostrar que, en la Proposición~\ref{prop1} (página \pageref{prop1}), la construcción dada forma efectivamente un rectángulo.\label{rectángulo}

\item En la Proposición~\ref{prop2} (página \pageref{prop2}), probar el  caso en el que $0<\frac{BC}{CD}$.\label{cortafuera}

\item Verificar que la prueba de la Proposición~\ref{prop2} (página \pageref{prop2}) no depende de la elección de los vértices. (Sugerencia: Sigue la misma prueba solamente cambiando de vértices y del respectivo pie de altura).\label{area}
\end{enumerate}




\chapter{Introducción}


REALIZAR LISTA DE EJERCICIOS DE TODOS LOS TEMAS.


\section{Segmentos de recta dirigídos}

\section{Relaciones entre segmentos de recta dirigidos}

\section{Razón de partición de un segmento de recta}

\section{Ángulos dirigidos}

\section{Funciones biyectivas}

\section{Puntos al infinito}

\section{Hileras y haces}




\chapter{Congruencia de triángulos}

\section{Criterios de congruencia de triángulos}


\subsection*{Ejercicios}
\begin{enumerate}
\item Demostrar que si $\square ABCD$ es un paralelogramo (ver Definición~\ref{paralelogramo}), entonces
\begin{enumerate}
\item $\triangle ABD\equiv\triangle CDB$
\item $\triangle ACD\equiv\triangle CAB$. \label{EPLI}
\end{enumerate}
\item Demostrar que si $\square ABCD$ es un paralelogramo y $\overline{AC}\cap\overline{BD}=\{P\}$, entonces $|AP|=|PC|$ y $|BP|=|PD|$. \label{IDDP}
\end{enumerate}




\chapter{Semejanza de triángulos}

\section{Teorema de Thales}
A continuación, se enunciará uno de los teoremas fundamentales en el estudio de la semejanza de triángulos, atribuido al matemático griego Thales de Mileto. 
%%%%%%
%%%%%%
%%%%%%
\begin{teo}[Primer teorema de Thales o teorema fundamental de la proporcionalidad]\label{Thales1}
Sean $\triangle ABC$, $N\in\overline{AB}$ y $M\in\overline{AC}$, entonces $\overline{BC}$ es paralela a          
$\overline{NM}$ si y solamente si $$\frac{AB}{AN}=\frac{AC}{AM}.$$
\end{teo}
\begin{dem}
Sean $\triangle ABC$, $N\in\overline{AB}$  y $M\in\overline{AC}$.
\begin{enumerate}
\item[($\Rightarrow$)]
Supongamos que $\overline{BC}$ es paralela a $\overline{NM}$. Consideremos $\triangle ABM$ y $\triangle ANM$. Observemos que comparten la altura por M, entonces por la Proposición~\ref{prop4} se tiene que: $$\frac{\A(\triangle ABM)}{\A(\triangle ANM)}=\frac{AB}{AN}.$$ 
Ahora consideremos $\triangle ACN$ y $\triangle AMN$ que comparten la altura por N, entonces por la Proposición~\ref{prop4}, tenemos que: $$\frac{\A(\triangle ACN)}{\A(\triangle AMN)}=\frac{AC}{AM}.$$ 

Observación 1. $\A(\triangle AMN)=\A(\triangle ANM)$.

Observación 2. $\A(\triangle NMB)=\A(\triangle NMC)$.

Ahora notemos que:

$\A(\triangle ABM)=\A(\triangle ANM)+\A(\triangle NMB)$

$\A(\triangle ACN)=\A(\triangle AMN)+\A(\triangle NMC)$

Entonces:
\begin{eqnarray*}
\frac{\A(\triangle ABM)}{\A(\triangle ANM)}
&=& \frac{\A(\triangle ANM)+\A(\triangle NMB)}{\A(\triangle ANM)}\\
&=& 1+\frac{\A(\triangle NMB)}{\A(\triangle ANM)}\\
\end{eqnarray*}
\begin{eqnarray*}
\frac{\A(\triangle ACN)}{\A(\triangle AMN)}
&=& \frac{\A(\triangle AMN)+\A(\triangle NMC)}{\A(\triangle AMN)}\\
&=& 1+\frac{\A(\triangle NMC)}{\A(\triangle AMN)}\\
\end{eqnarray*}

Por las observaciones tenemos que $\frac{\A(\triangle NMB)}{\A(\triangle ANM)}=\frac{\A(\triangle NMC)}{\A(\triangle AMN)}$.

Así, $$\frac{\A(\triangle ABM)}{\A(\triangle ANM)}=\frac{\A(\triangle ACN)}{\A(\triangle AMN)}
$$
Y por lo tanto,  $$\frac{AB}{AN}=\frac{AC}{AM}.$$
\item[($\Leftarrow$)] 
Sabemos que, $$\frac{AB}{AN}=\frac{AC}{AM}.$$ Supongamos que $\overline{BC}$ no es paralela a          
$\overline{NM}$. Entonces consideremos $C'\neq C$ tal que $C'\in\overline{AC}$ y  $\overline{NM}$ es paralela a          
$\overline{BC'}$.
Aplicando el resultado anterior a $\triangle ABC'$ se tiene que:
\begin{equation}
\frac{AB}{AN}=\frac{AC'}{AM}
\end{equation}	 
Pero por hipótesis, 
\begin{equation}
\frac{AB}{AN}=\frac{AC}{AM}
\end{equation}
De esto tenemos que: 
$$\frac{AC}{AM}=\frac{AC'}{AM}$$ entonces $AC=AC'$ y así $C=C'$ lo que es una contradicción. 
\end{enumerate}
\end{dem}
	
	
\begin{obs}
En el teorema anterior se obtuvo la siguiente relación, $$\frac{AB}{AN}=\frac{AC}{AM}$$ que nos dice más de lo parece.\\
Notemos que: \\
\begin{center}
$\frac{AB}{AN}=\frac{AC}{AM}$ $\Leftrightarrow$ $\frac{AN+NB}{AN}=\frac{AM+MC}{AM}$ $\Leftrightarrow$ $1+\frac{NB}{AN}= 1+\frac{MC}{AM}$ $\Leftrightarrow$  $\frac{NB}{AN}=\frac{MC}{AM}$.
\end{center}
Esto significa que dadas dos rectas paralelas, la proporción que existe entre las rectas transversales que las cortan siempre es la misma aunque dichas rectas transversales cambien. 
\end{obs}
	
	
Ahora como una consecuencia inmediata del teorema anterior enunciaremos el siguiente.
	
\begin{teo}[Segundo teorema de Thales]\label{Thales2} Sean $l, m, n$ tres rectas dadas (distintas) y $t_{1}, t_{2}$ dos rectas (distintas) tales que:
\begin{center}
\begin{tabular}{r r}

$l \cap t_{1}$=$\{A\}$ & $ l \cap t_{2}$=$\{D\}$\\
$m \cap t_{1}$=$\{B\}$ & $ m \cap t_{2}$=$\{E\}$\\
$n \cap t_{1}$=$\{C\}$ & $ n \cap t_{2}$=$\{F\}$\\
	
\end{tabular}
\end{center}
Tenemos que si $l$, $m$ y $n$ son paralelas entonces $$\frac{AB}{BC}=\frac{DE}{EF}.$$
	
\end{teo}
	
\begin{dem}
Supongamos que $l$, $m$ y $n$ son paralelas.
			
Sean $\overline{AF}$=$p$ y $m\cap p=\{G\}$ como $n$ es paralela a $m$ y tenemos que el $\triangle ACF$ es tal que $B\in \overline{AC}$ y $G\in \overline{AF}$ entonces, del Teorema~\ref{Thales1} se sigue que:  $$\frac{AB}{BC}=\frac{AG}{GF}.$$
Ahora, consideremos el $\triangle AFD$ y que $l$ es paralela a $m$ como $G\in \overline{AF}$ y $E\in \overline{DF}$ entonces, por el Teorema~\ref{Thales1} se tiene que: $$\frac{AG}{GF}=\frac{DE}{EF}.$$
Y por tanto, de las igualdades obtenidas podemos concluir que: $$\frac{AB}{BC}=\frac{DE}{EF}$$\\
\end{dem}
			
Además, si $$\frac{AB}{BC}=\frac{DE}{EF}$$ y dos de las rectas $l$, $m$ o $n$ son paralelas entonces las tres rectas son paralelas. 
			
\begin{dem}		
Supongamos que $$\frac{AB}{BC}=\frac{DE}{EF}$$ y sin perder generalidad que $l$ es paralela a $m$.

Ahora, para generar una contradicción, supongamos que $m$ no es paralela a $n$  y sean $m'$ paralela a $n$ por $B$  y $m'\cap t_{2}= \{E'\}$ con $E'\neq E$. Entonces, por el Teorema~\ref{Thales1} como $\overline{DE'}$ es paralela a $n= \overline{CF}$ se sigue que  $$\frac{AB}{BC}=\frac{DE'}{E'F}$$ pero por hipótesis sabemos que:  $$\frac{AB}{BC}=\frac{DE}{EF}$$ de este modo concluimos que  $$\frac{DE}{EF}=\frac{DE'}{E'F}$$ por lo que $DE'=DE$ y así $E'=E$ lo cual es una contradicción ya que $E'\neq E$. 
Por lo tanto $m$ es paralela a $n$. 
\end{dem}		
	
\section{Criterios de semejanza de triángulos}
\begin{df}\label{Semejanza df}
Sean $\triangle ABC$ y $\triangle DEF$ dos triángulos. Decimos que el $\triangle ABC$ es {\bf semejante} al $\triangle DEF$ si y solamente si:
	
$$|\angle ABC|=|\angle DEF|,  \;\;\;\;\;\;\;\;\;\;\;\;\; |\angle BCA|=|\angle EFD|,  \;\;\;\;\;\;\;\;\;\;\;\;\; |\angle CAB|=|\angle FDE|$$
	
Y existe $ k \in \mathbb{R} \backslash \{0\}$ tal que: 
	
$$|AB|=|k||DE|,  \;\;\;\;\;\;\;\;\;\;\;\;\; |BC|=|k||EF|,  \;\;\;\;\;\;\;\;\;\;\;\;\; |CA|=|k||FD|.$$
Y escribiremos $(\triangle ABC \cong \triangle DEF)$.
\end{df}
	
	
Así como en la congruencia de triángulos se tiene a un conjunto de condiciones tales que si se cumplen, entonces podemos asegurar que los triángulos son congruentes, en el caso de la semejanza también se tiene y son los llamados {\bf Criterios de semejanza de triángulos}, que son tres: ángulo-ángulo-ángulo, lado-ángulo-lado y lado-lado-lado que denotaremos por \textbf{cs(AAA)}, \textbf{cs(LAL)} y \textbf{cs(LLL)} respectivamente.

%%%%%%%
%%%%% REVISAR LO QUE ERNESTO DIJO SOBRE LA "K".
%%%%%%CORRECCIÓN REALIZADA :)	
\begin{teo}[Criterio se semejanza AAA]
Sean $\triangle ABC$ y $\triangle DEF$ tales que: 
$$|\angle ABC|=|\angle DEF|,  \;\;\;\;\;\;\;\;\;\;\;\;\; |\angle BCA|=|\angle EFD|,  \;\;\;\;\;\;\;\;\;\;\;\;\; |\angle CAB|=|\angle FDE|$$
entonces $\triangle ABC \cong \triangle DEF$
\end{teo}
	
\begin{dem}
Consideremos $0<AB$ y sea $N \in \overline{AB}$ tal que $0<AN$ y $|AN|=|DE|.$
Y también consideremos $0<AC$ y sea $M \in \overline{AC}$ tal que $0<AM$ y $|AM|=|DF|$.
	
Como $N \in \overline{AB}$  y $M \in \overline{AC}$, entonces  $|\angle NAM|=|\angle DEF|$ y además por hipótesis sabemos que $|\angle BCA|=|\angle EFD|$ por tanto, $|\angle NAM|=|\angle EFD|$, entonces $\triangle ANM \equiv \triangle DEF$ \textbf{cc(LAL)}. Como consecuencia, notemos que $|NM|=|EF|$,  $|\angle NMA|=|\angle EFD|$ y  $|\angle MNA|=|\angle FED|.$

Ahora, consideremos $0<NM$. Sea $L \in \overline{NM}$ tal que $0<NL$ y $\frac{NL}{LM}<0.$
Así, $|\angle LMC|=|\angle NMA|$  ya que son opuestos por el vértice y además por hipótesis, $|\angle BCA|=|\angle EFD|$ por ello, $|\angle BCA|=|\angle NMA|$ pues $|\angle NAM|=|\angle DEF|$, por tanto $\overline{NM}$ es paralela a $\overline{BC}$, entonces por el Teorema~\ref{Thales1}, concluimos que:
$$ \frac{|AB|}{|AN|}=\frac{|AC|}{|AM|}.$$ 
Y como $|AN|=|DE|$ y $|AM|=|DF|$, entonces:
$$ \frac{|AB|}{|DE|}=\frac{|AC|}{|DF|}.$$ 
De manera análoga, se tiene que $ \frac{|AC|}{|DF|}=\frac{|BC|}{|EF|}$ y así,
$$\frac{|AB|}{|DE|}=\frac{|AC|}{|DF|}=\frac{|BC|}{|EF|}=k \;\;\;\;\;\; (k\in\mathbb{R})$$
Por lo tanto, $\triangle ABC \cong \triangle DEF.$
\end{dem}	 
%%%%%%%
%%%%%
%%%%%%%

\begin{cor}
Como la suma de los ángulos internos de cualquier triángulo es $2\perp$, entonces el criterio de semejanza AAA (\textbf{cs(AAA)}) se reduce a AA (\textbf{cs(AA)}).
\end{cor}
	
Ahora, probaremos un lema que nos ayudará a demostrar el criterio de semejanza LAL.

%%%
%%% Pienso que puede ser ejercicio
%%%	
\begin{lema}\label{LemacsLAL}
Sean $\triangle ABC$, $\triangle DEF$ y $\triangle GHI$ tales que: $\triangle ABC \cong \triangle DEF$ y $\triangle DEF \equiv \triangle GHI.$ Entonces $\triangle ABC \cong \triangle GHI.$
\end{lema}
\begin{dem}
Como $\triangle ABC \cong \triangle DEF$, entonces: 
$$|\angle ABC|=|\angle DEF|,  \;\;\;\;\;\;\;\;\;\;\;\;\; |\angle BCA|=|\angle EFD|,  \;\;\;\;\;\;\;\;\;\;\;\;\; |\angle CAB|=|\angle FDE|$$
	
Y existe $ k \in \mathbb{R} \backslash \{0\}$ tal que:
	
$$|AB|=|k||DE|,  \;\;\;\;\;\;\;\;\;\;\;\;\; |BC|=|k||EF|,  \;\;\;\;\;\;\;\;\;\;\;\;\; |CA|=|k||FD|.$$
Además como $\triangle DEF \equiv \triangle GHI$, tenemos que:
$$|\angle DEF|=|\angle GHI|,  \;\;\;\;\;\;\;\;\;\;\;\;\; |\angle EFD|=|\angle HIG|,  \;\;\;\;\;\;\;\;\;\;\;\;\; |\angle FDE|=|\angle IGH|$$
	
y
	
$$|DE|=|GH|,  \;\;\;\;\;\;\;\;\;\;\;\;\; |EF|=|HI|,  \;\;\;\;\;\;\;\;\;\;\;\;\; |FD|=|IG|.$$
Por tanto,
$$|\angle ABC|=|\angle GHI|,  \;\;\;\;\;\;\;\;\;\;\;\;\; |\angle BCA|=|\angle HIG|,  \;\;\;\;\;\;\;\;\;\;\;\;\; |\angle CAB|=|\angle IGH|$$
	
Y existe $ k \in \mathbb{R} \backslash \{0\}$ tal que:
	
$$|AB|=|k||GH|,  \;\;\;\;\;\;\;\;\;\;\;\;\; |BC|=|k||HI|,  \;\;\;\;\;\;\;\;\;\;\;\;\; |CA|=|k||IG|.$$
Así, $\triangle ABC \cong \triangle GHI.$
\end{dem}
	
\begin{teo}[Criterio de semejanza LAL]
Sean $\triangle ABC$ y $\triangle DEF$ tales que: 
$$\frac{|AB|}{|DE|}=\frac{|CA|}{|FD|}$$ y $|\angle BAC|=|\angle EDF|$, entonces $\triangle ABC \cong \triangle DEF.$
\end{teo}
\begin{dem}
Consideremos:
	
$0<AB$ y sea $N \in \overline{AB}$ tal que $0<AN$ y $|AN|=|DE|$.
	
$0<AC$ y sea $M \in \overline{CA}$ tal que $0<AM$ y $|AM|=|DF|$.
Entonces,
$$\frac{|AB|}{|DE|}=\frac{|AB|}{|AN|}$$ y $$\frac{|CA|}{|FD|}=\frac{|AC|}{|AM|}.$$
Así, $$\frac{|AB|}{|AN|}=\frac{|AC|}{|AM|}.$$
Entonces, por el Teorema~\ref{Thales1}, se tiene que: $\overline{NM}=\overline{BC}$ y por ello $|\angle ABC|=|\angle ANM|$. Como $N \in \overline{AB}$ y $M \in \overline{AC}$, $|\angle BAC|=|\angle NAM|$, entonces $\triangle ABC \cong \triangle ANM$ \textbf{cs(AA)}. Y además, $\triangle ANM \equiv \triangle DEF$ \textbf{cc(LAL)} ya que $|AN|=|DE|$, $|AM|=|DF|$ y $|\angle BAC|=|\angle NAM|$, entonces aplicando el Lema~\ref{LemacsLAL}, podemos concluir que $\triangle ABC \cong \triangle DEF.$ 
\end{dem}
	
\begin{teo}[Criterio de semejanza LLL]
Sean $\triangle ABC$ y $\triangle DEF$ tales que: 
$$\frac{|AB|}{|DE|}=\frac{|BC|}{|EF|}=\frac{|CA|}{|FD|}.$$ 
Entonces $\triangle ABC \cong \triangle DEF.$
\end{teo}
\begin{dem}
Consideremos:
$0<AB$ y sea $N \in \overline{AB}$ tal que $0<AN$ y $|AN|=|DE|$.
	
$0<AC$ y sea $M \in \overline{CA}$ tal que $0<AM$ y $|AM|=|DF|$.
Entonces, 
$$\frac{|AB|}{|DE|}=\frac{|AB|}{|AN|}$$ y
$$\frac{-|CA|}{|FD|}=\frac{|CA|}{|MA|}=\frac{|AC|}{|AM|}.$$	
Ahora, como $N \in \overline{AB}$ y $M \in \overline{CA}$, entonces $|\angle BAC|=|\angle NAM|$ por tanto $\triangle ANM \cong \triangle ABC$ \textbf{cs(LAL)} lo que implica que:
$$\frac{|BC|}{|NM|}=\frac{|AB|}{|AN|}=\frac{|CA|}{|MA|}.$$
Notemos que $\frac{|BC|}{|NM|}=\frac{|CA|}{|MA|}\Leftrightarrow |BC|\left(\frac{|MA|}{|CA|}\right)=|NM|$ y además por hipótesis 
$\frac{|BC|}{|EF|}=\frac{|CA|}{|FD|}$, entonces $|BC|\left(\frac{|FD|}{|CA|}\right)=|EF|$ y como $|FD|=|MA|$, tenemos que: $|NM|=|EF|$ y así $\triangle ANM \equiv \triangle DEF$ \textbf{cc(LLL)}. 
Finalmente, aplicando el Lema~\ref{LemacsLAL} concluimos que $\triangle ABC \cong \triangle DEF.$
\end{dem}

%%%
%%% Pienso que sea ejercicio
%%%

\begin{prop}\label{c->s}
Sean $\triangle ABC$ y $\triangle DEF$ tales que $\triangle ABC \equiv \triangle DEF$, entonces $\triangle ABC \cong \triangle DEF.$
\end{prop}
\begin{pba}
Como $\triangle ABC \equiv \triangle DEF$, entonces tenemos que:
$|\angle CAB|=|\angle FDE|$, $|\angle ABC|=|\angle DEF|$ y $|\angle ACB|=|\angle DFE|$, así concluimos que $\triangle ABC \cong \triangle DEF$ \textbf{cs(AA)}.
\end{pba}	

\begin{obs} 
Notemos que la Proposición~\ref{c->s} nos dice que congruencia implica semejanza y de hecho la razón de semejanza $|k|=1$. Pero el regreso no sucede siempre, es decir si dos triángulos son semejantes no necesariamente son congruentes, esto sólo pasa cuando la razón de semejanza $|k|=1$
\end{obs}
	
\section{Teorema de Pitágoras y su recíproco}
Ahora vamos a probar un teorema que seguramente el lector conoce de tiempo atrás, el teorema de Pitágoras, para ello antes probaremos un lema que nos será útil para la prueba de dicho teorema.

%%%
%%% Tres puntos en el plano no colineales
%%% CORRECCIÓN REALIZADA :)

\begin{lema}\label{LemaPitágoras}
Sea $\triangle ABC$ (en donde $A, B, C$ son tres puntos en el plano no colineales) tal que $|\angle CBA|=\perp$, entonces la altura $h_{B} \cap CA =\{D\}$ cumple que
$0<\frac{AD}{DC}$.
\end{lema}
\begin{pba}
	
Supongamos que $\frac{AD}{DC}\leqslant 0$. 
Tenemos dos casos:
\begin{enumerate}
\item $\frac{AD}{DC}=0$.
Esto implica que $A=D$, lo que nos da como resultado un triángulo en el cual $C$ es un punto al infinito, lo cual es una contradicción ya que por hipotésis $C$ es un punto en el plano.
\item $\frac{AD}{DC}<0$. 
Supongamos que $0<\frac{CA}{AD}$.
Como el ángulo $|\angle CAB|$ es un ángulo externo del $\triangle ABD$ entonces $|\angle CAB|=|\angle ADB|+|\angle DBA|$ por el mismo argumento se tiene que $|\angle BAD|=|\angle BCA|+|\angle ABC|$, como por hipótesis tenemos que $|\angle CBA|=\perp$ entonces $|\angle BAD|=|\angle BCA|+ \perp $, además como $D$ es pie de altura $|\angle CAB|= \perp+|\angle DBA|$ . Notemos que $|\angle CAB|+|\angle BAD|= 2 \perp$ ya que $C, A, D$ son colineales. Entonces tenemos que, $|\angle BCA|+ \perp+ \perp+|\angle DBA|= 2 \perp $, así $|\angle BCA|+|\angle DBA|=0 $, entonces $|\angle BCA|=0=|\angle DBA|$. Por ello, $|\angle CAB|=|\angle ADB|= \perp$, por lo que podemos concluir que $A=D$. De esto se sigue que $\frac{AD}{DC}=0$ lo que es una contradicción pues supusimos que $\frac{AD}{DC}<0$, de manera análoga se llega a una contradicción si $0<\frac{DC}{CA}$. 
\end{enumerate}
Por lo tanto, $0<\frac{AD}{DC}$.
\end{pba}
	
\begin{teo} [De Pitágoras]\label{TeoPitágoras}
Sea $\triangle ABC$ tal que $|\angle CBA|= \perp$, entonces $AB^{2} + BC^{2} = CA^{2}$.
\end{teo}
\begin{dem} Sea $\triangle ABC$.
Supongamos que $|\angle CBA|= \perp$ . Consideremos $h_{B}$ la altura por el vértice B. Sea $h_{B} \cap \overline{CA} =\{D\}$. Consideremos $\triangle ABD$ y $\triangle BDC$. Notemos que: $\triangle ABC \cong \triangle ADB$ \textbf{cs(AA)}, pues $|\angle CBA|=|\angle BDA|=\perp$ y $|\angle BAC|=|\angle DAB|$ pues A, D y C son colineales, entonces $\frac{|AB|}{|AC|}=\frac{|AD|}{|AB|}$. De esto tenemos que $|AB||AB|=|AD||AC|$, entonces $|AB|^{2}=|AD||AC|.$

Además $\triangle ABC \cong \triangle BDC$ \textbf{cs(AA)}, pues $|\angle CBA|=|\angle CDB|=\perp$ y $|\angle BCA|=|\angle DCB|$ pues A, D y C son colineales, entonces $\frac{|AC|}{|BC|}=\frac{|BC|}{|DC|}$. De esto tenemos que $|BC||BC|=|AC||DC|$, entonces $|BC|^{2}=|AC||DC|$. 
	
De lo  anterior se tiene que: $|AB|^{2}+|BC|^{2}=|AD||AC|+|AC||DC|=|AC|(|AD|+|DC|).$

Ahora notemos que: Si $0<CA$, entonces $|AC|=CA$ y por el Lema~\ref{LemaPitágoras} $|DC|=CD$ y $|AD|=DA.$ Así, $|AC|(|AD|+|DC|)= CA(DA+CD)=CA(CA)=CA^{2}.$
En otro caso, si $CA<0$, entonces $|AC|=AC$ y por el Lema~\ref{LemaPitágoras} $|DC|=DC$ y $|AD|=AD.$ Así, $|AC|(|AD|+|DC|)= AC(AD+DC)=AC(AC)=AC^{2}=CA^{2}.$ Y como $|AB|^{2}= AB^{2}$ y $|BC^{2}= BC^{2}$. Concluimos que $AB^{2}+BC^{2}=CA^{2}$.
\end{dem}

\begin{teo}[Recíproco del teorema de Pitágoras]
Sea $\triangle ABC$ tal que $AB^{2} + BC^{2} = CA^{2}$, entonces $|\angle CBA|= \perp$.
\end{teo}
\begin{dem}
Sea $\triangle ABC$ y supongamos que $AB^{2} + BC^{2} = CA^{2}$. Consideremos $l_{B}$ la recta ortogonal a $\overline{AB}$ por $B$. Sea $D \in l_{B}$ tal que $|DB|=|BC|$, entonces $DB^{2}=BC^{2}$.
Además por hipótesis tenemos que $AB^{2} + BC^{2} = CA^{2}$, entonces $AB^{2} + DB^{2} = CA^{2}.$
Ahora, considerando $\triangle ABD$ por el Teorema~\ref{TeoPitágoras} concluimos que $AB^{2} + BD^{2} = DA^{2}$, entonces $CA^{2}=DA^{2}$, así $|CA|=|DA|.$ Por lo cual $\triangle ABC\equiv \triangle ABD$ \textbf{cc(LLL)}. Por tanto, $|\angle CBA|=|\angle DBA|=\perp.$ 
\end{dem}
     
%%%%%%%
%%%%% AGREGAR MÁS EJERCICIOS.
%%%%%%%     
\subsection*{Ejercicios}
\begin{enumerate}
\item Sean $\triangle ABC$, $\triangle DEF$ y $\triangle GHI$. Demostrar que:
\begin{itemize}
\item $\triangle ABC \cong \triangle ABC$ (es decir, la relación de semejanza es {\bf {reflexiva}}).
\item Si $\triangle ABC \cong \triangle DEF$, entonces $\triangle DEF \cong \triangle ABC$ (es decir, la relación de semejanza es {\bf {simétrica}}).
\item Si $\triangle ABC \cong \triangle DEF$ y $\triangle DEF \cong \triangle GHI$, entonces $\triangle ABC \cong \triangle GHI$ (es decir, la relación de semejanza es {\bf {transitiva}}).
\end{itemize}
Lo que demuestra que la relación de de semejanza es una {\bf{realción de equivalencia}}.
\item Sean $\triangle ABC$ y $\triangle DEF$ tales que $\frac{AB}{DE}=\frac{AC}{DF},$ $AC<AB,$ $DF<DE$ y $|\angle ACB|=|\angle DFE|,$ entonces $\triangle ABC\cong \triangle DEF.$
\item Dado un segmento de recta $AB$ y $k \in \mathbb{R}$, entontrar $C \in \overline{AB}$ tal que $\frac{AC}{CB}=k.$
\item Dividir un segmento de recta en $n$ segmentos de la misma longitud. Argumentar el resultado. 
\item Sea $\triangle ABC$ y $L$, $M$ y $N$ los puntos medios de $BC$, $AC$ y $AB$ respectivamente. Demostrar que $\triangle ABC \cong \triangle LMN.$
\item Sea $\triangle ABC$ y $L$, $M$ y $N$ los puntos medios de $BC$, $AC$ y $AB$ respectivamente. Demostrar que:
$$\triangle ABC \cong \triangle LMN \cong \triangle ANM \cong \triangle NBL \cong \triangle MLC.$$
\item Demostrar que si $\{A, B, C\}\subset l$, donde $l$ es una recta en el plano y $D$ cualquier otro punto en el plano, entonces: 
$$DA^{2}\cdot BC+ DB^{2}\cdot CA+ DC^{2}\cdot AB+ AB\cdot BC\cdot CA=0.$$
\item Demostrar que dado un paralelogramo $\square ABCD$ (donde los vértices están ordenados levógiramente o dextrógiramente) se tiene que $AC^{2}+BD^{2}=2(AB^{2}+BC^{2}).$
\end{enumerate}

%%%%%%%%%%%%%%%%%%%%
%***************************REVISAR QUÉ DEBES AGREGAR EN EL INDEX.
%%%%%%%%%%%%%%%%%%%%%
     
\section{Trigonometría}
$\bullet$ suma seno y coseno\\

Sea $\triangle ABC$, observemos que se tienen las siguientes razones entre sus lados:


$$\frac{|BC|}{|CA|}\;\;\;\;\;\;\;\;\frac{|AB|}{|CA|}\;\;\;\;\;\;\;\;\frac{|BC|}{|AB|}\;\;\;\;\;\;\;\;\frac{|AB|}{|BC|}\;\;\;\;\;\;\;\;\frac{|CA|}{|AB|}\;\;\;\;\;\;\;\;\frac{|CA|}{|BC|}$$

Veamos también que con las dos primeras mediante cálculos sencillos podemos obtener las demás, de manera que podemos comenzar a trabajar con esas dos y más adelante extender los resultados a las demás. 

\begin{df}\label{ratrig}
Sea el $\triangle ABC$ rectángulo de tal manera que $|\angle CBA|\,=\,\perp$.

Definimos para $|\angle BAC|$  las razones siguientes:
     

$$\sen (|\angle BAC|)= \frac{|BC|}{|CA|}\;\;\;\;\;\;\;\;\;\;\;\; \cos (|\angle BAC|)= \frac{|AB|}{|CA|}$$

Al segmento $CA$ le llamaremos \textcolor{red}{\bf hipotenusa}\index{hipotenusa} del $\triangle ABC$, el lado $BC$ cateto opuesto al ángulo $|\angle BAC|$ y $AB$ \textcolor{red}{\bf catetos adyacentes}\index{catetos adyacentes} al ángulo $|\angle BAC|$.
\end{df}

\begin{obs}\label{sencos}
Si en la definición~\ref{ratrig} cambiaramos al ángulo $|\angle ACB|$ tenemos que:
$$\sen (|\angle BAC|)= \cos(|\angle ACB|)\;\;\;\;\;\;\;\;\;\;\cos(|\angle BAC|)=\sen(|\angle ACB|)$$ 
\end{obs}


Ya definidas estas razones notemos que dado un ángulo $0\,<\,\alpha\,<\,\perp$  utilizando los lados de un triángulo rectángulo que tenga a $\alpha$ como uno de sus ángulos internos podemos calcular $\sen(\alpha)$ y $cos(\alpha)$, pero para poder asegurar que estas razones se comportan como funciones hay que probar que $\sen(\alpha)$ y $cos(\alpha)$ no toman distintos valores al cambiar el triángulo con el que se calculen.

A continuaión veremos que estas razones dependen del ángulo y no del triángulo que se utilice para calcularlas.

\begin{prop}\label{bndef}
Sea $0 < \alpha < \perp$ y $\{ A,B,C,D,E,F\}$ puntos en el plano tales que
\[ |\angle ABC| = \alpha = |\angle DEF| \quad \textit{(ángulos congruentes).}\]
entonces:
\[ \sen( |\angle ABC|) = \sen( |\angle DEF|)\] y
\[ \cos( |\angle ABC|) = \cos( |\angle DEF|)\]

\begin{pba}
Notemos que $A$, $B$ y $C$ no son colineales, esto pues $ 0<|\angle ABC|$, entonces en particular tenemos que $A\notin \overline{BC}$.

Así sean $\mathit{l}$ la recta ortogonal a $\overline{BC}$ por $A$ y $\mathit{l} \cap \overline{BC} = \{P\}$. Por construcción $\triangle ABP$ es rectángulo con $|\angle BPA| = \perp$ y $|\angle ABC| = |\angle ABP|$ pues $\{B,P,C\}$ son colineales, entonces por definición:

\begin{equation}\label{trig1}
\sen( |\angle ABC|)=  \sen(|\angle ABP|) = \frac{|PA|}{|AB|}
\end{equation}

\begin{equation}\label{trig2}
\cos( |\angle ABC|)=  \cos(|\angle ABP|) = \frac{|BP|}{|AB|}
\end{equation}

(En este caso podemos asegurar la igualdad del seno y coseno pues nos referimos exactamente al mismo ángulo)
 
De manera análoga tenemos podemos construir $\mathit{m}$ la recta ortogonal a $\overline{EF}$ y $\mathit{m} \cup \overline{EF} = \{Q\}$ y así el $\triangle DEQ$ es rectángulo con $|\angle EQD| = \perp$ y  $|\angle DEF| = |\angle DEQ|$ pues $\{E,F,Q\}$ son colineales, entonces por definición:

\begin{equation}\label{trig3}
\sen( |\angle DEF|)=  \sen(|\angle DEQ|) = \frac{|QD|}{|DE|}
\end{equation}

\begin{equation}\label{trig4}
       \cos( |\angle DEF|)=  \cos(|\angle DEQ|) = \frac{|EQ|}{|DE|}
\end{equation}

Notemos que $\triangle ABP \cong \triangle DEQ$ \textbf{cs(AA)} pues:
\[ |\angle ABP| = \alpha = |\angle DEQ| \quad\&\quad |\angle BPA| = \perp = |\angle EQD| \]

por lo que:
\[ \frac{|AB|}{|DE|} = \frac{|BP|}{|EQ|} = \frac{|PA|}{|QD|} \]

De donde obtenemos:

 \[ \frac{|AB|}{|DE|} = \frac{|PA|}{|QD|} \Rightarrow \frac{|QD|}{|DE|} = \frac{|PA|}{|AB|}\]

y de ~\ref{trig1} y ~\ref{trig3} tenemos que

 \[ \sen( |\angle ABC|) = \sen( |\angle DEF|)\]
 
De la misma manera 
 
 \[ \frac{|AB|}{|DE|} = = \frac{|BP|}{|EQ|} \Rightarrow \frac{|EQ|}{|DE|} = \frac{|BP|}{|AB|}\]
 
 y por ~\ref{trig2} y ~\ref{trig4} concluimos

\[ \cos( |\angle ABC|) = \cos( |\angle DEF|)\]
 
\end{pba}
\end{prop}
         
Hasta ahora sólo podemos definir las funciones trigonométricas en ángulos que permitan su cálculo mediante triángulos rectángulos, es decir, en ángulos $\alpha$ con $0 < \alpha < \perp$.

Nuestra intención será extender estas funciones de manera que podamos calcularlas en cualquier ángulo. 

Primero probaremos  una de las identidades más importantes de la trigonometría, la cual relaciona al seno y al coseno de un ángulo mediante el teorema de Pitágoras. 

\begin{prop} \label{idpit}
Sea $0\, < \alpha \, < \, \perp$, entonces:
         
\[ (\sen(\alpha))^2 + (\cos(\alpha))^2 = 1 \]
            
\begin{pba} 
Sea el $\triangle ABC$ rectángulo de tal manera que $|\angle ABC|\,=\,\perp$ y 
sin pérdida de generalidad sea $\alpha = |\angle BAC|$.\\
De la definición ~\ref{ratrig} y al aplicando el Teorema~\ref{TeoPitágoras} tenemos que:
\begin{align*}
(\sen(|\angle BAC|))^2 + (\cos(|\angle BAC|))^2 & = \left( \frac{|BC|}{|CA|} \right)^2 + \left( \frac{|AB|}{|CA|}\right)^2\\
&= \frac{BC^2}{CA^2} + \frac{BC^2}{CA^2}\\
&= \frac{BC^2 + CA^2}{CA^2}\\
&= \frac{CA^2}{CA^2}\\
&= 1
\end{align*}
\end{pba}          
\end{prop}    

\begin{prop}\label{sumsen}
Sean $0 < \alpha\, , \,\beta <\, \perp$ ángulos tales que $\alpha + \beta\, < \,\perp$, entonces:
     
     \[ \sen(\alpha + \beta) = \sen(\alpha)\cos(\beta) + \sen(\beta)\cos(\alpha) \]

     \[ \cos(\alpha + \beta) = \cos(\alpha)\cos(\beta) - \sen(\alpha)\sen(\beta) \]
\end{prop}     
\begin{pba}
Sean $l$ y $m$ rectas tales que $\angle (l\,\longrightarrow m)\,=\,\alpha$,
%Revisar si la construcción es un ejercicio de alguna sección anterior.
sobre $m$ contruimos una recta $n$ tal que $\angle ( m \longrightarrow n)\,=\,\beta$.  Así $\angle (l \longrightarrow n) \,=\, \alpha + \beta$.

Sean $l\,\cap\,m\,\cap\,n \,=\,\{ A \}$ ,  $B\in\,l\,\backslash\, \{A\}$,
$r$ la ortogonal a $l$ por $B$ y $r\,\cap\,n\,=\{C\}$.

Por construcción el  $\triangle ABC$ es rectángulo y $|\angle CBA |=\perp$, entonces:
\begin{equation}\label{sumsen1}
\sen(\alpha+\beta)\,=\,\sen(|\angle BAC)|)\,=\,\frac{|BC|}{|CA|}
\end{equation}
\begin{equation}\label{sumcos1}
\cos(\alpha + \beta) =\,\cos(|\angle BAC)|)\,=\, \frac{|AB|}{|CA|}
\end{equation}
     
Sean $s$ la recta ortogonal a $m$ por $C$ y $m\cap\,s\,=\{D\}$.

Luego construimos $t$ la paralela a $l$ por $D$ y $u$ la paralela a $r$ por $D$, y tomamos $r\,\cap\,t\,= \{ E\}$, $l\,\cap\,u\,= \{ F\}$ y $r\,\cap\,m\,= \{G\}$.

El $\triangle DAF$ es rectángulo por construcción pues $u\,\parallel\,r\,\perp\, l$ y $|\angle DFA|=\perp$, entonces:

\[\sen(\alpha)\,=\,\sen(|\angle FAD|)\,=\,\frac{|FD|}{|DA|}\]
\[\cos(\alpha)\,=\,\cos(|\angle FAD|)\,=\,\frac{|AF|}{|DA|}\]
Por lo tanto  \begin{equation}\label{sumsen2}
|DA|\sen(\alpha)\,=\,|FD|
\end{equation}
              
\begin{equation}\label{sumcos2}
|DA|\cos(\alpha)\,=\,|AF|
\end{equation}

Como $s \perp m$ por construcción, los  $\triangle CDG$  y $\triangle ADC$ son rectángulos con $|\angle CDG|=\perp$ y $|\angle CDA|\,=\,\perp$.
\\

Entonces para $\triangle ADC$:

\begin{equation}\label{sumsen3}
\sen(\beta)\,=\,\sen(|\angle DAC|)\,=\,\frac{|DC|}{|CA|}
\end{equation}

\begin{equation}\label{sumsen4}
\cos(\beta)\,=\,\cos(|\angle DAC|)\,=\,\frac{|AD|}{|CA|}
\end{equation}


Además $t \perp r$, por lo que t es la altura del $\triangle CDG$ por $D$, entonces:
\[ \triangle CDG\, \cong \, \triangle DEG\, \cong \, \triangle CED \quad \textit{(Ver Teorema~\ref{TeoPitágoras})} \]

También tenemos que $t$ es paralela a $l$ y $m$ una transversal implica que $\alpha\,=|\angle BAG|\,=\,|\angle GDE|$.

Por la semejanza $\triangle CDG\, \cong \, \triangle DEG$ podemos concluir entonces $\alpha\,=\,|\angle ECD|$ y al ser $\triangle CDG$ rectángulo por definición:
\[ \cos(\alpha)\,= \,\cos(|\angle ECD|)\,=\,\frac{|CE|}{|DC|}\]
\[ \sen(\alpha)\,= \,\sen(|\angle ECD|)\,=\,\frac{|ED|}{|DC|}\]
Por lo tanto:
\begin{equation}\label{sumsen5}
|DC|\cos(\alpha)\,=\,|CE|
\end{equation}
\begin{equation}\label{sumcos3}
|DC|\sen(\alpha)\,=\,|ED|
\end{equation}

Por construcción la recta $m$ está entre $l$ y $n$, por lo que $E$ divide internamente a $BC$, es decir, $|BC|=|BE|+|EC|$. Por último $\square EBFD$ es un paralelogramo lo que implica $|EB|=|FD|$ y $|BF|=|DE|$. por lo tanto $|BC|=|FD|+|CE|$.


Regresando a ~\ref{sumsen1} tenemos al aplicar ~\ref{sumsen2} y ~\ref{sumsen5}:
\begin{align*}
 \sen(\alpha+\beta)  
&=\,\frac{|FD|+|CE|}{|CA|}\\
&=\,\frac{|FD|}{|CA|} + \frac{|CE|}{|CA|}\\
&=\,\frac{|DA|\sen(\alpha)}{|CA|} + \frac{|DC|\cos(\alpha)}{|CA|}\\
&=\,\sen(\alpha)\left(\frac{|AD|}{|CA|}\right) + \left(\frac{|DC|}{|CA|}\right)\cos(\alpha)\\
\end{align*}
Así por ~\ref{sumsen3} y ~\ref{sumsen4}:

\[ \sen(\alpha + \beta) = \sen(\alpha)\cos(\beta) + \sen(\beta)\cos(\alpha) \]
  
Tenemos también que $F$ divide externamente a $AB$, por lo tanto $|AB|=|AF|-|BF|$ y como $|BF|=|DE|$, entonces $|AB|=|AF|-|ED|$. 
%%%%%%%%%%%%%%%%%%%%
%**************************Falta justificar que F divide externamente.
%%%%%%%%%%%%%%%%%%%%
Así para ~\ref{sumcos1}:
\begin{align*}
\cos(\alpha + \beta) 
&=  \frac{|AB|}{|CA|}\\
&= \frac{|AF|}{|CA|} - \frac{|ED|}{CA|}\\
\end{align*}  
  
Por ~\ref{sumcos2}, ~\ref{sumsen3}, ~\ref{sumsen4} y ~\ref{sumcos3} tenemos entonces:
\begin{align*}
\cos(\alpha + \beta) 
&= \frac{|DA|\cos(\alpha)}{|CA|} - \frac{|DC|\sen(\alpha)}{CA|}\\
&=\cos(\alpha)\left(\frac{|AD|}{|CA|}\right) - \sen(\alpha)\left(\frac{|DC|}{|CA|}\right)\\
&=\cos(\alpha)\cos(\beta) - \sen(\alpha)\sen(\beta)\\ 
\end{align*}  
\end{pba}

Antes de continuar daremos dos definiciones que nos ayudarán más adelante:
\begin{df}
Sean $\alpha$ y $\beta$ dos ángulos, decimos que $\alpha$ y $\beta$ son \textcolor{red}{\bf ángulos  complementarios}\index{ángulos ! complementarios} si $\alpha+\beta\,=\,\perp$, y son \textcolor{red}{\bf ángulos suplementarios}\index{ángulos !  suplementarios} si   $\alpha+\beta\,=\,2\perp$.

Note que en un triángulo rectángulo los ángulos distintos del ángulo recto son complementarios.
\end{df}

Para extender las funciones a cualquier valor de $\alpha$ haremos la siguiente construcción.
\begin{df}
Sean $l$ y $m$ dos rectas ortogonales y $l\cap m = \{A\}$, definimos:
\begin{itemize}
\item $l_{A}^{+} \,=\,\{X\in\, l \,\, | \,\, AX \geq 0\}$ el rayo positivo desde $A$ en $l$,
\item $l_{A}^{-} \,=\,\{X\in\, l \,\, | \,\, AX \leq 0\}$ el rayo negativo desde $A$ en $l$,
\item $m_{A}^{+} \,=\,\{X\in\, m \,\, | \,\, AX \geq 0\}$ el rayo positivo desde $A$ en $m$,
\item $m_{A}^{-} \,=\,\{X\in\, m \,\, | \,\, AX \leq 0\}$ el rayo negativo desde $A$ en $m$,
\end{itemize}
Observemos que por el cuarto postulado el ángulo formado por cualesquiera dos rayos de rectas distintas mide un recto, así partiendo de $l_{A}^{+}$ consideraremos la orientación positiva de los angulos de tal manera que \[|\angle( l_{A}^{+} \rightarrow m_{A}^{+})| = \angle (l_{A}^{+} \rightarrow m_{A}^{+})\]
\end{df}

Sean $\zeta(A,r)$ la circunferencia con centro en $A$ y radio $r>0$, $\{P\}=\zeta \cap l_{A}^{+}$ y $\{Q\}=\zeta \cap m_{A}^{+}$ . $\widehat{AB}$

Tomemos  $C \in\widehat{PQ}\backslash \{P,Q\} $, $m'$ la ortogonal a $l$ por $C$ y $\{B\} = m' \cap l_A^{+}$.

Por construcción $\triangle ABC$ es rectángulo y el $\angle BAC$ es menor a un recto pues está comprendido entre $\angle (l_{A}^{+} \rightarrow m_{A}^{+}) = \perp$.
%%%%%%%%%%%%%%%%
%***************************Revisar el uso de arcos. 
%%%%%%%%%%%%%%%%%
Calculamos el seno y coseno de $\alpha\,=\, \angle BAC$:

\[\sen(\alpha)\,=\,\frac{|BC|}{|CA|}\,=\frac{|BC|}{r}\]
\[\cos(\alpha)\,=\,\frac{|AB|}{|CA|}\,=\frac{|AB|}{r}\]

Por la proposición ~\ref{bndef} podemos considerar solamente el caso cuanro $r=1$ y tomando la orientación de las rectas (heredando en $m'$ la orientación de $m_{A}^{+}$ tenemos entonces:
             
\[\sen(\alpha)\,=\,BC\,\]
\[\cos(\alpha)\,=\,AB\,\]

Que son los catetos del $\triangle ABC$. 

Con esta circunferencia podremos extender las funciones trigonométricas a cualquier valor de $\alpha$.

Sean $\{P'\}=\zeta \cap l_{A}^{-}$ y $\{Q'\}=\zeta \cap m_{A}^{-}$ así dado $C \in \zeta$, $m'$ la ortogonal a $l$ por $C$ y $\{B\} = m' \cap l$.
\begin{df}Definimos el seno y el coseno de $\alpha=\angle BAC$ de la siguiente manera:
               
\[\sen(\alpha)=BC \]
\[\cos(\alpha)=AB \]
                      
Note que  estamos considerando los segmentos y el ángulo dirigidos.                      
\end{df}

\begin{obs}\label{ext} Veamos que ocurre en los posibles valores de $\alpha$:
\begin{itemize}
\item Cuando $\alpha\,=0$ tenemos que, $C=B=P$ con $B \in l_A^{+}$ y por lo tanto:
\[\sen(0)\,=\,CC\,=\,0\]
\[\cos(0)\,=\,AP\,=\,1\]
   
\item Cuando $0 < \alpha < \perp$ la definición coincide con la que dimos al inicio del capítulo (por construcción).
                   
\item si $\alpha\,=\perp$ tenemos que $C=Q$, $B=A$  y por lo tanto:
           
\[\sen(\perp)\,=\,AQ\,=\,1\]
\[\cos(\perp)\,=\,AA\,=\,0\]
  
\item si $\perp < \alpha < 2\perp$, $B \in l_A^{-}$, el $\triangle ABC$ tiene a $\alpha$ como ángulo externo y en este caso el ángulo interno $\angle CAB = 2\perp - \alpha$ y 
             
\[\sen(\alpha)\,=\sen(2\perp-\alpha)\]
\[\cos(\alpha)\,=-\cos(2\perp-\alpha)\]
              
\item si $\alpha\,=2\perp$ tenemos que $C=P'$, $C=P'=B$ con $B \in l_A^{-}$ y por lo tanto:
           
\[\sen(2\perp)\,=\,P'P'\,=\,0\]
\[\cos(2\perp)\,=\,AP'\,=\,-1\] 
              
\item si $2\perp < \alpha < 3\perp$, $B \in l_A^{-}$, el $\triangle ABC$ tiene como ángulo interno $\angle BAC = \alpha - 2\perp$ y 
             
\[\sen(\alpha)\,=-\sen(\alpha - 2\perp)\]
\[\cos(\alpha)\,=-\cos(\alpha - 2\perp)\]
                         
\item si $\alpha\,=3\perp$ tenemos que $C=Q'$, $A=B$ y por lo tanto:
           
\[\sen(3\perp)\,=\,AQ'\,=\,-1\]
\[\cos(3\perp)\,=\,AA\,=\,0\]
              
\item si $3\perp < \alpha < 4\perp$, $B \in l_A^{+}$ , el $\triangle ABC$ tiene como ángulo interno $\angle CAB = 4\perp - \alpha$ y
  
\[\sen(\alpha)\,=-\sen(4\perp-\alpha)\]
\[\cos(\alpha)\,=\cos(4\perp-\alpha)\]
  
\item por último $\alpha= 4\perp$, de manera análoga a $0$ tenemos:
  
\[\sen(4\perp)\,=\,0\]
\[\cos(4\perp)\,=\,1\]
  
\end{itemize}        
Todo esto se puede probar en la construcción construyendo un $\triangle AB'C'$ congruente al $\triangle ABC$ tal que el seno y coseno del ángulo en $A$ coincidan con el ángulo correspondiente a cada caso y se recomienda como ejercicio. 
\end{obs}

Ya que estamos considerando ángulos dirigidos, necesitamos definir las funciones seno y coseno para ángulos negativos, recordemos entonces que para los ángulos $-\alpha = 4\perp - \alpha$ por lo tanto:
\begin{df} Sea $\alpha \in [0,4\perp]$ definimos:

\[sen(-\alpha)=\sen(4\perp -\alpha)\]
\[cos(-\alpha)=\cos(4\perp -\alpha)\]
\end{df}

\begin{prop}Sea $\alpha \in [0,4\perp]$ entonces:
\[sen(-\alpha)=-\sen(\alpha)\]
\[cos(-\alpha)=\cos(\alpha)\]
\end{prop}
\begin{pba}
Utilizando la observación ~\ref{ext}:
\begin{itemize}
\item si $\alpha \in [0,\perp]$ entonces $4\perp -\alpha \in [3\perp,4\perp]$ y por lo tanto:
 
\[sen(-\alpha)=\sen(4\perp -\alpha)=-\sen(4\perp-(4\perp - \alpha))=-\sen(\alpha)\]
\[cos(-\alpha)=\cos(4\perp -\alpha)= \cos(4\perp-(4\perp - \alpha))= \cos(\alpha)\]
     
\item si $\alpha \in [\perp,2\perp]$ entonces $4\perp -\alpha \in [2\perp,3\perp]$ y por lo tanto:
 
\[sen(-\alpha)=\sen(4\perp -\alpha)=-\sen((4\perp-\alpha) - 2\perp)=-\sen(2\perp - \alpha)=-\sen(\alpha)\]
\[cos(-\alpha)=\cos(4\perp -\alpha)= \cos((4\perp-\alpha) - 2\perp)= \cos(2\perp - \alpha)=\cos(\alpha)\]
     
\item si $\alpha \in [2\perp,3\perp]$ entonces $4\perp -\alpha \in [\perp,2\perp]$ y por lo tanto:
 
\[sen(-\alpha)=\sen(4\perp -\alpha)= \sen(2\perp -(4\perp - \alpha))= \sen(\alpha - 2\perp)=-\sen(\alpha)\]
\[cos(-\alpha)=\cos(4\perp -\alpha)= \cos(2\perp -(4\perp - \alpha))= \cos(\alpha - 2\perp)=\cos(\alpha)\] 
     
\item si $\alpha \in [3\perp,4\perp]$ entonces $4\perp -\alpha \in [0,\perp]$ y por lo tanto:
 
\[sen(-\alpha)=\sen(4\perp -\alpha)= -\sen(\alpha)\]
\[cos(-\alpha)=\cos(4\perp -\alpha)= \cos(\alpha)\]
\end{itemize}     
\end{pba}

Veamos que estas nuevas definiciones son consistentes con lo que ya desarrollamos. 

\begin{prop}\label{sumsencoscom}Las proposiciones ~\ref{idpit} y ~\ref{sumsen} son ciertas para cualquier ángulo $\alpha, \beta \in [0,4\perp]$, es decir:
\end{prop}
\begin{enumerate}
\item[a)] \[ (\sen(\alpha))^2 + (\cos(\alpha))^2 = 1 \]
\item[b)] \[ \sen(\alpha + \beta) = \sen(\alpha)\cos(\beta) + \sen(\beta)\cos(\alpha) \]
          \[ \cos(\alpha + \beta) = \cos(\alpha)\cos(\beta) - \sen(\alpha)\sen(\beta) \]
\end{enumerate}
\begin{pba}Sean $\alpha, \beta \in [0,4\perp]$ en la construcción anterior. 

\begin{enumerate}
\item[a)]Por definición si $C \in \{P,Q,P',Q'\}$ el seno y el coseno representan los catetos de un trángulo rectángulo así por el Teorema~\ref{TeoPitágoras}
\begin{align*}
(\sen(\alpha))^2+(\cos(\alpha))^2 &= (BC)^2 + (AB)^2 \\
&= (CA)^2 =1
\end{align*}
Para los casos especiales $(\sen(\alpha))^2+(\cos(\alpha))^2 = 1 + 0 = 1$
\item[b)] Evaluaremos por casos:
\begin{enumerate}
\item[Caso 1] para $0$ y $\alpha\,\in\,[0,4\perp]$
\begin{align*}
\sen(0+\alpha)&=\sen(0)\cos(\alpha) + \cos(0)\sen(\alpha)= 0\cdot\cos(\alpha) + 1\cdot\sen(\alpha)= \sen(\alpha)\\
\cos(0+\alpha)&=\cos(0)\cos(\alpha) - \sen(0)\sen(\alpha)= 1\cdot\cos(\alpha) - 0\cdot\sen(\alpha)= \cos(\alpha)
\end{align*}
\item[Caso 2] Si $\alpha, \beta \in [0,\perp]$ son complementarios.
\begin{align*}
\sen(\alpha + \beta) 
&= \sen(\alpha)\cos(\beta) + \sen(\beta)\cos(\alpha)\\
&= \cos(\beta)\cos(\beta) + \sen(\beta)\sen(\beta)\\
&= (\cos(\beta))^2 + (\sen(\beta))^2\\
&= 1\\
\textit{Y}  \quad
\sen(\perp) &= 1\\                   
\cos(\alpha + \beta) 
&= \cos(\alpha)\cos(\beta) - \sen(\alpha)\sen(\beta) \\
&= \sen(\beta)\cos(\beta) - \cos(\beta)\sen(\beta)\\
&= 0\\
\textit{Y}   \quad        
cos(\perp)&= 0\\
\end{align*}

\item[Caso 3] para $\perp$ y $\alpha\,\in\,[0,\perp)$.

Por un lado, 
\[\sen(\perp)\cos(\alpha) + \cos(\perp)\sen(\alpha)= 1\cdot\cos(\alpha) + 0\cdot\sen(\alpha)= \cos(\alpha)\]
\[\cos(\perp)\cos(\alpha) - \sen(\perp)\sen(\alpha)= 0\cdot\cos(\alpha) - 1\cdot\sen(\alpha)= -\sen(\alpha)\]

Y como $\alpha+\perp \in [\perp,2\perp$ por la observación ~\ref{ext} (y al ser $\alpha$ y $\perp - \alpha$ complementarios). 

\[\sen(\perp + \alpha)=\sen(2\perp-(\perp + \alpha))=\sen(\perp - \alpha)=\cos(\alpha)\]
\[\cos(\perp + \alpha)=-\cos(2\perp-(\perp + \alpha))=-\cos(\perp - \alpha)=-\sen(\alpha)\]


\item[Caso 4] Si $\alpha + \beta \in (\perp,2\perp]$ entonces al menos un ángulo es menor a un recto, s.p.g sea $\alpha \in [0,\perp]$, sabemos que existe $\gamma \in [0,\perp]$ tal que:
\[ \alpha + \beta = \perp + \gamma \]
Así 
\[\sen(\alpha + \beta) = \sen(\perp + \gamma)=\cos(\gamma)\]
\[\cos(\alpha + \beta) = \cos(\perp + \gamma)=-\sen(\gamma)\]

Por otro lado $\beta = (\perp-\alpha) + \gamma$ y $\perp - \alpha \in [0,\perp]$, por lo que:
\begin{align*}
\sen(\beta)
&=\cos(\perp-\alpha)\sen(\gamma)+\sen(\perp -\alpha)\cos(\gamma)\\
&=\sen(\alpha)\sen(\gamma)+ \cos(\alpha)\cos(\gamma)\\
\cos(\beta)
&=\cos(\perp-\alpha)\cos(\gamma)-\sen(\perp -\alpha)\sen(\gamma)\\
&=\sen(\alpha)\cos(\gamma)-\cos(\alpha)\sen(\gamma)\\           
\end{align*}
Entonces: 
\begin{align*}
\sen(\alpha)\cos(\beta) + \sen(\beta)\cos(\alpha)&= (\sen(\alpha))^2\cos(\gamma)-\sen(\alpha)\cos(\alpha)\sen(\gamma)\\
&\quad + \sen(\alpha)\cos(\alpha)\sen(\gamma)+ (\cos(\alpha))^2\cos(\gamma)\\
&=\cos(\gamma)((\sen(\alpha))^2+(\cos(\alpha))^2)\\
&=\cos(\gamma)\\
\& &\\   
\cos(\alpha)\cos(\beta) - \sen(\beta)\sen(\alpha)&= \sen(\alpha))\cos(\alpha)\cos(\gamma)-(\cos(\alpha))^2\sen(\gamma)\\
&\quad - (\sen(\alpha))^2\sen(\gamma)- \sen(\alpha)\cos(\alpha)\cos(\gamma)\\
&=-\sen(\gamma)((\sen(\alpha))^2+(\cos(\alpha))^2)\\
&=-\sen(\gamma)\\            
\end{align*}
\item[Caso 5]  Si $\alpha + \beta \in (2\perp,4\perp]$ entonces al menos un ángulo es menor a dos rectos, s.p.g sea $\alpha \in [0,2\perp]$, sabemos que existe $\gamma \in [0,2\perp]$ tal que:
\[ \alpha + \beta = 2\perp + \gamma \]
El resto de la demostración es análoga al caso 4.
\end{enumerate}  
\end{enumerate}
\end{pba}                   

Ahora podemos saber que ocurre con la resta de ángulos:
\begin{cor}Sean $\alpha,\beta \in [0,4\perp]$ entonces:
\[ \sen(\alpha - \beta) = \sen(\alpha)\cos(\beta) - \sen(\beta)\cos(\alpha) \]
\[ \cos(\alpha - \beta) = \cos(\alpha)\cos(\beta) + \sen(\alpha)\sen(\beta) \]
\end{cor}
\begin{pba}
\begin{align*}
\sen(\alpha - \beta) 
&= \sen(\alpha + (4\perp - \beta))\\
&= \sen(\alpha)\cos(4\perp - \beta) - \sen(4\perp - \beta)\cos(\alpha)\\
&= \sen(\alpha)\cos(-\beta) + \sen( - \beta)\cos(\alpha)\\
&= \sen(\alpha)\cos(\beta) +(- \sen( \beta))\cos(\alpha)\\
&= \sen(\alpha)\cos(\beta) - \sen( \beta)\cos(\alpha)\\
\cos(\alpha - \beta) 
&= \cos(\alpha + (4\perp - \beta))\\
&= \cos(\alpha)\cos(4\perp - \beta) - \sen(4\perp - \beta)\sen(\alpha)\\
&= \cos(\alpha)\cos(-\beta) - \sen( - \beta)\sen(\alpha)\\
&= \cos(\alpha)\cos(\beta) -(- \sen( \beta))\sen(\alpha)\\
&= \cos(\alpha)\cos(\beta) + \sen( \beta)\sen(\alpha)\\                    
\end{align*} 
\end{pba}

\begin{teo}[Ley de cosenos]\label{LDC}
Sea $\triangle ABC$, entonces
$$CA^{2}=AB^{2}+BC^{2}-2|AB||BC|Cos(\angle CBA).$$
$$AB^{2}=BC^{2}+CA^{2}-2|BC||CA|Cos(\angle ACB).$$
$$BC^{2}=CA^{2}+AB^{2}-2|CA||AB|Cos(\angle BAC).$$
\end{teo}
\begin{dem}
Sea $l$ la recta ortogonal a $BC$ por $A$ y $BC\cap l=\{D\}.$ Entonces por el Teorema~\ref{TeoPitágoras} tenemos los siguiente: 
\begin{eqnarray*}
CA^{2}&=&AD^{2}+DC^{2}\\
&=& AD^{2}+(DB+BC)^{2}\\
&=& AD^{2}+DB^{2}+2|DB|\cdot |BC|+BC^{2}\\
&=& AD^{2}+(AB^{2}-DA^{2})+2|DB|\cdot |BC|+BC^{2}\;\;\text{(Por el Teorema~\ref{TeoPitágoras})}\\ 
&=& AB^{2}+BC^{2}+2|DB|\cdot |BC| 
 \end{eqnarray*}
Ahora, como $\cos \angle  CAB=\frac{BD}{AB}$, entonces $AB\cos \angle CAB=BD$ así $-AB\cos \angle CAB=DB$.

Por lo tanto,
$$ CA^{2}=AB^{2}+BC^{2}-2|AB||BC|\cos\angle CBA.$$

Análogamente, se puede demostrar que:
$$AB^{2}=BC^{2}+CA^{2}-2|BC||CA|Cos(\angle ACB).$$
$$BC^{2}=CA^{2}+AB^{2}-2|CA||AB|Cos(\angle BAC).$$

(ver sección de ejercicios, Ejercicio~\ref{LDCE}). 
\end{dem}

\begin{teo}[Ley de senos]\label{LDS}
Sea $\triangle ABC$, entonces
$$\frac{|AC|}{\sen(\angle ABC)}=\frac{|BA|}{\sen(\angle BCA)}=\frac{|CB|}{\sen(\angle CAB)}.$$
\end{teo} 
\begin{dem}
Sea $l$ la recta ortogonal a $\overline{AB}$ por $C$, tal que $l\cap\overline{AB}= \{T\}.$ Entonces,
$$\sen(\angle CAB)=\frac{|CT|}{|AC|}\;\;\;\;\;\;\;\;\;\;\;\;\;\;\;\;\sen(\angle ABC)=\frac{|CT|}{|CB|}$$
Así que:
$$|AC|\sen(\angle CAB)=|CT|\;\;\;\;\;\;\;\;\;\;\;\;\;\;\;\;|CB|\sen(\angle ABC)=|CT|$$
Por tanto, $|AC|\sen(\angle CAB)=|CB|\sen(\angle ABC)$. Entonces, 
$$\frac{|AC|}{\sen(\angle ABC)}=\frac{|CB|}{\sen(\angle CAB)}$$
Análogamente, se obtiene que $\frac{|AC|}{\sen(\angle ABC)}=\frac{|BA|}{\sen(\angle BCA)}$ y $\frac{|CB|}{\sen(\angle CAB)}=\frac{|BA|}{\sen(\angle BCA)}$ (ver sección de ejercicios, Ejercicio~\ref{LDSE}). 

Entonces, por transitividad concluimos que
$$\frac{|AC|}{\sen(\angle ABC)}=\frac{|BA|}{\sen(\angle BCA)}=\frac{|CB|}{\sen(\angle CAB)}.$$
\end{dem}

\subsection*{Ejercicios}
\begin{enumerate}
\item Demostrar que si $0<\alpha+\beta<\perp$, entonces:
\begin{enumerate}
\item $\sen(\alpha+\beta)=\sen(\alpha)\cos(\beta)+\sen(\beta)\cos(\alpha).$
\item $\cos(\alpha+\beta)=\cos(\alpha)\cos(\beta)-\sen(\alpha)\sen(\beta).$
\end{enumerate}
\item Sean $0<\alpha<\perp$ y $0<\beta<\perp$. Demostrar que si $\alpha+\beta=\perp$, entonces $\sen(\alpha)=\cos(\beta).$
\item En la Teorema~\ref{LDC} (página \pageref{LDC}), probar que:
$$AB^{2}=BC^{2}+CA^{2}-2|BC||CA|Cos(\angle ACB).$$
$$BC^{2}=CA^{2}+AB^{2}-2|CA||AB|Cos(\angle BAC).$$ .\label{LDCE}
\item En la Teorema~\ref{LDS} (página \pageref{LDS}), probar que:

$$\frac{|AC|}{\sen(\angle ABC)}=\frac{|BA|}{\sen(\angle BCA)}\;\; y \;\;\frac{|CB|}{\sen(\angle CAB)}=\frac{|BA|}{\sen(\angle BCA)}$$.\label{LDSE} 
\end{enumerate}


\section{Teorema generalizado de la bisectríz}
\begin{teo}[Generalizado de la bisectriz]\label{TGB}\index{Teorema ! generalizado de la bisectriz}
Sea $\triangle ABC$ y $L\in\overline{BC}\backslash \{B,C\}$, entonces
$$\frac{BL}{LC}=\frac{AB\;\sen(\angle BAL)}{CA\;\sen(\angle LAC)}$$
\end{teo}
\begin{dem} Sea $b$ la recta ortogonal a $\overline{AL}$ por $B$ y $c$ la recta ortogonal a $\overline{AL}$ por $C$. Entonces $\overline{AL}\cap b=\{P\}$ y $\overline{AL}\cap c=\{Q\}.$ De esta forma, por construcción tenemos que $\triangle BAP$ y  $\triangle CAQ$ son triángulos rectángulos, en donde  $\angle APB=\perp$ y $\angle CQA=\perp.$
Notemos que $\angle BAL=\angle BAP$ ya que $L$ y $P$ son colineales, así que $\sen(\angle BAL)=\sen(\angle BAP)=\frac{|BP|}{|AB|}=\frac{BP}{AB}.$
Por tanto, $AB\;\sen(\angle BAL)=BP.$ Análogamente, como $L$ y $Q$ son colineales $\angle LAC=\angle QAC=\frac{|QC|}{|AC|}=\frac{QC}{CA}.$ Entonces, $CA\;\sen(\angle LAC)=QC.$

En consecuencia, tenemos que:$$\frac{BP}{QC}=\frac{AB\;\sen(\angle BAL)}{CA\;\sen(\angle LAC)}.$$

Finalmente, probemos que $\frac{BP}{QC}=\frac{BL}{LC}.$ 
Primero notemos que $\triangle BPL\cong\triangle CQL$ \textbf{cs(AA)} ya que $|\angle BPL|=|\angle CQL|=\perp$ y $|\angle PLB|=|\angle QLC|$ pues $P$ y $Q$ son colineales al igual que $B$ y $C.$
Así, $\frac{|BP|}{|CQ|}=\frac{|BL|}{|CL|}$, entonces $\frac{BP}{QC}=\frac{BL}{LC}.$
Por lo tanto, podemos concluir que:$$\frac{BL}{LC}=\frac{AB\;\sen(\angle BAL)}{CA\;\sen(\angle LAC)}.$$
\end{dem}

%%%%%%%%%%%%%%%
%*************************Detallar el uso de las orientaciones. 
%%%%%%%%%%%%%%%

\begin{cor}[Teorema de la bisectriz]\index{Teorema ! de la bisectriz}
Sea $\triangle ABC$ y $L\in\overline{BC}$ tal que $AL$ es bisectriz del $\angle BAC$, entonces:
$$\frac{BL}{LC}=\frac{AB}{CA}.$$
\end{cor}
\begin{pba}
Como $L\in\overline{BC}$, entonces por el Teorema~\ref{TGB}, tenemos que: 
$$\frac{BL}{LC}=\frac{AB\;\sen(\angle BAL)}{CA\;\sen(\angle LAC)}.$$

Ahora, como $AL$ es bisectriz del $\angle BAC$, $\angle BAL=\angle LAC$ por tanto $\sen\angle BAC=\sen\angle LAC.$
De ésto podemos concluir que:
$$\frac{BL}{LC}=\frac{AB}{CA}.$$
\end{pba}

\section{Ángulos en la circunferencia}
Para el desarrollo de esta sección introduciremos la siguiente notación: Dada una circunferencia con centro en $O$ y radio $r\in\mathbb{R^{+}}$ escribiremos $\mathcal{C}(O,r)$.

\subsection{Ángulos inscritos en la circunferencia}
\begin{df}
Sea $\mathcal{C}(O, r)$ y $\{A,B\}\subseteq\mathcal{C}(O,r)$, el $\angle AOB$ será llamado \textcolor{red}{\bf ángulo central}\index{ángulo ! central}. Si $P\in\mathcal{C}(O,r)$, entonces $\angle APB$ es un \textcolor{red}{\bf ángulo inscrito}\index{ángulo ! inscrito}. 
\end{df}

\begin{df}
Sea $\mathcal{C}(O,r)$, $\{A,B\}\subseteq\mathcal{C}(O,r)$ y $\alpha\in\mathbb{R}^{+}$. El conjunto de puntos $P$ en $\mathcal{C}(O,r)$ para los que se cumple que: Si $0<\angle AOB=\alpha$, entonces $0\leq\angle AOP\leq\alpha$, es el \textcolor{red}{\bf arco de circunferencia dirigido}\index{arco de circunferencia dirigido} AB y se denotará como $\widehat{AB}.$
Al considerar el arco de circunferencia dirigido, se tiene que: $\widehat{AB}=-\widehat{BA}$.
\end{df}

\begin{prop}\label{P1AIC}
Sea $\mathcal{C}(O,r)$ una circunferencia, $\{A,B\}\subseteq\mathcal{C}(O,r)$ y $P\in\widehat{BA}\backslash\{B, A\}$, entonces $\angle AOB=2\angle APB.$
\end{prop}
\begin{pba}
Sea $\{C\}=\mathcal{C}(O,r)\cap\overline{OA}\backslash\{A\}$ y $\{D\}=\mathcal{C}(O,r)\cap\overline{OB}\backslash\{B\}.$ Entonces tenemos los siguientes casos:

\begin{itemize}
\item $P=C.$

Supongamos que $P=C.$

Consideremos $\triangle BOP$, tenemos que $|OP|=|OB|$ ya que ambos segmentos son radios de $\mathcal{C}(O,r)$, por tanto $\triangle BOP$ es isósceles y así $|\angle OBP|=|\angle OPB|$. Como $|\angle AOB|=|\angle OBP|+|\angle OPB|$, entonces $|\angle AOB|=2|\angle OPB|.$ Ahora, como $P\in\overline{OA}$, tenemos que $|\angle AOB|=2|\angle APB|$ por lo tanto, $\angle AOB=2\angle APB.$

\item $P=D.$

Supongamos que $P=D.$

Consideremos $\triangle AOP$, tenemos que $|OP|=|OA|$ ya que ambos segmentos son radios de $\mathcal{C}(O,r)$, por tanto $\triangle BOP$ es isósceles y así $|\angle APO|=|\angle OAP|$. Como $|\angle AOB|=|\angle OAP|+|\angle APO|$, entonces $|\angle AOB|=2|\angle APO|.$ Ahora, como $P\in\overline{OB}$, tenemos que $|\angle AOB|=2|\angle APB|$ por lo tanto, $\angle AOB=2\angle APB.$
%%%%%%%%%%%%%%%%%%%%
%**********************Quizá la prueba de este caso debería ser ejercicio como en el caso 4. COMENTARLO
%%%%%%%%%%%%%%%%%%%%%
\item $P\in\widehat{BC}\backslash\{C\}.$

Supongamos que $P\in\widehat{BC}\backslash\{C\}$ y sea $\overline{PO}\cap\mathcal{C}(O,r)\backslash\{P\}=\{Q\}.$

Primero notemos que: $|\angle APB|=|\angle OPB|-|\angle OPA|.$ Por el caso anterior tenemos que $2|\angle OPB|=|\angle QOB|$ y $2|\angle OPA|=|\angle QOA|$, ahora notemos que $|\angle QOB|=|\angle POD|$ y $|\angle QOA|=|\angle POC|$ ya que son ángulos  opuestos por el vértice, por tanto, $|\angle OPB|=\frac{|\angle POD|}{2}$ Y $|\angle OPA|=\frac{|\angle POC|}{2}.$ Así, tenemos que: $|\angle APB|=\frac{|\angle POD|}{2}-\frac{|\angle POC|}{2}=\frac{|\angle POD|-|\angle POC|}{2}=\frac{|\angle QOB|-|\angle QOA|}{2}=\frac{|\angle AOB|}{2}.$ 

Por lo tanto, podemos concluir que $2\angle APB=\angle AOB.$

\item $P\in\widehat{DA}\backslash\{D\}.$ La prueba es análoga (ver sección de ejercicios, Ejercicio~\ref{P1AICC4}).

\item $P\in\widehat{CD}\backslash\{D\}.$

Supongamos que $P\in\widehat{CD}\backslash\{D\}$ y sea $\overline{PO}\cap\mathcal{C}(O,r)\backslash\{P\}=\{Q\}.$

Por el primer caso tenemos que $\angle QOB=2\angle QPB$ y $\angle AOQ=2\angle APQ$, entonces $\angle AOQ+\angle QOB=2\angle APQ+2\angle QPB=2(\angle APQ+\angle QPB)$ por tanto, $\angle AOB=2\angle APB.$
\end{itemize}
\end{pba}
%%%%%%%%%%%%%%%
%***************************Revisar los detalles. 
%%%%%%%%%%%%%%%%
\begin{cor}\label{C1AIC}
Sea $\mathcal{C}(O,r)$ una circunferencia, $\{A,B\}\subseteq\mathcal{C}(O,r)$ y $\{P,Q\}\subseteq\widehat{BA}\backslash\{B, A\}$, entonces $\angle APB=\angle AQB.$
\end{cor}
\begin{pba}
Por la Proposición~\ref{P1AIC} sabemos que $2\angle APB=\angle AOB$ y $2\angle AQB=\angle AOB$, entonces $2\angle APB=2\angle AQB$ por lo tanto, $\angle APB=\angle AQB.$ 
\end{pba}

\begin{cor}
Sea $\mathcal{C}(O,r)$ una circunferencia, $\{A,B\}\subseteq\mathcal{C}(O,r)$, $P\in\widehat{BA}\backslash\{B, A\}$ y $Q\in\widehat{AB}\backslash\{A,B\}$, entonces $\angle APB+\angle BQA=2\perp.$
\end{cor}
\begin{pba}
Por la Proposición~\ref{P1AIC} tenemos que: 
$$\angle APB+\angle AQB=\frac{\angle AOB}{2}+\frac{\angle BOA}{2}=\frac{\angle AOB+\angle BOA}{2}=\frac{4\perp}{2}=2\perp.$$
\end{pba}

\subsection{Ángulos semi-inscritos en la circunferencia}

$\bullet$ ley senos.

$\bullet$ construcciones regla y compás

\begin{df}
Sea $\mathcal{C}(O,r)$ y $P\in\mathcal{C}(O,r)$, $t$ es \textcolor{red}{\bf tangente}\index{tangente} a $\mathcal{C}(O,r)$ por $P$ si y solamente si $P\in t$ y $t$ es ortogonal a $\overline{OP}.$
\end{df}

\begin{df}
Sea $\mathcal{C}(O,r)$ y $t$ una tangente a $\mathcal{C}(O,r)$ por $T\in\mathcal{C}(O,r)$. Sea $A\in\mathcal{C}(O,r)\backslash\{T\}$ arbitraria, entonces definimos el \textcolor{red}{\bf ángulo semi-inscrito}\index{ángulo ! semi-inscrito} como el águlo cuyo vértice es el punto $T$ y que inicia en la recta $\overline{TA}$ y termina en la recta $t$, al cual denotaremos como $\angle (\overline{TA}\longrightarrow^\text{T}t)$.
\end{df}
%%%%%%%%%%%%%%%%%%%%%%
%***************************Revisar notación. 
%%%%%%%%%%%%%%%%%%%%%%
\begin{prop}\label{PTP}
Si $t$ es tangente a $\mathcal{C}(O,r)$ por $P$, entonces $t\cap\mathcal{C}(O,r)=\{P\}.$
\end{prop}
\begin{pba}
Supongamos para generar una contradicción que $t\cap\mathcal{C}(O,r)=\{P,Q\}$ donde $P\neq Q$, por definición $\angle (t \longrightarrow^\text{P}\overline{OP})=\perp$. Observemos que $\triangle OPQ$ tiene dos lados congruentes ya que $|OP|=|OQ|$. Entonces $|\angle OQP|=|\angle OPQ|=\perp$ por lo cual $\overline{OP}$ es paralela a $\overline{OQ}$ pero $\overline{OP}\cap\overline{OQ}=\{O\}$ lo cual no es posible. 

Por lo tanto $P=Q$ y así $t\cap\mathcal{C}(O,r)=\{P\}.$
\end{pba}

Ya que se definió qué es una recta tangente a una circunferencia por un punto, debería ser natural hacernos la siguiente pregunta. 

¿Cómo construir una recta tangente a $\mathcal{C}(O,r)$ por cualquier punto $P$ del plano?

Observemos que se tienen los siguientes dos casos:
\begin{itemize}
\item Caso 1: $P\in\mathcal{C}(O,r)$. 

Basta trazar la recta ortogonal $t$ a $OP$ por $P$. De esta manera, $t$ es la recta tangente a $\mathcal{C}(O,r)$ por $P$.

\item Caso 2: $P\notin\mathcal{C}(O,r)$.

Ahora, notemos que aquí tenemos dos subcasos:
\begin{enumerate}
\item $|OP|<r$.

En este caso no es posible construir una recta tangente a $\mathcal{C}(O,r)$ por $P$, veamos por qué. Supongamos que existe $l$ tangente a $\mathcal{C}(O,r)$ por $P$, como  $|OP|<r$ tenemos que $|\mathcal{C}(O,r)\cap l|=2$ pero esto contradice la Proposición ~\ref{PTP}.

\item $r<|OP|$.

Consideremos el segmento $OP$, sea $Q$ el punto medio de $OP$ y construyamos $\mathcal{C}(Q,|QP|)$. Entonces tenemos que $\mathcal{C}(O,r)\cap\mathcal{C}(Q,|QP|)=\{R,S\}$, como $R\in\mathcal{C}(Q,|QP|)$, por la Proposición~\ref{P1AIC} sabemos que $2|\angle ORP|=|\angle OQP|$ puesto que $Q\in OP$, $|\angle OQP|=2\perp$ por lo tanto $2|\angle ORP|=2\perp$, entonces $|\angle ORP|=\perp$ y así tenemos que $OR$ es ortogonal a $RP$.

De hecho, de manera análoga se tiene que $OS$ es ortogonal a $SP$. Es decir, si $P\notin\mathcal{C}(O,r)$, se pueden construir dos rectas tangentes a $\mathcal{C}(O,r)$ por $P$.  
\end{enumerate}
\end{itemize}

\begin{prop}
Sea $\mathcal{C}(O,r)$, $T\in\mathcal{C}(O,r)$ y $t$ la recta tangente a $\mathcal{C}(O,r)$ por $T$, entonces para $A\in\mathcal{C}(O,r)\backslash\{T\}$ arbitraria, $2 \angle (\overline{TA}\longrightarrow^\text{T}t)=\angle AOT$.
\end{prop}
\begin{pba} 
Sea $A\in\mathcal{C}(O,r)\backslash\{T\}$ y $\{P\}=\mathcal{C}(O,r)\cap\overline{OT}\backslash\{T\}$, entonces $\angle (\overline{PT}\longrightarrow^\text{T}t)=\perp$ ya que $t$ es la recta tangente a $\mathcal{C}(O,r)$ por $T$.
Ahora, observemos que $\angle PTA+\angle (\overline{TA}\longrightarrow^\text{T}t)=\perp$. Por la Proposición~\ref{P1AIC}, tenemos que $\frac{\angle POA}{2}+\angle (\overline{TA}\longrightarrow^\text{T}t)=\perp$. Así que, $\angle (\overline{TA}\longrightarrow^\text{T}t)=\perp-\frac{\angle POA}{2}.$
De este modo, nos resta demostrar que $\perp-\frac{\angle POA}{2}=\frac{\angle AOT}{2}.$

En enfecto, notemos que $\angle POA+\angle AOT=\angle POT=2\perp$ ya que $O\in\overline{PT}$. Entonces, $\frac{\angle POA}{2}+\frac{\angle AOT}{2}=\perp$, por ello $\perp-\frac{\angle POA}{2}=\frac{\angle AOT}{2}.$ y así $\angle (\overline{TA}\longrightarrow^\text{T}t)=\frac{\angle AOT}{2}.$

Por lo tanto, $2 \angle (\overline{TA}\longrightarrow^\text{T}t)=\angle AOT$.
\end{pba}

\begin{prop}\label{PLGA}
Sea $\mathcal{C}(O,r)$ una circunferencia y $\{A,B\}\subseteq\mathcal{C}(O,r)$ dos puntos fijos. El lugar geométrico de los puntos $P$ tales que $|\angle APB|=\alpha\;\;(0<\alpha<2\perp)$  son dos arcos de circunferencia del mismo radio que contienen a $A\;y\;B.$
\end{prop}
\begin{pba}

\end{pba}

Finalmente, daremos otra prueba del Teorema~\ref{LDS} haciendo uso de lo recientemente aprendido. 
\begin{teo}[Ley de senos]\label{LDS2}
Sea $\triangle ABC$, si $\mathcal{C}(O,r)$ es la circunferencia que inscribe al $\triangle ABC$, entonces tenemos que:
$$\frac{|AC|}{\sen(\angle ABC)}=\frac{|BA|}{\sen(\angle BCA)}=\frac{|CB|}{\sen(\angle CAB)}=2r.$$
\end{teo}
\begin{dem}
Sea $l=\overline{OC}$ y $l\cap\mathcal{C}(O,r)=\{C,D\}.$ Consideremos $\triangle ADC$, como $CD$ es diámetro, entonces $\triangle ADC$ es rectángulo (ver sección de ejercicios, Ejercicio~\ref{LGCE}), por lo cual $|\angle DAC=\perp$ y como $\angle ABC$ y $\angle ADC$ abren el mismo arco, por el Corolario~\ref{C1AIC} sabemos que  $\angle ABC=\angle ADC$ así que 
$$\sen(\angle ABC)=\sen(\angle ADC)=\frac{|AC|}{|DC|}=\frac{|AC|}{2r}.$$

Por lo tanto, $$2r=\frac{|AC|}{\sen(\angle ABC)}.$$

Sea $m=\overline{OA}$ y $m\cap\mathcal{C}(O,r)=\{A,E\}.$ Consideremos $\triangle ABE$, como $AE$ es diámetro, entonces $\triangle ABE$ es rectángulo (ver sección de ejercicios, Ejercicio~\ref{LGCE}), por lo cual $|\angle ABE=\perp$ y como $\angle BEA$ y $\angle BCA$ abren el mismo arco, por el Corolario~\ref{C1AIC} sabemos que  $\angle BEA=\angle BCA$ así que 
$$\sen(\angle BCA)=\sen(\angle BEA)=\frac{|BA|}{|AE|}=\frac{|BA|}{2r}.$$

Por lo tanto, $$2r=\frac{|BA|}{\sen(\angle BCA)}.$$

Análogamente se prueba que $2r=\frac{|CB|}{\sen(\angle CAB)}$ (ver sección de ejercicios, Ejercicio~\ref{LDS2E}).

Entonces podemos concluir que 
$$\frac{|AC|}{\sen(\angle ABC)}=\frac{|BA|}{\sen(\angle BCA)}=\frac{|CB|}{\sen(\angle CAB)}=2r.$$
\end{dem}
%%%%%%%%%%%%%%%%%%
%************************ Revisar, me parece que hay casos. 
%%%%%%%%%%%%%%%%%%

\subsection*{Ejercicios}
\begin{enumerate}

\item En la Proposición~\ref{P1AIC} (página \pageref{P1AIC}), probar el  caso en el que $P\in\widehat{DA}\backslash\{D\}$.\label{P1AICC4}
\item Sea $\mathcal{C}(O,r)$ una circunferencia y $\{A,B\}\subseteq\mathcal{C}(O,r)$ dos puntos fijos. Demostrar que el lugar geométrico de los puntos $P$ tales que $|\angle APB|=\perp$, es una circunferencia de diámetro $AC$.\label{LGCE}   
\item En el Teorema~\ref{LDS2} (página \pageref{LDS2}), probar que $2r=\frac{|CB|}{\sen(\angle CAB)}$.\label{LDS2E}
%%%%%%%%%%%%%%%%%%
%*************************Definición de diámetro ¿Es necesaria?. 
%%%%%%%%%%%%%%%%%%%%%
\end{enumerate}

\section{Puntos y rectas importantes sobre el triángulo}
\subsection{Las medianas}
\begin{df}
Sea $\triangle ABC$, $L$ el punto medio de $BC$, $M$ el punto medio de $CA$ y $N$ el punto medio de $AB$.
Una \textcolor{red}{\bf mediana}\index{mediana} del $\triangle ABC$ es la recta determinada por un vértice y el punto medio del lado opuesto, es decir, $\overline{AL}$, $\overline{BM}$ y $\overline{CN}$ son las medianas del $\triangle ABC$.
\end{df}

Las medianas generan a uno de los puntos importantes del triángulo para saber más acerca de éste, veamos el siguiente teorema. 

\begin{teo}\label{LMC}
Sea $\triangle ABC$, entonces las medianas del $\triangle ABC$ son rectas concurrentes.
\end{teo}
\begin{dem}
Sea $\triangle ABC$ (ordenado levógiramente), $L$ el punto medio de $BC$, $M$ el punto medio de $CA$ y $N$ el punto medio de $AB$. Así, $BL=LC$, $CM=MA$ y $AN=NB$, de esto se sigue que $\frac{LC}{BL}=1=\frac{NB}{AN}$, entonces por el Teorema~\ref{Thales1} $\overline{LN}$ es paralela a $\overline{CA}$.
Como $|\angle ABC|=|\angle NBL|$ (pues $A$, $N$ y $C$, $L$ son colineales) y $|\angle NLB|=|\angle ACB|$ (por ser $\overline{LN}$ y $\overline{CA}$ paralelas), $\triangle NBL\cong\triangle ABC$ \textbf{cs(AA)} por tanto, $\frac{|BL|}{|BC|}=\frac{|NB|}{|AB|}=\frac{|LN|}{|CA|}$. Ahora notemos que como $L$ es punto medio de $BC$, $|BC|=2|BL|$ así que $\frac{|LN|}{|CA|}=\frac{|BL|}{|BC|}=\frac{1}{2}$. Finalmente, vamos a establecer que $0<ML$, $0<LN$ y $0<NM$ con esto tenemos que $\frac{LN}{CA}=\frac{1}{2}.$
Análogamente se tiene que: $\frac{NM}{BC}=\frac{1}{2}=\frac{ML}{AB}$. 

Como $\frac{MA}{CM}=1=\frac{LC}{BL}$, por el Teorema~\ref{Thales1} $\overline{ML}$ es paralela a $\overline{AB}$ por lo que $\triangle ABC\cong\triangle MLC$ \textbf{cs(AA)} pues $|\angle ACB|=|\angle MCL|$ y $|\angle CLM|=|\angle CBA|$, por tanto, $\frac{|ML|}{|AB|}=\frac{|LC|}{|BC|}=\frac{|CM|}{|CA|}=\frac{1}{2}$ ($M$ es punto medio de $CA$). Y como $0<ML$, $\frac{ML}{AB}=\frac{1}{2}$. 

Y de la misma manera para la última razón, como $\frac{NB}{AN}=1=\frac{MA}{CM}$, por el Teorema~\ref{Thales1} $\overline{NM}$ es paralela a $\overline{BC}$ por lo que $\triangle ABC\cong\triangle ANM$ \textbf{cs(AA)} pues $|\angle NAM|=|\angle BAC|$ y $|\angle MNA|=|\angle CBA|$, por tanto, $\frac{NM}{BC}=\frac{MA}{CA}=\frac{AN}{AB}=\frac{1}{2}$ ($N$ es punto medio de $AB$). Y como $0<NM$, $\frac{NM}{BC}=\frac{1}{2}$.

Sean $m_{1}$ la mediana por $A$, $m_{2}$ la mediana por $B$ y $m_{3}$ la mediana por $C$. Debemos probar que $m_{1}\cap m_{2}\cap m_{3}\neq\emptyset$. 
En efecto, sea $m_{1}\cap m_{2}=\{G\}$, consideremos $\triangle MLG$ y $\triangle BAG$, entonces tenemos que $\triangle MLG\cong\triangle BAG$ \textbf{cs(AA)} ya que $|\angle LGM|=|\angle AGB|$ por ser opuestos por el vértice y $|\angle BML|=|\angle ABM|$ pues $\overline{ML}$ es paralela a $\overline{AB}$, por tanto, $\frac{|ML|}{|BA|}=\frac{|LG|}{|AG|}=\frac{|GM|}{|GB|}$. Ahora, por la correspondencia de los ángulos de ambos triángulos se considerarán ordenados levógiramente, es decir, $\frac{ML}{BA}=\frac{LG}{AG}=\frac{GM}{GB}$.

Por otra parte, sea $m_{1}\cap m_{3}=\{P\}$ consideremos $\triangle LNP$ y $\triangle ACP$, entonces tenemos que $\triangle LNP\cong\triangle ACP$ \textbf{cs(AA)} ya que $|\angle PLN|=|\angle PAC|$ y $|\angle LNP|=|\angle ACP|$ pues $\overline{LN}$ es paralela a $\overline{CA}$, por tanto, $\frac{|LN|}{|AC|}=\frac{|NP|}{|CP|}=\frac{|LP|}{|AP|}$. Como se hizo anteriormente, debido a la correspondencia que hay entre los ángulos de estos triángulos también se consideran ordenados levógiramente, así se tiene que $\frac{LN}{AC}=\frac{NP}{CP}=\frac{LP}{AP}$.

Para terminar, bastará demostrar que $P=G$. Notemos que con respecto al orden inicial de los triángulos $\triangle ABC$ y $\triangle MLN$ (levógiro), tenemos que:

$\frac{ML}{BA}=-\frac{1}{2}$ y $\frac{LN}{AC}=-\frac{1}{2}$, entonces $\frac{LG}{AG}=-\frac{1}{2}=\frac{LP}{AP}$, por lo cual $\frac{LG}{GA}=\frac{1}{2}=\frac{LP}{PA}$. Por tanto, $\frac{LG}{GA}=\frac{LP}{PA}$ pero esto es posible si y solamente si $P=G$.

Así, concluimos que $m_{1}\cap m_{2}\cap m_{3}=\{G\}\neq\emptyset.$
\end{dem}

Recordemos que en uno de los párrafos anteriores se mencionó que el punto de intersección de las medianas de un triángulo es uno de los puntos importantes que hay sobre éste. A tal punto se le denotará con la letra $G$ y es llamado el \textcolor{red}{\bf gravicentro}\index{gravicentro}, \textcolor{red}{\bf centro de gravedad}\index{centro ! de gravedad}, \textcolor{red}{\bf centroide}\index{centroide} o \textcolor{red}{\bf baricentro}\index{baricentro} del triángulo.

\subsection{Las bisectrices}
Otras rectas importantes del triángulo son las bisectrices. Para caracterizarlas daremos una breve introducción. 

\begin{df}
Sea $l$ una recta en el plano y $P$ un punto en el plano, la \textcolor{red}{\bf distancia}\index{distancia} de $P$ a $l$ que se denotará como $d(P,l)$ es la longitud del segmento de recta que es ortogonal a $l$ por $P$.
\end{df}

\begin{df}\label{db}
Sean $l$ y $m$ dos rectas distintas en el plano, la \textcolor{red}{\bf bisectriz}\index{bisectriz} de $l$ y $m$ es el lugar geométrico de los puntos en el plano que equidistan de $l$ y $m$. 
\end{df}

\begin{lema}\label{lb}
Sean $\triangle ABC$ y $\triangle DEF$ triángulos rectángulos tales que sus hipotenusas son congruentes y tienen un par de lados congruentes, entonces $\triangle ABC\equiv\triangle DEF$.
\end{lema}

\begin{pba}
Supongamos sin perder generalidad que $|\angle ABC=\perp|$, $|\angle DEF|=\perp$ y que $|AB|=|DE|$, entonces $|CA|=|FD|$. Así que aplicando el Teorema~\ref{TeoPitágoras} a los triángulos $\triangle ABC$ y $\triangle DEF$, tenemos que: 
$$AC^{2}=AB^{2}+BC^{2}\;\;\;\;\;\;\;\;\;\;\;\;\;\;\;\;DF^{2}=DE^{2}+EF^{2}$$
Entonces, $BC^{2}=AC^{2}-AB^{2}=DF^{2}-DE^{2}=EF^{2}$ así $BC=EF$.

Por lo tanto, $\triangle ABC\equiv\triangle DEF$ \textbf{cc(LLL)}.
\end{pba}

\begin{teo}
Las bisectrices de $l$ y $m$ son las rectas que bisecan el ángulo formado por $l$ y $m$.
\end{teo}
\begin{dem}
Sea $P$ en una de las bisectrices de $l$ y $m$, $n$ la recta ortogonal a $l$ por $P$ y $t$ la ortogonal a $m$ por $P$. Consideremos $n\cap l=\{A\}$, $t\cap m=\{B\}$ y $l\cap m=\{O\}$, como $P$ está en una de las bisectrices de $l$ y $m$, de la Definición~\ref{db} se sigue que $|AP|=|PB|$, además $|PO|=|PO|$ y por construcción $|\angle PAO|=|\angle OBP|=\perp$, así $\triangle APO$ y $\triangle BPO$  son triángulos rectángulos con esto tenemos todas las hipotesis del Lema~\ref{lb}, por tanto $\triangle APO\equiv\triangle BPO$, entonces $|\angle AOP|=|\angle POB|$.
\end{dem}

Antes de continuar haremos una importante observación. 
\begin{obs}
Dado un $\triangle ABC$ al considerar un par de sus lados, la Definición~\ref{db} permite distinguir dos tipos de bisectrices: Una \textcolor{red}{\bf bisectriz interna}\index{bisectriz ! interna} es aquella recta que biseca algún ángulo interno del $\triangle ABC$, mientras que una \textcolor{red}{\bf bisectriz externa}\index{bisectriz ! externa} es aquella que biseca algún ángulo externo del $\triangle ABC$.
\end{obs}

\begin{teo}
Sea $\triangle ABC$, entonces las bisectrices internas del $\triangle ABC$ concurren. 
\end{teo}
\begin{dem}
Sean $b_{A}$, $b_{B}$ y $b_{C}$ las bisectrices internas por el vértice $A$, $B$ y $C$ respectivamente. Consideremos $b_{A}\cap b_{B}=\{I\}$, entonces por la Definición~\ref{db} tenemos que $d(I,AB)=d(I,CA)$ y $d(I,AB)=d(I,BC)$, así concluimos que $d(I,CA)=d(I,BC)$ y por lo tanto $I\in b_{C}$.
Con esto se tiene que $b_{A}\cap b_{B}\cap b_{C}=\{I\}$
\end{dem}

\begin{teo}\label{TBEII}
Sea $\triangle ABC$, $b_{A'}$ y $b_{C'}$ las bisectrices externas por $A$ y $C$ respectivamente y $b_{B}$ las bisectriz interna por $B$, entonces $b_{A'}\cap b_{C'}\cap b_{B}\neq\emptyset$.
\end{teo}
\begin{dem}
Sea $b_{A'}\cap b_{C'}=\{E_{B}\}$, entonces por la Definición~\ref{db} $d(E_{B},AB)=d(E_{B},CA)$ y $d(E_{B},CA)=d(E_{B},BC)$, por lo cual $d(E_{B},AB)=d(E_{B},BC)$. Así concluimos que $E_{B}\in b_{B}$. Por lo tanto, $b_{A'}\cap b_{C'}\cap b_{B}=\{E_{B}\}\neq\emptyset$
\end{dem}

Al punto donde concurren las bisectrices internas del $\triangle ABC$ se le denotará por $I$ y se conoce como \textcolor{red}{\bf incentro}\index{incentro}, además existe una circunferencia con centro este punto que es tangente a los lados del triángulo y se queda completamente contenida en él, conocida como \textcolor{red}{\bf incírculo}\index{incírculo}. Por otra parte, el Teorema~\ref{TBEII} nos dice que las bisectrices externas de dos ángulos exteriores de una triángulo concurren con la bisectriz interna por el vértice restante y a este punto de concurrencia se le conoce como \textcolor{red}{\bf excentro}\index{excentro}. 
\subsection{Las mediatrices}

\begin{df}\label{dm}
Sea $PQ$ un segmento de recta y $R$ el punto medio de $PQ$, entonces la \textcolor{red}{\bf mediatriz}\index{mediatriz} de $PQ$ es la recta ortogonal a $PQ$ por $R$. Otra forma de definir a la mediatriz de un segmento es como el lugar geométrico de los puntos $S$ tal que la dictancia a cada extremos del segmento es la misma, es decir, $SP=SQ$.
\end{df}

Dado un $\triangle ABC$ y $L$, $M$, $N$ los puntos medios de los lados $AB$, $BC$, $CA$ respectivamente. Entonces las mediatrices del $\triangle ABC$ son las rectas ortogonales a $AB$ por $L$, a $BC$ por $N$ y a $CA$ por $M$.

\begin{teo}\label{LMDTC}
Sea $\triangle ABC$, entonces las mediatrices del $\triangle ABC$ concurren.
\end{teo}
\begin{dem}
Sean $m_{AB}$, $m_{BC}$ y $m_{CA}$ las mediatrices de los lados $AB$, $BC$ y $CA$ respectivamente. Debemos probar que $m_{AB}\cap m_{BC}\cap m_{CA}\neq\emptyset$, sea $m_{AB}\cap m_{BC}=\{O\}$. Como $O\in m_{AB}$ por la Definición~\ref{dm} se tiene que $|OA|=|OB|$, de igual manera por estar $O\in {BC}$, $|OB|=|OC|$, entonces $|OA|=|OC|$, esto es, $O\in m_{CA}$.

Por lo tanto, $m_{AB}\cap m_{BC}\cap m_{CA}=\{O\}\neq\emptyset$. 
\end{dem}
 
Además llamaremos al punto $O$ \textcolor{red}{\bf circuncentro}\index{circuncentro} del $\triangle ABC$, notemos que $\mathcal{C}(O,|OA|)$ contiene a los vértices del $\triangle ABC$. Y a $\mathcal{C}(O,|OA|)$ la llamaremos \textcolor{red}{\bf circuncircunferencia}\index{circuncircunferencia} o la circunferencia que inscribe al $\triangle ABC$. 
\subsection{Las alturas} 
Ahora, veremos otras de las rectas importantes del triángulo, las cuáles ya hemos tratado previamente, las alturas de un triángulo (véase Definición~\ref{ADUT}).

\begin{teo}
Sea $\triangle ABC$, entonces las alturas del $\triangle ABC$ concurren. 
\end{teo}
\begin{dem}
Sea $h_{A}$, $h_{B}$ y $h_{C}$ las alturas por $A$, $B$ y $C$ respectivamente. 

Construir $a$ la recta paralela  a $BC$ por $A$, $b$ la paralela a $CA$ por $B$ y $c$ la paralelala a $AB$ por $C$. Sean $a\cap b=\{C'\}$, $b\cap c=\{A'\}$ y  $c\cap a=\{B'\}$. 
Observemos que $\square ACBC'$ es un paralelogramos (ver Definición~\ref{paralelogramo}) pues $\{A,C'\}\subset a$, $\{B,C'\}\subset b$ y $a\parallel BC$, $b\parallel AC$. Entonces por el Ejercicio~\ref{EPLI}(página \pageref{EPLI}), $|AC|=|BC'|$ y $|AC'|=|BC|$. Análogamente tenemos que $\square AB'CB$ es un paralelogramo, entonces $|AB'|=|CB|$, $|B'C|=|AB|$ y $\square ACA'B$ es un paralelogramo , entonces $|AC|=|A'B|$, $|CA'|=|BA|$. 

Por lo tanto, $|A'B|=|AC|=|BC'|$, $|AB'|=|BC|=|AC'|$ y $|B'C|=|AB|=|CA'|$. Ahora, como $\{A',B,C'\}\subset b$ y $|A'B|=|BC'|$, entonces $B$ está en la mediatriz de $A'C'$, de igual manera como $\{B',A,C'\}\subset a$ y $|B'A|=|AC'|$, entonces $A$ está en la mediatriz de $B'C'$ y como $\{B',C,A'\}\subset c$ y $|B'C|=|CA'|$, entonces $C$ está en la mediatriz de $B'A'$. 

La mediatriz del $A'C'$ es la recta ortogonal al segmento $A'C'$ por $B$ y además $\overline{AC}\parallel b$, entonces la mediatriz del $A'C'$ es ortogonal a $CA$ por $B$. Por tanto la mediatriz de $A'C'$ es $h_{B}$, análogamente la mediatriz de $B'C'$ es $h_{A}$ y la mediatriz de $B'A'$ es $h_{C}$. 
Para concluir, recordemos que por el Teorema~\ref{LMDTC} tenemos que la mediatrices del $\triangle A'B'C'$ son concurrentes, entonces $h_{A}$, $h_{B}$ y $h_{C}$ son concurrentes. 

Por lo tanto,  $h_{A}\cap h_{B}\cap h_{C}\neq\emptyset$.
\end{dem}

Al punto donde concurren las alturas los llamamos \textcolor{red}{\bf ortocentro}\index{ortocentro} y se denotará con la letra $H$. 
\section{La recta de Euler}
\begin{teo}\label{TRE}
Sea $\triangle ABC$, entonces el ortocentro, el circuncentro y el gravicentro del $\triangle ABC$ son colineales. A la recta que contiene a estos tres puntos se le conoce como \textcolor{red}{\bf Recta de Euler}\index{Recta de Euler} del $\triangle ABC$.
\end{teo}
\begin{dem}
Sea $\triangle ABC$, construir $h_{A}$ y $h_{B}$ y sean $h_{A}\cap\overline{BC}=\{D\}$, $h_{B}\cap\overline{CA}=\{E\}$ y $h_{A}\cap h_{B}\cap h_{C}=\{H\}$. Construir $m_{BC}$ la mediatriz de $BC$ y $m_{CA}$ la mediatriz de $CA$, sean $m_{BC}\cap m_{CA}\cap m_{AB}=\{O\}$, $m_{BC}\cap\overline{BC}=\{L\}$ y $m_{CA}\cap\overline{CA}=\{M\}$.

Afirmación: $|AH|=2|OL|$, arguméntemos por qué.

Primero notemos que como $L$ y $M$ son puntos medios de $BC$ y $CA$, tenemos que $BL=LC$ y $CM=MA$, entonces $\frac{LB}{CL}=\frac{MA}{CM}$ así que por el Teorema~\ref{Thales1} $\overline{LM}\parallel\overline{BA}$.
Consideremos $\triangle ABC$ y $\triangle MLC$, entonces $\triangle ABC\cong\triangle MLC$ \textbf{cs(AA)} ya que $|\angle ACB|=|\angle MCL|$ (pues $M$, $A$ y $L$, $B$ son colineales) y  $|\angle BAC|=|\angle LMC|$ y (por ser $\overline{BA}\parallel\overline{LM}$) por tanto, $\frac{|AB|}{ML|}=\frac{|BC|}{|LC|}=\frac{|CA|}{|CM|}$ y como $M$ es punto medio de $CA$, $2|CM|=|CA|$, así  $\frac{|AB|}{ML|}=\frac{|CA|}{|CM|}=2$, entonces $\frac{|AB|}{|LM|}=2$, por tanto $|AB|=2|LM|$.

Ahora observemos que $\triangle ABH\cong\triangle LMO$ \textbf{cs(AA)} pues $|\angle BAH|=|\angle MLO|$ Y $|\angle ABH|=|\angle LMO|$. 
%%%%%%%%%%%
%*****************
%%%%%%%%%%%
Por tanto, $\frac{|AB|}{|LM|}=\frac{|BH|}{|MO|}=\frac{|AH|}{|OL|}$ y como ya sabemos $|AB|=2|LM|$, así que $\frac{|AH|}{|OL|}=\frac{|AB|}{|LM|}=2$ con lo que concluimos que $|AH|=2|OL|$.

Sea $\overline{AL}\cap\overline{BM}\cap\overline{CN}=\{G\}$ (donde $N$ es punto medio de $AB$) y tomemos en cuenta que $\overline{AL}$ es transversal a $m_{BC}$ y $h_{A}$ que son rectas paralelas, de esto se tiene que $|\angle GAH|=|\angle GLO|$, además como ya sabemos $|AH|=2|OL|$ y $|AG|=2|GL|$ (véase Teorema~\ref{LMC}), entonces $\triangle GAH\cong\triangle GLO$ \textbf{cs(LAL)}, por tanto $|\angle AGH|=|\angle LGO|$ con lo que tenemos que $H$, $G$ y $O$ son colineales. 
\end{dem}

En el último párrafo de la demostración del Teorema~\ref{TRE} vimos que $\triangle GAH\cong\triangle GLO$, entonces $\frac{|GA|}{|GL|}=\frac{|AH|}{|LO|}=\frac{|GH|}{|GO|}$ y como $2|LO|=|AH|$, tenemos que $|GH|=2|GO|$.

Consideremos el $\triangle LMN$, sabemos que $\overline{NM}$ es paralela a $\overline{BC}$ y $\overline{AB}$ es paralela a $\overline{ML}$, así que $\square LBNM$ es un paralelogramo, sea $\overline{LN}\cap\overline{BM}=\{X\}$, entonces por el ejercicio~\ref{IDDP} (página \pageref{IDDP}) tenemos que $|LX|=|XN|$ y $|BX|=|XM|$, entonces $\overline{MX}=\overline{MB}$ (pues $M$ y $X$ son colineales) es mediana del $\triangle LMN$, de la misma manera se puede probar que $\overline{AL}$ y $\overline{CN}$, es decir, las medianas del $\triangle ABC$ también son medianas del $\triangle LMN$. Por lo tanto, $G$ es el gravicentro del $\triangle ABC$ y del $\triangle LMN$.

Por otro lado tenemos que $m_{BC}$ es ortogonal a $\overline{BC}$ por $L$ ya que es mediatriz del $BC$ y como $\overline{NM}$ es paralela a $\overline{BC}$, entonces $m_{BC}$ es ortogonal a $\overline{MN}$ y $L\in m_{BC}$, por lo cual $m_{BC}=h_{L}$. Análogamente $m_{CA}=h_{M}$ y $m_{AB}=h_{N}$, es decir, las mediatrices del $\triangle ABC$ son las alturas del $\triangle LMN$. Por lo tanto, el circuncentro del $\triangle ABC$ es el ortocentro del $\triangle LMN$. 

Observemos además que el circuncentro del $\triangle LMN$ (llamémosle $O_{1}$) pertenece a la recta de Euler del $\triangle ABC$ y además cumple que $|GO|=2|GO_{1}|$.

Recordemos que al inicio de este breve análisis teníamos que $|GH|=2|GO|$, por lo cual podemos concluir que el circuncentro del $\triangle LMN$ es el punto medio entre el circuncentro del $\triangle ABC$ y el ortocentro del $\triangle ABC$. 
\section{Circunferencia de los nueve puntos}
\begin{teo}[La circunferencia de los nueve puntos] Sea $\triangle ABC$, entonces los pies de las alturas, los puntos medios de cada uno de sus lados y los puntos medios de los segmentos determinados por cada uno de los vértices y el ortocentro se encuentran en una misma circunferencia. A esta circunferencia se le conoce como la \textcolor{red}{\bf circunferencia de los nueve puntos}\index{Circunferencia ! de los nueve puntos} del $\triangle ABC$ que denotaremos como $\mathcal{C}_{9}(\triangle ABC)$.
\end{teo}
\begin{dem}
Sea $\triangle ABC$, $m_{AB}$ la mediatriz de $AB$, $m_{BC}$ la mediatriz de $BC$, $m_{CA}$ la mediatriz de $CA$, $h_{A}\cap h_{B}\cap h_{C}=\{H\}$, $m_{AB}\cap m_{BC}\cap m_{CA}=\{O\}$, $m_{AB}\cap\overline{AB}=\{N\}$, $m_{BC}\cap\overline{BC}=\{L\}$, $m_{CA}\cap\overline{CA}=\{M\}$, $h_{A}\cap\overline{BC}=\{D\}$, $h_{B}\cap\overline{CA}=\{E\}$, $h_{C}\cap\overline{BA}=\{F\}$, $P\in h_{A}$ tal que $|AP|=|PH|$, $Q\in h_{B}$ tal  que $|BQ|=|QH|$ y $R\in h_{C}$ tal que $|CR|=|RH|$. Entonces existe una circunferencia $\mathcal{C}$ tal que $\{L,M,N,P,Q,R,D,E,F\}\subset\mathcal{C}$. 

Consideremos $\triangle LMN$ y sea $X$ el circuncentro del $\triangle LMN$, sabemos que $\mathcal{C}(X,|XL|)$ circunscribe al $\triangle LMN$. Primero recordemos que $\overline{AB}\parallel\overline{LM}$ y $|AB|=2|LM|$, $\overline{BC}\parallel\overline{MN}$ y $|BC|=2|MN|$, $\overline{CA}\parallel\overline{NL}$ y $|CA|=2|NL|$ (véase la demostración del Teorema~\ref{LMC}, página \pageref{LMC}).

Ahora tomemos $\triangle HAB$ y $\triangle HPQ$, como $|HA|=2|HP|$, $|HB|=2|HQ|$  y $|\angle AHB|=|\angle PHQ|$, entonces $\triangle HAB\cong\triangle HPQ$ \textbf{cs(LAL)}, entonces $\frac{|HA|}{|HP|}=\frac{|AB|}{|PQ|}=\frac{|HB|}{|HQ|}=2$, por lo cual $|AB|=2|PQ|$. Y como  $\frac{|HA|}{|HP|}=\frac{|HB|}{|HQ|}$, por el Teorema~\ref{Thales1} $\overline{AB}\parallel\overline{PQ}$, por lo tanto $\overline{LM}\parallel\overline{PQ}$.

Por otra parte, tenemos que $\triangle AHC\cong\triangle APM$ \textbf{cs(LAL)} ya que $|AH|=2|AP|$, $|CA|=2|MA|$ y $|\angle HAC|=|\angle PAM|$, entonces $\frac{|AH|}{|AP|}=\frac{|HC|}{|PM|}=\frac{|CA|}{|MA|}=2$ y por el Teorema~\ref{Thales1} $\overline{HC}\parallel\overline{PM}$. Además como $|BH|=2|BQ|$, $|BC|=2|BL|$ y $|\angle CBH|=|\angle LBQ|$, entonces $\triangle BHC\cong\triangle BQL$ \textbf{cs(LAL)} así que $\frac{|BH|}{|BQ|}=\frac{|HC|}{|QL|}=\frac{|BC|}{|BL|}=2$ y por el Teorema~\ref{Thales1} $\overline{HC}\parallel\overline{QL}$, por tanto $\overline{QL}\parallel\overline{PM}$.

Por lo anterior, podemos considerar el paralelogramo $\square QLM$ y notemos que como $\overline{PQ}\parallel\overline{AB}$ y $\overline{AB}$ es ortogonal a $\overline{HC}$ que a su vez es paralela a $\overline{PM}$, entonces $\overline{PQ}$ es ortogonal $\overline{PM}$, es decir, $|\angle QPM|=\perp$. Así, $\square QLMP$ es un rectángulo y si $\overline{QM}\cap\overline{LP}=\{Y\}$, $|QY|=|YM|$ y $|LY|=|YP|$ con lo que concluimos que $\{Q,L,M,P\}\subset\mathcal{C}(Y,|YL|)$. Además observemos que $E\in\mathcal{C}(Y,|YL|)$ pues $|\angle QEM|=\perp$.

Análogamente, podemos considerar el rectángulo $\square QRMN$ y si $\overline{QM}\cap\overline{RN}=\{Z\}$, entonces $|QZ|=|ZM|$ y $|RZ|=|ZN|$, por tanto $Z=Y$ con lo que tenemos que $\{Q,L,M,P,E,R,N\}\subset\mathcal{C}(Y,|YL|)$. Como existe una única circunferencia que inscribe al $\triangle LMN$, entonces $\mathcal{C}(X,|XL|)=\mathcal{C}(Y,|YL|)$, así que $\{Q,L,M,P,E,R,N\}\subset\mathcal{C}(X,|XL|)$. Como $P$, $L$ y $Y=X$ son colineales, $PL$ es diámetro de $\mathcal{C}(X,|XL|)$ y por ser $|\angle PDL|=\perp$, $D\in\mathcal{C}(X,|XL|)$. También notemos que $|\angle NFR|=\perp$, entonces $F$ pertenece a la circunferencia de diámetro $NR$, por tanto $F\in\mathcal{C}(X,|XL|)$ ya que $R$, $N$ y $Y$ son colineales.

Concluyendo así que $\{L,M,N,P,Q,R,D,E,F\}\subset\mathcal{C}(X,|XL|)$.
\end{dem}

El centro de $\mathcal{C}_{9}(\triangle ABC)$ es el punto medio entre el circuncentro y el ortocentro del $\triangle ABC$, es decir, el circuncentro del triángulo determinado por los tres puntos medios de cada lado del $\triangle ABC$ que se conoce como el \textcolor{red}{\bf triángulo medial}\index{triángulo ! medial} del $\triangle ABC$. 


\subsection*{Ejercicios}
\begin{enumerate}
\item Sean $l$ y $m$ dos rectas en el plano, $b_{1}$ y $b_{2}$ las bisectrices de $\angle (l\longrightarrow m)$ y
$\angle (m\longrightarrow l)$ respectivamente, entonces $b_{1}$ es ortogonal a $b_{2}$.
\end{enumerate}



\chapter{Teorema de Ceva}
\begin{teo}[Teorema de Ceva]\index{Teorema ! de Ceva}\label{Teo de Ceva}
Sea $\triangle ABC$, $L\in\overline{BC}\backslash \{B,C\}$, $M\in\overline{CA}\backslash \{C,A\}$ y $N\in\overline{AB}\backslash \{A,B\}$, entonces $\overline{AL}\cap\overline{BM}\cap\overline{CN}\neq\emptyset$ si y solamente si 
$$\frac{AN}{NB}\cdot\frac{BL}{LC}\cdot\frac{CM}{MA}=1$$.
\end{teo}

\begin{dem}
Sea $\triangle ABC$, $L\in\overline{BC}\backslash \{B,C\}$, $M\in\overline{CA}\backslash \{C,A\}$ y $N\in\overline{AB}\backslash \{A,B\}$.
\begin{enumerate}
\item[($\Rightarrow$)] Supongamos que $\overline{AL}\cap\overline{BM}\cap\overline{CN}\neq\emptyset$, sea $\overline{AL}\cap\overline{BM}\cap\overline{CN}=\{O\}$. 

Construir $a$ la recta paralela a $\overline{BC}$ por $A$. Consideremos $a\cap\overline{CN}=\{C'\}$ y $a\cap\overline{BM}=\{B'\}$. Entonces:

\begin{itemize}
\item $\triangle C'AN\cong\triangle CBN$ \textbf{cs(AA)} ya que $|\angle ANC'|=|\angle BNC|$ y $|\angle CC'A|=|\angle C'CB|$, entonces $\frac{|AN|}{|BN|}=\frac{|C'A|}{|CB|}=\frac{|C'N|}{|CN|}$.
\item $\triangle B'AM\cong\triangle BCM$ \textbf{cs(AA)} pues $|\angle B'MA|=|\angle BMC|$ y $|\angle AB'M|=|\angle CBM|$, entonces $\frac{|CM|}{|AM|}=\frac{|BC|}{|B'A|}=\frac{|BM|}{|B'M|}$.
\item $\triangle BOL\cong\triangle B'OA$ \textbf{cs(AA)} pues $|\angle B'OA|=|\angle BOL|$ y $|\angle AB'O|=|\angle LBO|$. Entonces $\frac{|BL|}{|B'A|}=\frac{|OL|}{|OA|}=\frac{|BO|}{|B'O|}$. 
\item $\triangle COL\cong\triangle C'OA$ \textbf{cs(AA)} ya que $|\angle AOC'|=|\angle LOC|$ y $|\angle OC'A|=|\angle OCL|$, entonces $\frac{|C'A|}{|CL|}=\frac{|OA|}{|OL|}=\frac{|C'O|}{|CO|}$.
\end{itemize}
Por lo tanto, 
$\frac{|BL|}{|B'A|}=\frac{|CL|}{|C'A|}$, entonces $\frac{|BL|}{|CL|}=\frac{|B'A|}{|C'A|}$.

Ahora, consideremos $\triangle ABC$ dirigido levógiramente y tal que $0<\frac{AN}{NB}$, $0<\frac{BL}{LC}$ y $0<\frac{CM}{MA}$. Así tenemos que:

$$\frac{AN}{NB}=\frac{C'A}{BC}\;\;\;\;\;\;\;\;\;\;\;\;\frac{CM}{MA}=\frac{BC}{AB'}\;\;\;\;\;\;\;\;\;\;\;\;\frac{BL}{LC}=\frac{AB'}{C'A}$$

Por lo tanto, $$\frac{AN}{NB}\cdot\frac{BL}{LC}\cdot\frac{CM}{MA}=\frac{C'A}{BC}\cdot\frac{AB'}{C'A}\cdot\frac{BC}{AB'}=1$$.

Observemos que los puntos $L$, $M$ y $N$ no necesariamente se encuentran en los segmentos $BC$, $CA$ y $AB$ respectivamente (como se consideró en la prueba). Veamos un par de ejemplos:

\begin{itemize}
\item Consideremos $\triangle ABC$ dirigido levógiramente y tal que $0<\frac{NA}{AB}$, $0<\frac{BL}{LC}$ y $0<\frac{CA}{AM}$. Entonces:
$$\frac{NA}{NB}=\frac{AC'}{BC}\;\;\;\;\;\;\;\;\;\;\;\;\frac{CM}{AM}=\frac{BC}{B'A}\;\;\;\;\;\;\;\;\;\;\;\;\frac{BL}{LC}=\frac{B'A}{AC'}$$

Por lo tanto, $\frac{NA}{NB}\cdot\frac{BL}{LC}\cdot\frac{CM}{AM}=\frac{AC'}{BC}\cdot\frac{B'A}{AC'}\cdot\frac{BC}{B'A}=1$, entonces

$$\frac{AN}{NB}\cdot\frac{BL}{LC}\cdot\frac{CM}{MA}=\left(-\frac{AC'}{BC}\right)\cdot\left(\frac{B'A}{AC'}\right)\cdot\left(-\frac{BC}{B'A}\right)=(-1)(1)(-1)=1$$.

\item Sea $\triangle ABC$ dirigido levógiramente y tal que $0<\frac{AB}{BN}$, $0<\frac{BL}{LC}$ y $0<\frac{MC}{CA}$. Entonces:
$$\frac{AN}{BN}=\frac{AC'}{BC}\;\;\;\;\;\;\;\;\;\;\;\;\frac{MC}{MA}=\frac{BC}{B'A}\;\;\;\;\;\;\;\;\;\;\;\;\frac{BL}{LC}=\frac{B'A}{AC'}$$

Por lo tanto, $\frac{AN}{BN}\cdot\frac{BL}{LC}\cdot\frac{MC}{MA}=\frac{AC'}{BC}\cdot\frac{B'A}{AC'}\cdot\frac{BC}{B'A}=1$, entonces

$$\frac{AN}{NB}\cdot\frac{BL}{LC}\cdot\frac{CM}{MA}=\left(-\frac{AC'}{BC}\right)\cdot\left(\frac{B'A}{AC'}\right)\cdot\left(-\frac{BC}{B'A}\right)=(-1)(1)(-1)=1$$.

\end{itemize}
\item [($\Leftarrow$)] Supongamos que $\frac{AN}{NB}\cdot\frac{BL}{LC}\cdot\frac{CM}{MA}=1$ y que  $\overline{AL}\cap\overline{BM}\cap\overline{CN}=\emptyset$. 

Sea $\overline{AL}\cap\overline{BM}=\{O\}$ y $\overline{CO}\cap\overline{AB}=\{N'\}$, entonces $\overline{AL}\cap\overline{BM}\cap\overline{CN'}=\{O\}$. Por la primera implicación de la demostración, sabemos que $\frac{AN'}{N'B}\cdot\frac{BL}{LC}\cdot\frac{CM}{MA}=1$ y por hipótesis $\frac{AN}{NB}\cdot\frac{BL}{LC}\cdot\frac{CM}{MA}=1$ por lo que $\frac{AN'}{N'B}\cdot\frac{BL}{LC}\cdot\frac{CM}{MA}=\frac{AN}{NB}\cdot\frac{BL}{LC}\cdot\frac{CM}{MA}$ así que $\frac{AN}{NB}=\frac{AN'}{N'B}$. Por tanto, $N=N'$ y $\overline{AL}\cap\overline{BM}\cap\overline{CN}=\{O\}$ lo cual es un contradicción pues supusimos que $\overline{AL}\cap\overline{BM}\cap\overline{CN}=\emptyset$. 

Por lo tanto, $\overline{AL}\cap\overline{BM}\cap\overline{CN}\neq\emptyset$. 
\end{enumerate}
\end{dem}




\chapter{Teorema de Menelao}
\begin{teo}[Teorema de Menelao]\label{Teo de Menelao}\index{Teorema ! de Menelao}
Sea $\triangle ABC$, $L\in\overline{BC}\backslash \{B,C\}$, $M\in\overline{CA}\backslash \{C,A\}$ y $N\in\overline{AB}\backslash \{A,B\}$, entonces $L$, $M$ y $N$ son colineales si y solamente si 
$$\frac{AN}{NB}\cdot\frac{BL}{LC}\cdot\frac{CM}{MA}=-1$$.
\end{teo}
\begin{dem} Sea $\triangle ABC$, $L\in\overline{BC}\backslash \{B,C\}$, $M\in\overline{CA}\backslash \{C,A\}$ y $N\in\overline{AB}\backslash \{A,B\}$.
\begin{enumerate}
\item[($\Rightarrow$)] Supongamos que $L$, $M$ y $N$ son colineales.

Sea $o$ tal que $\{L,M,N\}\subset o$, $a$ ortogonal a $o$ por $A$, $b$ ortogonal a $o$ por $B$ y $c$ ortogonal a $o$ por $C$. Y llamemos $a\cap o=\{P\}$, $b\cap o=\{Q\}$ y $c\cap o=\{R\}$.
Entonces:

\begin{itemize}
\item $\triangle QBN\cong\triangle PAN$ \textbf{cs(AA)} pues $|\angle PNA|=|\angle QNB|$ y $|\angle APN|=|\angle BQN|=\perp$, entonces $\frac{|AN|}{|BN|}=\frac{|AP|}{|BQ|}=\frac{|NP|}{|NQ|}$.
\item $\triangle QBL\cong\triangle RCL$ \textbf{cs(AA)} ya que $|\angle BLQ|=|\angle CLR|$ y $|\angle LRC|=|\angle LQB|=\perp$, entonces $\frac{|BL|}{|CL|}=\frac{|BQ|}{|CR|}=\frac{|QL|}{|RL|}$.
\item $\triangle CMR\cong\triangle AMP$ \textbf{cs(AA)} pues $|\angle CMR|=|\angle AMP|$ y $|\angle MRC|=|\angle MPA|=\perp$, entonces $\frac{|CM|}{|AM|}=\frac{|CR|}{|AP|}=\frac{|MR|}{|MP|}$.

Por tanto, 
$$\frac{|AN|}{|BN|}\cdot\frac{|BL|}{|CL|}\cdot\frac{|CM|}{|AM|}=\frac{|AP|}{|BQ|}\cdot\frac{|BQ|}{|CR|}\cdot\frac{|CR|}{|AP|}=1.$$

Así tenemos que $$\frac{|AN|}{|BN|}\cdot\frac{|BL|}{|CL|}\cdot\frac{|CM|}{|AM|}=1$$.

Notemos que en este teorema también los segmentos dirigidos juegan un papel importante. 
Veamos algunos casos:
\begin{itemize}
\item Consideremos $\triangle ABC$ dirigido levógiramente y sean $0<\frac{NA}{AB}$, $0<\frac{BC}{CL}$ y $0<\frac{CA}{AM}$. Entonces
$$\frac{NA}{NB}\cdot\frac{BL}{CL}\cdot\frac{CM}{AM}=1$$
Por tanto, 
$$\frac{AN}{NB}\cdot\frac{BL}{LC}\cdot\frac{CM}{MA}=-1$$.

\item Sea $\triangle ABC$ dirigido levógiramente, $0<\frac{AN}{NB}$, $0<\frac{LB}{BC}$ y $0<\frac{CM}{MA}$. Así
$$\frac{AN}{NB}\cdot\frac{LB}{LC}\cdot\frac{CM}{MA}=1$$
Entonces,
$$\frac{AN}{NB}\cdot\frac{BL}{LC}\cdot\frac{CM}{MA}=-1$$.
\end{itemize}

\item[($\Leftarrow$)] Supongamos que $\frac{AN}{NB}\cdot\frac{BL}{LC}\cdot\frac{CM}{MA}=-1$ y que $L$, $M$ y $N$ no son colineales.

Sea $\overline{LM}\cap\overline{AB}=\{N'\}$, entonces $L$, $M$ y $N'$ son colineales. Usando la primera implicación de esta prueba sabemos que 
$$\frac{AN'}{N'B}\cdot\frac{BL}{LC}\cdot\frac{CM}{MA}=-1$$.
Por lo tanto,
$$\frac{AN}{NB}\cdot\frac{BL}{LC}\cdot\frac{CM}{MA}=\frac{AN'}{N'B}\cdot\frac{BL}{LC}\cdot\frac{CM}{MA}$$.
Entonces, $$\frac{AN}{NB}=\frac{AN'}{N'B}$$
con esto tenemos que $N=N'$ y $L$, $M$ y $N$ son colineales, lo que contradice nuestra supusición. 

\end{itemize}
\end{enumerate}
\end{dem}




\chapter{Teorema de Desargues}

\begin{df}\label{PDUP} Sean $\triangle ABC$ y $\triangle DEF$ ($A\neq D$, $B\neq E$, $C\neq F$).
Decimos que $\triangle ABC$ y $\triangle DEF$ están en \textcolor{red}{\bf perspectiva desde un punto}\index{perspectiva ! desde un punto} $O$ si y solamente si $\overline{AD}\cap\overline{BE}\cap\overline{CF}=\{O\}$. Al punto $O$ lo llamamos \textcolor{red}{\bf centro de perspectiva}\index{centro ! de perspectiva}.
\end{df}
\begin{df}\label{PDUR} Sean $\triangle ABC$ y $\triangle DEF$ ($A\neq D$, $B\neq E$, $C\neq F$).
Decimos que $\triangle ABC$ y $\triangle DEF$ están en \textcolor{red}{\bf perspectiva desde una recta}\index{perspectiva ! desde una recta} $o$ si y solamente si $\overline{AB}\cap\overline{DE}=\{P\}$, $\overline{BC}\cap\overline{EF}=\{Q\}$, $\overline{CA}\cap\overline{FD}=\{R\}$ y $\{P,Q,R\}\subset o$.
\end{df}

\begin{teo}[Teorema de Desargues]\index{Teorema ! de Desargues}
$\triangle ABC$ y $\triangle DEF$ están en perspectiva desde un punto si y solamente si $\triangle ABC$ y $\triangle DEF$ están en perspectiva desde una recta. 
\end{teo}
\begin{dem}
\begin{enumerate}
\item[($\Rightarrow$)]Supongamos que $\triangle ABC$ y $\triangle DEF$ están en perspectiva desde un punto $O$, es decir $\overline{AD}\cap\overline{BE}\cap\overline{CF}=\{O\}$.

Sea $\overline{AB}\cap\overline{DE}=\{P\}$, $\overline{BC}\cap\overline{EF}=\{Q\}$ y $\overline{CA}\cap\overline{FD}=\{R\}$. Mostremos que $P$, $Q$ y $R$ son colineales.

Primero consideremos $\triangle OAB$ y la recta $\overline{DE}$, entonces por el Teorema~\ref{Teo de Menelao} tenemos que:
$$\frac{AP}{PB}\cdot\frac{BE}{EO}\cdot\frac{OD}{DA}=-1$$

Ahora, tomemos $\triangle OBC$ y la recta $\overline{EF}$, por el Teorema~\ref{Teo de Menelao}, entonces:
$$\frac{OE}{EB}\cdot\frac{BQ}{QC}\cdot\frac{CF}{FO}=-1$$

Y finalmente, consideremos $\triangle OCA$ y la recta $\overline{FD}$, entonces por el Teorema~\ref{Teo de Menelao} se tiene que:
$$\frac{OF}{FC}\cdot\frac{CR}{RA}\cdot\frac{AD}{DO}=-1$$

De esto tenemos lo siguiente,
$$\frac{AP}{PB}\cdot\frac{BE}{EO}\cdot\frac{OD}{DA}\cdot\frac{OE}{EB}\cdot\frac{BQ}{QC}\cdot\frac{CF}{FO}\cdot\frac{OF}{FC}\cdot\frac{CR}{RA}\cdot\frac{AD}{DO}=(-1)(-1)(-1)=-1$$
Entonces
$$\frac{AP}{PB}\cdot\frac{BQ}{QC}\cdot\frac{CR}{RA}=-1$$
Por lo tanto, aplicando nuevamente el Teorema~\ref{Teo de Menelao} concluimos que $P$, $Q$ y $R$ son colineales y así $\triangle ABC$ y $\triangle DEF$ están en perspectiva desde una recta.

\item[($\Leftarrow$)] Supongamos que $\triangle ABC$ y $\triangle DEF$ están en perspectiva desde una recta. Esto es, si $\overline{AB}\cap\overline{DE}=\{P\}$, $\overline{BC}\cap\overline{EF}=\{Q\}$ y $\overline{CA}\cap\overline{FD}=\{R\}$, entonces $P$, $Q$ y $R$ son colineales. Debemos demostrar que $\overline{AD}\cap\overline{BE}\cap\overline{CF}\neq\emptyset$. 

Notemos que $\triangle ADR$ y $\triangle BEQ$ están en perspectiva desde un punto ya que $\overline{AB}\cap\overline{DE}\cap\overline{RQ}=\{P\}$, entonces por la primera implicación de este teorema tenemos que $\triangle ADR$ y $\triangle BEQ$ están en perspectiva desde una recta, es decir, si $\overline{AD}\cap\overline{BE}=\{O\}$ y como sabemos $\overline{DR}\cap\overline{EQ}=\{F\}$ y $\overline{AR}\cap\overline{BQ}=\{C\}$, entonces $O$, $F$ y $C$ son colineales con lo que se tiene que $O\in\overline{CF}$ y así $\overline{AD}\cap\overline{BE}\cap\overline{CF}=\{O\}$ por lo que concluimos que $\triangle ABC$ y $\triangle DEF$ están en perspectiva desde $O$. 
\end{enumerate}
\end{dem}

\subsection*{Ejercicios}
\begin{enumerate}
\item 
\end{enumerate}





\chapter{Semejanza de polígonos}

\section{Polígonos semejantes}

\begin{df} Sean $A_{1},A_{2},...,A_{n}$ ($3\leq n$) puntos en el plano. El \textcolor{red}{\bf polígono}\index{polígono} $A_{1}A_{2}A_{3}...A_{n}$ es la figura que tiene como vértices a dichos puntos y como lados a los segmentos $A_{i}A_{i+1}$ ($i\in \{1,2,...,n\}$) y a $A_{n}A_{1}$.
\end{df}
\begin{df}
Sean $P_{0}$, $P_{1}$ dos polígonos con el mismo número de lados, si sus lados correspondientes son proporcionales y sus ángulos correspondientes son iguales de tal forma que la correspondencia es biunívoca, entonces decimos que $P_{0}$ y $P_{1}$ son \textcolor{red}{\bf polígonos semejantes}\index{polígonos ! semejantes}.
\end{df}

Entre los polígonos con los que hemos trabajado, se encuentran los triángulos y las condiciones para que dos triángulos sean semejantes pueden consultarse en la Definición~\ref{Semejanza df}, ahora veamos que se pueden definir dos tipos de semejanza considerando los segmentos dirigidos, las semejanzas son las siguientes:
\begin{itemize}
\item $\triangle ABC$ es \textcolor{red}{\bf directamente semejante}\index{directamente semejante} al $\triangle DEF$ si y solamente si $\triangle ABC\cong\triangle DEF$ y $\frac{AB}{DE}=\frac{BC}{EF}=\frac{CA}{FD}$.
\item $\triangle ABC$ es \textcolor{red}{\bf inversamente semejante}\index{inversamente semejante} al $\triangle DEF$ si y solamente si $\triangle ABC\cong\triangle DEF$ y $\frac{AB}{ED}=\frac{BC}{FE}=\frac{CA}{DF}$.
\end{itemize}

\section{Figuras homotéticas}
\begin{df}\label{PHdf}
Sean $P=A_{1}A_{2}A_{3}...A_{n}$ y $Q=B_{1}B_{2}B_{3}...B_{n}$ ($3\leq n$). 
Decimos que $P$ y $Q$ son \textcolor{red}{\bf polígonos homotéticos}\index{polígonos ! homotéticos} si y solamente si
\begin{itemize}
\item Las rectas determinadada por vértices correspondientes concurren, es decir, si $l_{i}=\overline{A_{i}B_{i}}$ para toda $i\in\{1,2,3,...,n\}$, entonces 
$$\bigcap_{i=1}^{n}l_{i}=\{O\}$$
donde $O$ es un punto en el plano llamado \textcolor{red}{\bf centro de homotecia}\index{centro ! de homotecia}.
\item Para toda $i\in\{1,2,3,...,n\}$ existe $k\in\mathbb{R}\backslash\{0\}$ tal que
$$\frac{0A_{i}}{OB_{i}}=k$$
A $k$ se le conoce como \textcolor{red}{\bf razón de homotecia}\index{razón de homotecia}.
\end{itemize}
\end{df}

Notemos que para toda $i\in\{1,2,3,...,n\}$, $A_{i}A_{i+1}$ es paralela a $B_{i}B_{i+1}$
esto debido a que $\frac{OA_{i}}{OB_{i}}=\frac{OA_{i+1}}{OB_{i+1}}$, entonces por el Teorema~\ref{Thales1} $A_{i}A_{i+1}$ es paralela a $B_{i}B_{i+1}$.

\section{Simetría con respecto a un punto}

Ahora haremos una observación importante, consideremos $P$ y $Q$ dos polígonos homotéticos con centro de homotecia $O$, como el lector podrá notar, la razón de homotecia $k$ es un número en $\mathbb{R}\backslash \{0\}$, es decir, $k$ puede positiva o negativa. 
Una caso particular es aquel en el que $k=-1$, cuando esto sucede decimos que $P$ y $Q$ son dos \textcolor{red}{\bf figuras simétricas}\index{figuras simétricas} con respecto al punto $O$ que funge el papel de \textcolor{red}{\bf centro de simetría}\index{centro ! de simetría}.

\section{Líneas antiparalelas}
\begin{df}
Sean $a,b,c,d$ cuatro rectas en el plano. Decimos que $a$ y $b$ son \textcolor{red}{\bf antiparalelas}\index{antiparalelas} de $c$ y $d$ si y solamente si la bisectriz  $l$ del $|\angle(a\longrightarrow b)|$ interseca a $c$ y $d$ tal que $|\angle(l\longrightarrow c)|=|\angle(d\longrightarrow l)|$.
\end{df}
Observemos que si $l\cap c=\{P\}$, $l\cap d=\{Q\}$ y $c\cap d=\{R\}$, entonces al considerar $\triangle PQR$ tenemos que $|\angle RPQ|=|\angle PQR|$, es decir $\triangle PQR$ es isósceles, por tanto $|RQ|=|PR|$. 

\begin{prop}
Si $a$ y $b$ son antiparalelas a $c$ y $d$, entonces $c$ y $d$ son antiparalelas a $a$ y $b$. Es decir, la relación de antiparalalelismo es simétrica.
\end{prop}
\begin{pba}
Construir la bisectriz del $|\angle(c\longrightarrow d)|$ que llamaremos $m$ y sea $l$ la biectriz de $|\angle(a\longrightarrow b)|$.
Debemos probar que $|\angle(m\longrightarrow a)|=|\angle(b\longrightarrow m)|$.

Sea $c\cap l=\{P\}$, $d\cap l=\{Q\}$, $c\cap d=\{R\}$, $m\cap b=\{Y\}$, $m\cap a=\{X\}$, $b\cap a=\{Z\}$ y $l\cap m=\{J\}$. Ahora, consideremos $\triangle PJR$ y $\triangle QJR$ como $|\angle JRP|=|\angle JRQ|$ y $|\angle JPR|=|\angle JQR|$ con lo que concluimos que $\triangle PJR\cong\triangle QJR$ \textbf{cs(AA)} por tanto $\frac{|PJ|}{|QJ|}=\frac{|PR|}{|QR|}=\frac{|JR|}{|JR|}=1$ por   ello $\triangle PJR\equiv\triangle QJR$, entonces $|\angle PJR|=|\angle QJR|$ y como $|\angle PJR|+|\angle RJQ|=2\perp$ así $|\angle PJR|=\perp$.

Además $\triangle ZYJ\equiv\triangle ZXJ$ \textbf{cs(ALA)} ya que por lo anterior $|\angle YJZ|=|\angle XJZ|$, $|ZJ|=|ZJ|$ y $|\angle YZJ|=|\angle XZJ|$ pues $\overline{ZJ}=l$ es bisectriz de $|\angle XZY|$ por lo tanto $|\angle ZYJ|=|\angle ZXJ|$ y así $|\angle(m\longrightarrow a)|=|\angle(b\longrightarrow m)|$.
\end{pba}

\begin{prop}
Si $a$ y $b$ son antiparalelas respecto a $c$ y $d$, entonces sus bisectrices son ortogonales. 
\end{prop}

\begin{pba}
Sean $l$ la bisectriz de $|\angle(a\longrightarrow b)|$, $m$ la bisectriz de $|\angle(c\longrightarrow d)|$, $a\cap b=\{O\}$, $c\cap d=\{P\}$, $c\cap l=\{Q\}$ y $d\cap l=\{R\}$.
Consideremos $\triangle PQR$ como $|\angle RQP|=|\angle PRQ|$ por lo tanto como $m$ es bisectriz de $|\angle(c\longrightarrow d)|$, es bisectriz de $|\angle QPR|$ y por tanto altura del $\triangle PQR$ (véase ejercicio $\clubsuit$), así que $m$ es ortogonal a $\overline{QR}=l$.
\end{pba}

\section{Cuadriláteros cíclicos}
\begin{df}
Sea $\{A_{1},A_{2},A_{3},...,A_{n}\}$ un conjunto puntos en el plano. Decimos que $A_{1},A_{2},A_{3},...,A_{n}$ son \textcolor{red}{\bf concíclicos}\index{concíclicos} si y solamente si existe $\mathcal{C}(O,r)$ tal que $\{A_{1},A_{2},A_{3},...,A_{n}\}\subset\mathcal{C}(O,r)$.
\end{df}

\begin{df}
Sean $A_{1},A_{2},A_{3},...,A_{n}$ puntos en el plano. Un polígono $A_{1}A_{2}A_{3}...A_{n}$
es un \textcolor{red}{\bf polígono convexo}\index{polígono ! convexo} si y solamente si para cualesquiera dos puntos en el interior del polígono el segmento que los une está contenido en el interior del polígono. 
\end{df}
\begin{prop}
$\square ABCD$ es un polígono convexo y $A,B,C,D$ son concíclicos si y solamente si $|\angle ABC|+|\angle CDA|=2\perp$ y $|\angle BCD|+|\angle DAB|=2\perp$.
\end{prop}
\begin{pba}
\begin{enumerate}
\item [($\Rightarrow$)] Supongamos que $\square ABCD$ es un polígono convexo y $A,B,C,D$ son concíclicos, entonces existe $\mathcal{C}$ una circunferencia tal que $\{A,B,C,D\}\subset\mathcal{C}$. 
Fijemos a $A$ y $C$, así por el Proposición~\ref{P1AIC} $|\angle ABC|=\frac{|\angle AOC|}{2}$ y $|\angle CDA|=\frac{|COA|}{2}$ por tanto $|\angle ABC|+|\angle CDA|=\frac{|\angle AOC|+|\angle COA|}{2}=\frac{4\perp}{2}=2\perp$. 
Análogamente se prueba que $|\angle BCD|+|\angle DAB|=2\perp$.
\item [($\Leftarrow$)] Ahora supongamos que $|\angle ABC|+|\angle CDA|=2\perp$ y $|\angle BCD|+|\angle DAB|=2\perp$. Consideremos $|\angle ABC|$ y fijemos $A$ y $C$, ahora consideremos $X=\{P\in\mathbb{E}^{2}|\angle APC=\angle ABC\}$ y $Y=\{Q\in\mathbb{E}^{2}|\angle AQC=2\perp\backslash\angle ABC\}$, entonces por la Proposición~\ref{PLGA} $\{A,B,C,D\}\subset\mathcal{C}$.
\end{enumerate}
\end{pba}
\begin{prop}
Sea $\mathcal{C}(O,r)$ y $\{A,B,C,D\}\subset\mathcal{C}(O,r)$ ordenados dextrógiramente, entonces $\overline{AB}$ y $\overline{CD}$ son antiparalelas respecto a $\overline{BC}$ y $\overline{DA}$.
\end{prop}
\begin{pba}
Sea $\overline{AB}\cap\overline{CD}=\{P\}$ y sea $l$ la bisectriz interna del $\angle APD$. 
Si $l\cap\overline{BC}=\{Q\}$, $l\cap\overline{AD}=\{R\}$ y $\overline{AD}\cap\overline{BC}=\{S\}$, debemos probar que $\angle SQR=\angle RQS$.

Por hipótesis $\{A,B,C,D\}\subset\mathcal{C}(O,r)$, entonces $|\angle ABC|+|\angle CDA|=2\perp$. También $|\angle ABC|+|\angle CBP|=2\perp$, entonces $|\angle CDA|=|\angle CBP|$. Por tanto, $\triangle PBQ\cong\triangle PDR$ \textbf{cs(AA)}. Así que $\frac{|PB|}{|PD|}=\frac{|BQ|}{|DR|}=\frac{|PQ|}{|PR|}$, $|\angle PQB|=|\angle PRD|$ y como $|\angle PQB|=|\angle CQR|$  (por ser opuestos por el vértice), entonces $|\angle PRD|=|\angle CQR|$, así $\angle DRP=\angle RQC$. 

Como $S$, $R$ y $D$ son colineales dado que $\{S,R,D\}\subset\overline{AD}$, $\{R,P,Q\}\subset l$, entonces $\angle DRP=\angle SRQ$. Y como $\{Q,C,S\}\subset\overline{BC}$, entonces $\angle RQC=\angle RQS$. Por tanto, $\angle RQS=\angle SRQ$.
Por lo que $\overline{AB}$ y $\overline{CD}$ son antiparalelas respecto a $\overline{BC}$ y $\overline{DA}$.
\end{pba}

\section{Teorema de Ptolomeo}

Primero generalizado y después corolario

\begin{teo}
Sea $\square ABCD$ un cuadrilátero convexo ordenado (levógiramente o dextrógiramente), entonces $|AC||BD|\leq |AB||CD|+|BC||DA|.$
\end{teo}
\begin{dem}
Consideremos a $\alpha=\angle CAB$ y $\beta=\angle ACB$ ángulos internos de $\triangle ABC$.
Sean $l$ una recta en el plano tal que $A\in l$ y $\angle (\overline{DA}\longrightarrow l)=\alpha$, $m$ una recta en el plano tal que $D\in m$ y $\angle(m\longrightarrow\overline{DA})=\beta$ y $l\cap m=\{E\}$.

Así tenemos que $\triangle ADE\cong\triangle ACB$ \textbf{cs(AA)} pues $|\angle CAB|=|\angle DAE|$ y $|\angle BCA|=|\angle EDA|$, entonces $\frac{|AD|}{|AC|}=\frac{|DE|}{|CB|}=\frac{|AE|}{|AB|}$.

También $\triangle ADC\cong\triangle AEB$ \textbf{cs(LAL)} ya que $\angle DAC=\angle DAE+\angle EAC=\angle EAC+\angle CAB=\angle EAB$ y como $\frac{|AD|}{|AC|}=\frac{|AE|}{|AB|}$, entonces $\frac{|AD|}{|AE|}=\frac{|AC|}{|AB|}$ por lo que $\frac{|AD|}{|AE|}=\frac{|DC|}{|EB|}=\frac{|AC|}{|AB|}$.

Como $\frac{|AD|}{|AC|}=\frac{|DE|}{|CB|}$, entonces
$$|DE|=\frac{|AD||CB|}{|AC|}$$
Además por ser $\frac{|DC|}{|EB|}=\frac{|AC|}{|AB|}$, entonces
$$|EB|=\frac{|AB||DC|}{|AC|}$$

Por tanto como $|DB|\leq |DE|+|EB|$, tenemos que 
$$|DB|\leq \frac{|AD||CB|}{|AC|}+\frac{|AB||DC|}{|AC|}$$
Y así,
$$|AC||DB|\leq |AD||CB|+|AB||DC|$$.
\end{dem}

\begin{teo}[Teorema de Ptolomeo]\index{Teorema ! de Ptolomeo}
Un cuadrilátero convexo es cíclico si y solamente si el producto de sus diagonales es igual a la suma del producto de sus lados opuestos. 
\end{teo}
\begin{dem}
Sea $\square ABCD$ convexo ordenado (levógiramente o dextrógiramente). Y consideremos a $\alpha=\angle CAB$ y $\beta=\angle ACB$ ángulos internos de $\triangle ABC$.
Sean $l$ una recta en el plano tal que $A\in l$ y $\angle (\overline{DA}\longrightarrow l)=\alpha$, $m$ una recta en el plano tal que $D\in m$ y $\angle(m\longrightarrow\overline{DA})=\beta$ y $l\cap m=\{E\}$.

Tenemos que $|AC||DB|=|AB||CD|+|BC||DA|$.
$\leftrightarrow$ $\{O,E,D\}$ son colineales.
$\leftrightarrow$ $\angle EDB=0$.
$\leftrightarrow$ $\angle ADB=\angle ACB=\beta$.
$\leftrightarrow$ $\{D,C\}\subset X=\{P\in\mathbb{E}^{2}|\angle APB=\beta\}$
($X$ son dos arcos de circunferencia del mismo radio puesto que $\square ABCD$ es convexo y ordenado, se tiene que $D$ y $C$ pertenecen al mismo arco).
$\leftrightarrow$ $\square ABCD$ es cíclico. 
\end{dem}


\section{Circunferencias homotéticas}
En la Definición~\ref{PHdf} hablamos sobre el concepto de polígonos homotéticos, ahora veremos un caso particular de estos polígonos, las circunferencia homotética. 

Dadas dos circunferencias $\mathcal{C}(O,r)$ y $\mathcal{C}(O',r')$ ($O\neq O'$ y $r\neq r'$) es posible demostrar que son figuras homotéticas. 

En efecto, sean $A\in\mathcal{C}(O,r)\backslash l$, $a$ la recta determinada por $\{A,O\}$
y $a'$ la recta paralela a $a$ por $O'$.
Ahora, consideremos $\mathcal{C}(O',r')\cap a'=\{A',A''\}$. Sean $m$ la recta determinada por $A$ y $A'$, $n$ la recta determinada por $A$ y $A''$, $\{H\}=l\cap m$ y $\{K\}=l\cap n$.

Afirmación: $H$ y $K$ son centros de homotecia.

Notemos que $\triangle HAO\cong\triangle HA'O'$ \textbf{cs(AA)} pues $|\angle AHO|=|\angle A'HO'|$ y $|\angle OAH|=|\angle O'A'H|$, entonces $\frac{|HA|}{|HA'|}=\frac{|KO|}{|KO'|}=\frac{|OA|}{|O'A'|}=\frac{|r|}{|r'|}$.

También $\triangle OAK\cong\triangle O'A''K$ \textbf{cs(AA)} pues $|\angle AKO|=|\angle A''KO'|$ y $|\angle OAH|=|\angle O'A''K|$, entonces $\frac{|AK|}{|A''K|}=\frac{|OK|}{|O'K|}=\frac{|OA|}{|O'A''|}=\frac{|r|}{|r'|}$.

\begin{teo}
Si $\mathcal{C}(O,r)$ y $\mathcal{C}(O',r')$ tienen tangentes comunes, entonces las tangentes contienen a alguno de los centros de homotecia. 
\end{teo}
\begin{dem}
Sea $l$ tangente común a $\mathcal{C}(O,r)$ y $\mathcal{C}(O',r')$ en $T$ y $T'$ respectivamente, $m$ la recta determinada por $\{O,O'\}$, $l\cap m=\{K'\}$.

Observemos que $OT$ es paralela a $O'T'$. Entonces $\triangle OTK'\cong\triangle O'T'K'$ \textbf{cs(AA)} por tanto $\frac{|OT|}{|O'T'|}=\frac{|OK'|}{|O'K'|}=\frac{|TK'|}{|T'K'|}$

Por ello, $K'\in\{H,K\}$, es decir, $K'$ es alguno de los centros de homotecia y se encuentra en las tangentes comunes de $\mathcal{C}(O,r)$ y $\mathcal{C}(O',r')$.
\end{dem}

Sea $\mathcal{C}(O,r)$ y $P$ un punto cualquiera en el plano. 
Sean $A\in\mathcal{C}$, $\{B\}=\mathcal{C}\cap\overline{PA}\backslash\{A\}$, $C\in\mathcal{C}$, $\{D\}=\mathcal{C}\cap\overline{PC}\backslash\{C\}$, asì tenemos que $\{A,B,C,D\}\subset\mathcal{C}$. 

Notemos que dadas estàs condiciones tenemos los siguientes tres casos:

\begin{itemize}
\item $r<|PO|$

En este caso tenemos que $\triangle PAD\cong\triangle PCB$ \textbf{cs(AA)} pues $|\angle DPA|=|\angle BPC|$ y como $|\angle PAD|+|\angle DCB|=2\perp$ y $|\angle DCB|+|\angle BCP|=2\perp$, entonces $|\angle BCP|=|\angle PAD$. Por tanto, $\frac{|PA|}{|PC|}=\frac{|AD|}{|CB|}=\frac{|PD|}{|PB|}$, consideremos $0<AB$ y $0<CD$, entonces $\frac{AP}{PC}=\frac{PD}{BP}$. Así $AP\cdot BP=PD\cdot PC$, entonces $PA\cdot PB=PC\cdot PD$. 

Ahora consideremos $E\in\mathcal{C}\backslash\{A,B,C,D\}$ y $\{F\}=\mathcal{C}\cap\overline{PE}\backslash\{E\}$. 
Entonces tenemos que $\triangle PAF\cong\triangle PEB$ \textbf{cs(AA)} pues $|\angle FPA|=|\angle BPE|$ y como $|\angle PAF|+|\angle FEB|=2\perp$ y $|\angle FEB|+|\angle BEP|=2\perp$. Por tanto, $|\angle PAF|=|\angle BEP|$. Así tenemos que $\frac{|PA|}{|PE|}=\frac{|PF|}{|PB|}\frac{|AF|}{|EB|}$, considerando $0<AB$ y $0<EF$, tenemos que $\frac{AP}{EP}=\frac{FP}{BP}$, entonces $AP\cdot BP=FP\cdot EP$.
Así $PA\cdot PB=PE\cdot PF=PC\cdot PD$. 
Esto es para cualquier $X\in\mathcal{C}$ tal que $\{Y\}=\mathcal{C}\cap\overline{PX}\backslash\{X\}$ se tiene que $PA\cdot PB=PY\cdot PX$, también notemos que si $X=Y$, entonces $PA\cdot PB=PX\cdot PX$ y en esta situación $\overline{PX}$ es tangente a $\mathcal{C}$ en $X$.
\item $|PO|<r$. \label{POC2} La prueba es análoga (ver sección de ejercicios, Ejercicio~\ref{PO<r}).
\item $|PO|=r$. 
Para que $|PO|=r$, debería pasar que $P\in\mathcal{C}$, entonces $P=B=D$, entonces $PA\cdot PB=PA\cdot PP=0$. 
\end{itemize}
\begin{df}
Sean $P$ un punto en el plano, $\mathcal{C}(O,r)$ una circunferencia y $l$ una recta tal que $P\in l$ y $\mathcal{C}\cap l=\{A,B\}$, entonces a la constante $PA\cdot PB$ se le conoce como la \textcolor{red}{\bf potencia del punto $P$ respecto a $\mathcal{C}$}\index{potencia del punto $P$ respecto a $\mathcal{C}$} que denotaremos como $Pot_{\mathcal{C}(O,r)}(P)$.
\end{df}
\section{Puntos homólogos y antihomólogos}
\begin{df}
Sean $\mathcal{C}(O,r)$ y $\mathcal{C}(O',r')$ ($O\neq O'$). Sea $A\in\mathcal{C}(O,r)$ y $\overline{HA}\cap\mathcal{C'}(O',r')=\{A',B'\}$, notemos que solamente $A'$ tiene la propiedad $\frac{|HA|}{|HA'|}=\frac{r}{r'}$ ($H$ es uno de los centros de homotecia de $\mathcal{C}, \mathcal{C'}$). Decimos que $A$ y $A'$ son \textcolor{red}{\bf puntos homólogos}\index{puntos ! homólogos} respecto a $H$ y $A$, $B'$ con \textcolor{red}{\bf puntos antihomólogos} respecto a $H$.
\end{df}

Ahora veamos algunas propiedades de los puntos homólogos y antihomólogos. Para esto consideremos dos circunferencias $\mathcal{C}(O,r)$ y $\mathcal{C}(O',r')$.
\begin{itemize}
\item Si $A,B\in\mathcal{C}(O,r)$, entonces la recta $\overline{AB}$ es paralela a la recta determinada por sus homólogos. 

\begin{pba}
Sean $A$ y $A'$ homólogos respecto a $X\in\{H,K\}$, entonces $\frac{|XA|}{|XA'|}=\frac{r}{r'}=\frac{|XB|}{|XB'|}$ donde $B$ es homólogo a $B'$ respecto a $X$. Así tenemos que $\frac{|XA|}{|XA'|}=\frac{|XB|}{|XB'|}$, entonces por el Teorema~\ref{Thales1} $\overline{AB}$ es paralela a $\overline{A'B'}$.
\end{pba}
\item Sean $l$ una recta que contenga a $X\in\{H,K\}$ tal que $|l\cap\mathcal{C}(O,r)|=2$ y $m\neq l$ tal que $X\in m$ y $|m\cap\mathcal{C}(O,r)|=2$.

Sean $l\cap\mathcal{C}(O,r)=\{A,B\}$, $\{A',B'\}\subset\mathcal{C'}(O,r)$, $m\cap\mathcal{C}(O,r)$ y $\{C',D'\}\subset\mathcal{C'}(O',r')$ tal que $A$ es homólogo a $A'$, $B$ es homólogo a $B'$, $C$ es homólogo a $C'$ y $D$ es homólogo $D'$. Entonces $\{A,D,B',C'\}$ y $\{A',D',B,C\}$ son concíclicos ($l$ y $m$ son antiparalelas respecto a $\overline{AD}$ y $\overline{B'C'}$ y respecto a $\overline{A'D'}$ y $\overline{BC}$).

\begin{pba}
Sea $\{A,B,C,D\}$ ordenados (levògiramente o dextrògiramente), entonces $\square ABCD$ es un cuadrilátero cíclico convexo, así $\angle ABC+\angle CDA=2\perp$ pero $\angle ABC=\angle AB'C'$ pues $BC$ es paralelo a $B'C'$, entonces $\angle AB'C'+\angle CDA=2\perp$. Por lo tanto, $\{A,D,B',C'\}$.

De manera análoga se puede probar que $\{A',B',C,D\}$ es cíclico. 
\end{pba}
\item Sea $A\in\mathcal{C}(O,r)$, si respecto a $X\in\{H,K\}$ ($H,K$ centro de Homotecia de $\mathcal{C}(O,r))$ y $\mathcal{C'}(O',r')$) $A$ y $B'$ son antihomólogos, entonces $XA\cdot XB'$ es constante.

\begin{pba}
$XA\cdot XB'$ es constante pues el producto es la potencia de $X$ respecto a la circunferencia que inscribe a $A,B'$ y cualquier otro par de puntos antihomólogos en $\mathcal{C}(O,r)$ (no colineales con $A$ y $X$).
\end{pba}
\item Si $A$ y $B'$ son puntos antihomólogos, $A\in\mathcal{C}(O,r)$ y $l$ es la tangente a $\mathcal{C}(O,r)$ en $A$, 
$B'\in\mathcal{C'}(O',r')$ y $l'$ es la tangente a $\mathcal{C'}(O',r')$ en $B'$, y  $l\cap l'=\{P\}$, entonces $\triangle APB'$ es isósceles. 

\begin{pba}
Supongamos que $A$ y $B'$ son antihomólogos. Sean K uno de los centro de homotecia de $\mathcal{C}$ y $\mathcal{C'}$, $\overline{KA}\cap\mathcal{C}(O,r)\backslash\{A\}=\{B\}$ y $\overline{KB'}\cap\mathcal{C'}(O',r')\backslash\{B'\}=\{A'\}$.

Observemos que $\frac{|KO|}{|KO'|}=\frac{|KB|}{|KB'|}=\frac{|OB|}{|OB'|}=\frac{r}{r'}$, entonces por el Teorema~\ref{Thales1} $OB\parallel O'B'$. Vamos a probar que  $\triangle APB'$ es isósceles. 

Tenemos que $\triangle AOB$ es isósceles pues $|OA|=|OB|=r$, entonces $|\angle OAB|=|\angle ABO|$. Además $\triangle A'O'B'$ es isósceles pues $|O'B'|=|O'A'|=r'$, asì $|\angle O'A'B'|=|\angle A'B'O'|$.

Ahora, notemos que por ser $l$ y $l'$ tangentes a $\mathcal{C}$ y $\mathcal{C'}$ respectivamente, tenemos que $|\angle O'B'P|=\perp=|\angle OAP|$. Y como $|\angle O'B'P|=|\angle O'B'A'|+|\angle A'B'P|$ y $|\angle OAP|=|\angle OAB|+|\angle BAP|$, por tanto $|\angle O'B'A'|+|\angle A'B'P|=|\angle OAB|+|\angle BAP|$. También por ser $OB\parallel O'B'$, se tiene que $|\angle O'B'A'|= |\angle OBA|=|\angle OAB|$, por lo que $|\angle OAB|+|\angle A'B'P|=|\angle OAB|+|\angle BAP|$, por lo tanto $|\angle A'B'P|=|\angle BAP|$ y con esto se concluye que $\triangle APB'$ es isósceles. 
\end{pba}
\end{itemize}

\section{Circunferencia de similitud}
\begin{df}
Sean $\mathcal{C}(O,r)$ y $\mathcal{C'}(O',r')$ dos circunferencias ($O\neq O'$) y $H,K$ los centros de homotecia. La \textcolor{red}{\bf circunferencia de similitud}\index{circunferencia ! de similitud} de $\mathcal{C}$ y $\mathcal{C'}$ es la circunferencia que tiene como diámetro al segmento $HK$.
\end{df}

\begin{teo}\label{CDS}
La circunferencia de similitud de $\mathcal{C}(O,r)$ y $\mathcal{C'}(O',r')$ ($O\neq O'$) es el lugar geométrico de los puntos $P$ en el plano tales que $\frac{PO}{O'P}=\frac{r}{r'}$ y que además cumple que el ángulo formado por las tangentes de $P$ a $\mathcal{C}$ es igual al formado por las tangentes de $P$ a $\mathcal{C'}$.
\end{teo}

\begin{dem}
Sean $H$, $K$ los centros de homotecia de $\mathcal{C}(O,r)$ y $\mathcal{C'}(O',r')$.
Sea $P$ en el plano tal que $\frac{PO}{O'P}=\frac{r}{r'}$. Como $H$ y $K$ tienen la propiedad $\frac{PO}{O'P}=\frac{HO}{HO'}$ y $\frac{PO}{O'P}=\frac{OK}{KO'}$.
Del $\triangle POO'$ tenemos que como $K\in\overline{OO'}$, entonces 
$$\frac{OK}{KO'}=\frac{PO \sen(\angle OPK)}{O'P \sen(\angle KPO')}=\frac{PO}{O'P}$$.

Por tanto, $\sen(\angle OPK)=\sen(\angle KPO')$ y así $\angle OPK=\angle KPO'$. Con lo que concluimos que $\overline{PK}$ es bisectriz interna del $\angle OPO'$.

Del $\triangle POO'$, como $H\in\overline{OO'}$, entonces $\frac{PO}{O'P}=\frac{HO}{HO'}=\frac{-OH}{HO'}$. Como $\frac{OH}{HO'}=\frac{PO \sen(\angle OPH)}{O'P \sen(\angle HPO')}$. Entonces
$$\frac{-OH}{HO'}=\frac{-PO \sen(\angle OPH)}{O'P \sen(\angle HPO')}=\frac{PO}{O'P}$$

De esto $-\sen(\angle OPH)=\sen(\angle HPO')$, entonces $\sen(\angle OPH)=-\sen(\angle HPO')=sen(-\angle HPO')=\sen(\angle O'PH)$.

Por tanto, $\overline{PH}$ es bisectriz del $\angle O'PO$. Y así tenemos que $\overline{PH}$ es bisectriz externa del $\angle OPO'$

Entonces $\overline{PK}$ es ortogonal a $\overline{PH}$, por lo tanto $P\in\mathcal{C^{*}}(O^{*},r^{*})$, donde $\mathcal{C^{*}}$ es la circunferencia de diámetro $KH$.

Ahora supongamos que $P$ está en la circunferencia de  similitud. Sea $Q\in\overline{OO'}$ tal que $\angle OPK=\angle KPO'$. Entonces como $\overline{PK}$ es ortogonal a $\overline{PH}$, $\overline{PK}$ es bisectriz interna del $\angle OPO'$ y $\overline{PH}$ es bisectriz externa del $\angle OPO'$, tenemos que $\frac{QH}{HO'}=-\frac{QK}{KO'}$ por otra parte también tenemos que $\frac{OK}{KO'}=\frac{OH}{HO'}$. Por tanto, $\frac{QK}{QH}=\frac{OK}{OH}$ así que $Q=O$.

Considerando $\triangle OPO'$ como $\overline{PK}$ es tal que $\angle OPK=\angle KPO'$, entonces al tener que $$\frac{OK}{KO'}=\frac{PO \sen(\angle OPK)}{O'P \sen(\angle KPO')}=\frac{PO}{O'P}$$. Concluimos que $\frac{PO}{O'P}=\frac{OK}{KO'}=\frac{r}{r'}$.

Por lo tanto $P$ cumple la propiedad. 
\end{dem}

\section{Circunferencia de Apolonio}
\begin{teo}
E lugar geométrico de los puntos $P$ en el plano, tales que su distancia a dos puntos fijos $A$, $B$ tiene una razón constante es una circunferencia. A dicha circunferencia se le llama \textcolor{red}{\bf circunferencia de Apolonio}\index{circunferencia ! de Apolonio}.
\end{teo}

\begin{dem}
Sean dos puntos fijo $A,B$ tal que  la razón de sus distancias a $P\in\mathbb{E}^{2}$ sea $k\in\mathbb{R}^{+}$. Construyamos $\mathcal{C}(A,r_{1})$ y $\mathcal{C}(B,r_{2})$, tal que $\frac{r_{1}}{r_{2}}=k$. Así por el  Teorema~\ref{CDS} tenemos que el lugar geomètrico de los puntos $P$  que cumplen que $\frac{|PA|}{|PB|}=k$ es una circunferencia, la circunferencia de similitud de $\mathcal{C}(A,r_{1})$ y $\mathcal{C}(B,r_{2})$. 
\end{dem}

\subsection*{Ejercicios}
\begin{enumerate}

\item En la pàgina~\pageref{POC2} probar el  caso en el que $|PO|<r$.\label{PO<r}
\end{enumerate}





\chapter{Razón cruzada}

\section{Definición}
\begin{df}\label{DRC}
Sea $l$ una recta en el plano y $\{A,B,C,D\}\subset l$, definimos la \textcolor{red}{\bf razòn cruzada de $A,B,C,D$}\index{razón cruzada de $A,B,C,D$} como $$\frac{AC}{CB}\Big/\frac{AD}{DB}$$ Y la denotamos como $l\{A,B;C,D\}$.
\end{df}
\begin{df}
Sean $a,b,c,d$ rectas en el plano tales que $a\cap b\cap c\cap d=\{L\}$ ($L\in\mathbb{E}^{2}$) y sean $A\in a, B\in b, C\in c, D\in d$ ($O\notin\{A,B,C,D\}$). Definimos la \textcolor{red}{\bf razón cruzada de $a,b,c,d$}\index{razón cruzada de $a,b,c,d$} como $$\frac{\sen\angle AOC}{\sen\angle COB}\Big/\frac{\sen\angle AOD}{\sen\angle DOB}$$ Y la denotamos por $L\{a,b,c,d\}$.
\end{df}
\section{Relaciones de razón cruzada de hileras y haces}
\begin{teo}\label{T1RRC}
Sean $l$ una recta en el plano y $O$ un punto cualquiera tal que $O\notin l$ y $\{A,B,C,D\}\subset l$. Entonces $l\{A,B,C,D\}=O\{\overline{OA},\overline{OB};\overline{OC},\overline{OD}\}$.
\end{teo}
\begin{dem}
Primero consideremos $\triangle OAB$ y que $C\in\overline{AB}$. Entonces por el Teorema~\ref{TGB} tenemos que:
$$\frac{AC}{CB}=\frac{OA\;\sen(\angle AOC)}{BO\;\sen(\angle COB)}$$

Ahora consideremos $\triangle OAB$ y que $D\in\overline{AB}$, entonces por el Teorema~\ref{TGB},
$$\frac{AD}{DB}=\frac{OA\;\sen(\angle AOD)}{BO\;\sen(\angle DOB)}$$
De esto se concluye que 
$$\frac{AC}{CB}\Big/\frac{AD}{DB}=\frac{OA\;\sen(\angle AOC)}{BO\;\sen(\angle COB)}\Big/\frac{OA\;\sen(\angle AOD)}{BO\;\sen(\angle DOB)}=\frac{\sen(\angle AOC)}{\sen(\angle COB)}\Big/\frac{\sen(\angle AOD)}{\sen(\angle DOB)}$$.

Por lo tanto, $l\{A,B,C,D\}=O\{\overline{OA},\overline{OB};\overline{OC},\overline{OD}\}$.

\end{dem}

\begin{cor}\label{C1RC}
Sea $\{A,B,C,D\}\subset l$ y sean $P,Q$ dos puntos tales que $\{P,Q\}\nsubseteq l$, entonces $$P\{\overline{PA},\overline{PB};\overline{PC},\overline{PD}\}=Q\{\overline{QA},\overline{QB};\overline{QC},\overline{QD}\}$$.
\end{cor}
\begin{pba}
Ver secciòn de ejercicios, Ejercicio~\ref{E1RC}.
\end{pba}

\begin{cor}\label{C2RC}
Sean $a,b,c,d$ rectas concurrentes $a\cap b\cap c\cap d=\{L\}$ y sean $p,q$ rectas tales que $L\notin p$ y $L\notin q$. Consideremos $\{A\}=p\cap a, \{B\}=p\cap b, \{C\}=p\cap c, \{D\}=p\cap d, \{E\}=q\cap a, \{F\}=q\cap b, \{G\}=q\cap c, \{H\}=q\cap d$.  Entonces
$$p\{A,B;,C,D\}=q\{E,F;G,H\}$$.
\end{cor}
\begin{pba}
Ver sección de ejercicios, Ejercicio~\ref{E2RC}.
\end{pba}

\section{Los seis valores de la razón cruzada}
Dados cuatro puntos en una recta, el càlculo de la cantidad de razones cruzadas que podemos encontrar está asociado al número de permutaciones de los cuatro puntos, lo cual nos diría que son veinticuatro razones cruzadas. Sin embargo, vamos a argumentar que en realidad son sólo seis ya que se repiten formando seis grupos de cuatro  en los que las razones cruzadas en cada grupo son las mismas.

Sean $l$ una recta en el plano, $\{A,B,C,D\}\subset l$ y $\lambda\in\mathbb{R}$.
Supongamos que $l\{A,B;C,D\}=\lambda$, aplicando la Definición~\ref{DRC} tenemos que $l\{B,A;D,C\}=l\{C,D;A,B\}=l\{D,C;B,A\}=\lambda$. 

Del mismo modo, haciendo uso de la Definición~\ref{DRC} se tiene que $l\{A,B;D,C\}=l\{B,A;C,D\}=l\{D,C;A,B\}=l\{C,D;B,A\}=\frac{1}{\lambda}$.

Ahora, como tenemos cuatro puntos colineales, haremos uso del Teorema de Euler el cual nos daba una identidad que nos será de ayuda. 
$$AB\cdot CD+AC\cdot DB+AD\cdot BC=0$$

$l\{A,C;B,D\}=\frac{AB}{BC}\Big/\frac{AD}{DC}=\left(\frac{AB}{BC}\right)\left(\frac{DC}{AD}\right)=-\frac{AB\cdot CD}{AD\cdot BC}=\frac{AC\cdot DB+AD\cdot BC}{AD\cdot BC}=\frac{AC\cdot DB}{AD\cdot BC}+1=-\left(\frac{AC}{CB}\cdot\frac{DB}{AD}\right)+1=-\left(\frac{AC}{CB}\Big/\frac{AD}{DB}\right)+1=-(l\{A,B;C,D\})+1=-(\lambda)+1=1-\lambda$ y aplicando de nuevo la Definición~\ref{DRC} tenemos que $l\{B,D;A,C\}=l\{C,A;D,B\}=l\{D,B;C,A\}=1-\lambda$.

Análogo a lo que se hizo con las dos primeras razones, tenemos que $l\{A,C;D,B\}=l\{B,D;C,A\}=l\{C,A;B,D\}=l\{D,B;A,C\}=\frac{1}{(1-\lambda)}$.

Ahora consideremos $l\{A,D;B,C\}$, aplicando la Definición~\ref{DRC} tenemos que $l\{A,D;B,C\}=\frac{AB}{BD}\Big/\frac{AC}{CD}=\frac{AB}{BD}\cdot\frac{CD}{AC}$, entonces por el Teorema de Euler $l\{A,D;B,C\}=\frac{-(AC\cdot DB+AD\cdot BC)}{AC\cdot BD}=\frac{AC\cdot DB+AD\cdot BC}{AC\cdot DB}=\frac{AC\cdot DB}{AC\cdot DB}+\frac{AD\cdot BC}{AC\cdot DB}=1+\left(\frac{AD}{DB}\cdot\frac{BC}{AC}\right)=1-\left(\frac{AD}{DB}\cdot\frac{BC}{CA}\right)=1-l\{A,B;D,C\}=1-\frac{1}{\lambda}=\frac{\lambda -1}{\lambda}$. Aplicando de nuevo la definición de razón cruzada tenemos que también $l\{B,C;D,A\}=l\{C,B;D,A\}=l\{D,A;C,B\}=\frac{\lambda -1}{\lambda}$.

Como se hizo anteriormente, tenemos que $l\{A,D;C,B\}=l\{B,C;A,D\}=l\{C,B;A,D\}=l\{D,A;B,C\}=\frac{\lambda}{\lambda - 1}$.

Así tenemos el siguiente teorema.
\begin{teo}
Sean $\{A,B,C,D\}\subset l$ tales que $\{A,B;C,D\}=\lambda$, entonces para cualquier permutación de estos puntos, las razones cruzadas son 
$$\lambda,\;\frac{1}{\lambda},\; 1-\lambda,\;\frac{1}{(1-\lambda)},\;\frac{\lambda -1}{\lambda},\;\text{ó}\;\frac{\lambda}{\lambda - 1}.$$
\end{teo}


\section{Construcción del cuarto elemento dados tres}

Sea $l$ una recta tal que $\{A,B,C\}\subset l$ y $\lambda\in\mathbb{R}$, debemos construir $D\in l$ tal que $l\{A,B;C,D\}=\lambda$. 

Construcción: Sea $l$ una recta tal que $\{A,B,C\}\subset l$, $m\neq l$ una recta por $C$. Construir $\{E,F\}\subset m$ tales que $\frac{CE}{CF}=\lambda$. 

Sean $n$ la recta determinada por $A,E$, $o$ la recta determinada por $B,F$, $n\cap o=\{P\}$ y $p$ la recta paralela a $m$ por $P$, $p\cap l=\{D\}$, el punto que buscamos pues tenemos que $\triangle ACE\cong\triangle ADP$ \textbf{cs(AA)} ya que $|\angle CAE|=|\angle DAP|$ y $|\angle APD|=|\angle AEC|$. También tenemos que $\triangle BDP\cong\triangle BCF$ \textbf{cs(AA)} pues $|\angle CBF|=|\angle DBP|$ y $|\angle BPD|=|\angle BFC|$. Así, de las semejanzas anteriores tenemos que $\frac{AC}{AD}=\frac{CE}{DP}$ y $\frac{BD}{BC}=\frac{DP}{CF}$, entonces $\frac{AC}{AD}\cdot\frac{BD}{BC}=\frac{CE}{CF}$, con lo que tenemos que $\frac{AC}{BC}\cdot\frac{BD}{AD}=\lambda$, por lo tanto $\frac{AC}{CB}\cdot\frac{DB}{AD}=\lambda$.

Con esto concluimos que la $D$ encontrada es tal que $l\{A,B;,C,D\}=\lambda$.

Además notemos que la construcción realizada asegura la existencia y unicidad del cuarto elemento.
\section{Propiedades de la razón cruzada en una circunferencia}
\begin{obs}\label{ORCC}
Sea $\mathcal{C}(O,r)$ y $\{P,Q,R,X,Y\}\subset \mathcal{C}$. Notemos los siguiente:
\begin{enumerate}
\item Si $P\in\widehat{YX}$ y $Q\in\widehat{YX}$, entonces $\angle XPY=\angle XQY$.
\item Si $P\in\widehat{YX}$ y $R\in\widehat{XY}$, entonces $\angle XPY+\angle YRX=2\perp$, así $\angle XPY=2\perp -\angle YRX$, entonces 
\begin{eqnarray*}
\sen(\angle XPY)
&=& (2\perp -\sen\angle YRX)\\
&=& \sen(2\perp)\cos(-\angle YRX)+\sen(-\angle YRX)\cos(2\perp)\\
&=& -\sen(-\angle YRX)\\
&=& \sen(\angle YRX)\\
\end{eqnarray*}
\end{enumerate}
\end{obs}

\begin{prop}\label{PPRCC}
Sea $\mathcal{C}(O,r)$ una circunferencia en el plano y $\{A,B,P,Q,R,S\}\subset\mathcal{C}$ (puntos distintos), entonces $A\{P,Q;R,S\}=B\{P,Q;R,S\}$.
\end{prop}
\begin{pba}
Tenemos que 
\begin{eqnarray*}
A\{P,Q;R,S\}
&=& \frac{\sen(\angle PAR)}{\sen(\angle RAQ)}\Big/\frac{\sen(\angle PAS)}{\sen(\angle SAQ)}\\
&=& \frac{\sen(\angle PAR)}{\sen(\angle RAQ)}\Big/\frac{\sen(\angle PAS)}{\sen(\angle SAQ)}\\
&=& \left(\frac{\sen(\angle PAR)}{\sen(\angle RAQ)}\right)\left(\frac{\sen(\angle SAQ)}{\sen(\angle PAS)}\right)\\
&=& \left(\frac{\sen(\angle PBR)}{\sen(\angle RBQ)}\right)\left(\frac{\sen(\angle SBQ)}{\sen(\angle PBS)}\right)\;\;\;(\text{Observaciòn}~\ref{ORCC})\\ 
&=& \frac{\sen(\angle PBR)}{\sen(\angle RBQ)}\Big/\frac{\sen(\angle PBS)}{\sen(\angle SBQ)}\\
&=& B\{P,Q;R,S\}\\
\end{eqnarray*}
\end{pba}
\section{Teorema de Pascal}
\begin{teo}[Teorema de Pascal]\index{Teorema ! de Pascal}
Sean $\{A,B,C,D,E,F\}\subset\mathcal{C}$. Consideremos el hexágono $ABCDEF$. Sean $AB\cap DE=\{P\}$, $BC\cap EF=\{Q\}$, $CD\cap FA=\{R\}$, entonces $P,Q,\;\text{y}\;R$ son colineales.
\end{teo}
\begin{dem}
Sean $DE\cap FA=\{V\}$ y $AB\cap EF=\{W\}$.
Por la Proposición~\ref{PPRCC} tenemos que 
$D\{A,F;E,C\}=B\{A,F;E,C\}$. Ahora notemos que $A,F,W,R$ son colineales, sea $l=\overline{FA}$ y además $l$ es transversal al haz en $D$, entonces por el Teorema~\ref{T1RRC} tenemos que $D\{A,F;E,C\}=l\{A,F;V,R\}$.

También notemos que $Z,F,E,Q$ son colineales, sea $m=\overline{EF}$, tenemos que $m$ es transversal al haz en $B$, entonces por el Teorema~\ref{T1RRC},  $B\{A,F;E,C\}=m\{Z,F;E,Q\}$.

Por tanto, $l\{A,F;V,R\}=m\{Z,F;E,Q\}$. Ahora, aplicando nuevamente el Teorema~\ref{T1RRC} se tiene que $l\{A,F;V,R\}=P\{A,F;V,R\}$ y $m\{Z,F;E,Q\}=P\{Z,F;E,Q\}$. Con lo que concluimos que $P\{A,F;V,R\}=P\{Z,F;E,Q\}$.

Para finalizar, observemos que $\overline{PA}=\overline{PZ}, \overline{PF}=\overline{PF}, \overline{PE}=\overline{PV}$  y dado que el cuarto elemento es único, concluimos que $\overline{PR}=\overline{PQ}$, por lo tanto $P$, $Q$ y $R$ son colineales. 
\end{dem}

\section{Teorema de Brianchon}
\begin{teo}[Teorema de Brianchon]\index{Teorema ! de Brianchon}
Sea $\mathcal{C}(O,r)$ una circunferencia en el plano y $ABCDEF$ un hexágono circunscrito a $\mathcal{C}$, entonces $AD\cap BE\cap CF\neq\emptyset$.
\end{teo}
%%%%%%%%%%%%%%%%%%%%% 
%Este tema no se vió, no se me ocurrió como probar el teorema usando herramientas de Moderna 1
%%%%%%%%%%%%%%%%%%%%
%%%%%%%%%%%%%%%%%%%%
\section{Teorema de Pappus}
\begin{teo}[Teorema de Pappus]\index{Teorema ! de Pappus}
Sean $l$ y $m$ dos rectas en el plano tales que $\{A,C,E\}\subset l$ y $\{B,D,F\}\subset m$. Consideremos el hexágono $ABCDEF$ y sean $AB\cap DE=\{P\}, BC\cap EF=\{Q\}, CD\cap FA=\{R\}$, entonces $P$, $Q$ y $R$ son colineales. 
\end{teo}
\begin{dem}
Sean $DE\cap FA=\{G\}$ y $CD\cap EF=\{H\}$. Por el Teorema~\ref{T1RRC} tenemos que $A\{E,B;D,F\}=C\{E,B;D,F\}$. Notemos que $E,P,D,G$ son colineales y sea $\overline{DE}=l$, entonces por el Teorema~\ref{T1RRC} tenemos que $l\{E,P;D,G\}=A\{E,B;D,F\}$. También notemos que $E,Q,H,F$ son colineales y sea $\overline{EF}=m$, entonces por el Teorema~\ref{T1RRC} se tiene que $m\{E,Q;H,F\}=C\{E,B;D,F\}$. Por último aplicando de nuevo el Teorema~\ref{T1RRC} tenemos que $R\{E,P;D,G\}=R\{E,Q;H,F\}$ y como $\overline{RE}=\overline{RE}, \overline{RD}=\overline{RH}, \overline{RG}=\overline{RF}$ y dado que el cuarto elemento es único, entonces $\overline{RQ}=\overline{RP}$ por tanto $P$, $Q$ y $R$ son colineales. 

\end{dem}
\section{Puntos autocorrespondientes}

\section{Regla geométrica de la falsa posición}

\section{Problema de Apolonio}


\subsection*{Ejercicios}
\begin{enumerate}
\item Probar el Corolario~\ref{C1RC}, pàgina \pageref{C1RC}. \label{E1RC}
\item Probar el Corolario~\ref{C2RC}, pàgina \pageref{C2RC}. \label{E2RC}

\end{enumerate}




\chapter{Puntos y líneas armónicos}

\section{División armónica}
\begin{df}\label{DCA}
Sean $\{A,B,C,D\}\subset l$, $A$ y $B$ son \textcolor{red}{\bf conjugados armónicos}\index{conjugados armónicos} de $C$ y $D$ si y solamente si $l\{A,B;C,D\}=-1$ y lo  denotaremos como $l(A,B;C,D)$.

De manera similar, sean $a,b,c,d$ rectas concurrentes, $a\cap b\cap c\cap d=\{O\}$, $a$ y $b$ son conjugadas armónicas de $c$ y $d$ si y solamente si $O\{a,b;c,d\}=-1$ y lo denotaremos como $O(a,b;c,d)$.
\end{df}
\section{La naturaleza recíproca de la división armónica}
De la Definición~\ref{DCA} observemos que
$l(A,B;C,D)\leftrightarrow l\{A,B;C,D\}=-1\leftrightarrow \frac{AC}{CB}\Big/\frac{AD}{DB}=-1\leftrightarrow \frac{AC}{CB}=-\frac{AD}{DB}\leftrightarrow\frac{CA}{AD}=-\frac{CB}{BD}$. Esto se resume en el siguiente teorema.

\begin{teo}\label{THA}
Sean $\{A,B,C,D\}\subset l$, si $A$ y $B$ son conjugados armónicos con respecto a $C$ y $D$, entonces $C$ y $D$ son conjugados armónicos con respecto de $A$ y $B$. 
\end{teo}

Si tenemos $\{A,B,C,D\}\subset l$ de tal forma que un par de puntos es armónico con respecto al otro par y viceversa entonces a estos cuatro puntos se les conoce como \textcolor{red}{\bf puntos armónicos}\index{puntos armónicos} o también se dice que conforman un \textcolor{red}{\bf hilera armónica}\index{hilera armónica}.

\section{Construcción de conjugados armónicos}
En está sección veremos como construir el conjugado armónico de un punto con respecto a otros dos, más adelante el lector se percatar que está construcción nos es única.

Sea $l$ una recta en el plano y sean $A,B,C\in l$.

Construcción:
\begin{itemize}
\item Trazar por $A$ una recta $a$.
\item trazar por $B$ una recta $b$ tal que $a\parallel b$.
\end{itemize}
Sea $l'$ por $C$ tal que $l'\cap a=\{M\}$ y $l'\cap b=\{N\}$ con $l\neq l'$. Sea $R\in b$ tal que $NB=BR$ y sea $MR\cap l=\{D\}$, entonces $l\{A,B;C,D\}=-1$.

En efecto, consideremos $\triangle AMC$ y $\triangle BNC$, entonces $\triangle AMC\cong\triangle BNC$ \textbf{cs(AA)} ya que $|\angle MCA|=|\angle NCB|$ y $|\angle AMC|=|\angle BNC|$. Entonces por el Teorema~\ref{Thales1} se tiene que $\frac{AM}{BN}=\frac{MC}{NC}=\frac{AC}{BC}$, así $\frac{AC}{BC}=\frac{AM}{BN}$ si y sólo si $\frac{AC}{CB}=\frac{AM}{NB}$.

Por otra parte, también tenemos que $\triangle AMD\cong\triangle BRD$ \textbf{cs(AA)} pues $|\angle MDA|=|\angle RDB|$ y $|\angle BRD|=|\angle AMD|$. Entonces, $\frac{AM}{BR}=\frac{MD}{RD}=\frac{AD}{BD}$, así $\frac{AD}{BD}=\frac{AM}{BR}$ si y sólo si $-\frac{AD}{DB}=\frac{AM}{BR}$ si y sólo si $\frac{AD}{DB}=-\frac{AM}{BR}$. Y como $NB=BR$, entonces:
\begin{eqnarray*}
\left(\frac{AC}{CB}\right)\left(\frac{AD}{DB}\right)=\left(\frac{AM}{NB}\right)\left(-\frac{AM}{BR}\right)
&\leftrightarrow & \left(\frac{AC}{CB}\right)\left(\frac{DB}{AD}\right)=\left(\frac{NM}{AM}\right)\left(-\frac{AM}{NB}\right)\\
&\leftrightarrow & \left(\frac{AC}{CB}\right)\left(\frac{DB}{AD}\right)=-1\\
&\leftrightarrow & \left(\frac{AC}{CB}\right)=\left(-\frac{AD}{DB}\right)\\
&\leftrightarrow & \frac{AC}{CB}\Big/\frac{AD}{DB}=-1\\
\end{eqnarray*}

Por lo tanto, $$l\{A,B;C,D\}=-1$$.
\begin{obs}\label{OBPAI}
Antes de continuar debe mencionarse que considerando la construcción anterior, hay un caso particular, cuando $C$ es el punto medio de $AB$, en esta caso resulta que el conjugado armónico de $C$ con respecto de $A$ y $B$ es el punto al infinito en $\overline{AB}$.
\end{obs}
\section{Propiedades de los puntos armónicos}
\begin{prop}
$l(A,B;C,D)\leftrightarrow l(B,A;C,D)\leftrightarrow l(A,B;D,C)\leftrightarrow l(B,A;D,C)\leftrightarrow l(C,D;A,B)\leftrightarrow l(D,C;A,B)\leftrightarrow l(C,D;B,A)\leftrightarrow l(D,C;B,A)$.
\end{prop}

\begin{prop}\label{P2PPA}
Sean $\{A,B,C,D,O\}\subset l$ tal que $|AO|=|OB|$, entonces $l\{A,B;C,D\}=-1$ si y solamente si $OB^{2}=OC\cdot OD$.
\end{prop}
\begin{pba}

\begin{eqnarray*}
l\{A,B;C,D\}=-1
&\leftrightarrow & \frac{AC}{CB}\Big/\frac{AD}{DB}=-1\\
&\leftrightarrow & \frac{AC}{CB}=-\frac{AD}{DB}\\
&\leftrightarrow & \frac{AO+OC}{CO+OB}=-\frac{AO+OD}{DO+OB}\\
&\leftrightarrow & \frac{OB+OC}{CO+OB}=-\frac{OB+OD}{DO+OB}\;\; (pues\;AO=OB)\\
&\leftrightarrow & \frac{OB+OC}{OB-OC}=\frac{OB+OD}{OD-OB}\\
&\leftrightarrow & OB+OC(OD-OB)=OB+OD(OB-OC)\\
&\leftrightarrow & OB\cdot OD+OC\cdot OD-OB^{2}-OC\cdot OB\\
& & =OB^{2}+OD\cdot OB-OB\cdot OC-OD\cdot OC\\
&\leftrightarrow & OB\cdot OD-OD\cdot OB-OB^{2}-OB^{2}-OC\cdot OB\\
& & +OC\cdot OB=-OD\cdot OC-OC\cdot OD\\
&\leftrightarrow & -2OB^{2}=-2(OC\cdot OD)\\
&\leftrightarrow & OB^{2}=OC\cdot OD\\
\end{eqnarray*}
\end{pba}

\section{Líneas armónicas}
Sean $a=\overline{OA}, b=\overline{OB}, c=\overline{OC}, d=\overline{OD}$ rectas en el plano, 
entonces decimos que $c$ y $d$ son conjugadas armónicas con respecto a $a$ y $b$ si 

$$\frac{\sen(\angle AOC)}{\sen (\angle COB)}=-\frac{\sen (\angle AOD)}{\sen (\angle DOB)}$$
($O$ un punto en el plano). Como se mencionó en la Definición~\ref{DCA}, cuando se presenta esta situación en un haz de rectas la notación a utilizar será $O(a,b;c,d)$ ò en este caso $O(A,B;C,D)$.
 
Si un haz de cuatro rectas $a,b,c,d$ cumple la igualdad anterior también se dice que $a$ y $b$ están separadas armónicamente por $c$ y $d$.

Ahora enunciaremos un teorema similar al Teorema~\ref{THA} pero en lugar de puntos utilizaremos rectas.

\begin{teo}\label{TRA}
Sean $a,b,c,d$ tales que $a\cap b\cap c\cap d=\{O\}$ (donde $O$ es un punto en el plano), si $a$ y $b$ son conjugadas armónicas con respecto a $c$ y $d$, entonces $c$ y $d$ son conjugadas armónicas con respecto a $a$ y $b$. 
\end{teo}
\begin{dem}
Ver sección de ejercicios, Ejercicio~\ref{ETRA}.
\end{dem}
Si cuatro líneas cumplen el Teorema~\ref{TRA}, se dice que son \textcolor{red}{\bf líneas armónicas}\index{líneas armónicas} y el haz que constituyen se conoce como \textcolor{red}{\bf haz armónico}\index{haz armónico}.
\section{Transversal de un haz armónico}
\begin{teo}\label{T1TDHA}
\begin{enumerate}

\item Sean $a,b,c,d$ líneas armónicas en el plano tales que $a\cap b\cap c\cap d=\{O\}$  y $l$ una recta tal que $O\notin l$. Si $l\cap a=\{A\}, l\cap b=\{B\}, l\cap c=\{C\}, l\cap d=\{D\}$, entonces $A,B,C,D$ son puntos armónicos. 
\item Sean $\{A,B,C,D\}\subset l$ puntos armónicos y $O$ un punto en el plano tal que $O\notin l$, entonces $\overline{OA}, \overline{OB}, \overline{OC}, \overline{OD}$ son líneas armónicas.
\end{enumerate}
\end{teo}
\begin{dem}
Primero probemos $1$. Como $a,b,c,d$ son líneas armónicas, sin perder generalidad supongamos que $O\{a,b;c,d\}=-1$, entonces por el Teorema~\ref{T1RRC} tenemos que $l\{A,B;C,D\}=-1$. Por tanto, $A,B;C,D$ son puntos armònicos.

Análogamente probaremos $2$. Como $A,B;C,D$ son puntos armónicos, entonces supongamos que $l\{A,B;C,D\}=-1$ por lo cual aplicando el Teorema~\ref{T1RRC} se tiene que $O\{\overline{OA}, \overline{OB}; \overline{OC}, \overline{OD}\}=-1$ por tanto $\overline{OA}, \overline{OB}, \overline{OC}, \overline{OD}$ son líneas armónicas.
\end{dem}

\begin{cor}
Sean $a,b,c,d$ rectas en el plano tales que $a\cap b\cap c\cap d=\{O\}$, $l$ una recta tal que $O\notin l$ y $l\cap a=\{A\}, l\cap b=\{B\}, l\cap c=\{C\}, l\cap d=\{D\}$, $l\{A,B;C,D\}=-1$. Si $m$ es una recta en el plano ($m\neq l$ y $O\notin m$) tal que $m\cap a=\{E\}, m\cap b=\{F\}, m\cap c=\{G\}, m\cap d=\{H\}$, entonces $m\{E,F;G,H\}=-1$. 
\end{cor}
\begin{pba}\label{CTDHA}
Como $l\{A,B;C,D\}=-1$, entonces por el Teorema~\ref{T1TDHA} (inciso 2), se tiene que $O\{a,b;c,d\}=-1$. Además, como $m\cap a=\{E\}, m\cap b=\{F\}, m\cap c=\{G\}, m\cap d=\{H\}$, aplicando de nuevo Teorema~\ref{T1TDHA} (inciso 1), tenemos que $m\{E,F;G,H\}=-1$.
\end{pba}
\section{Hileras armónicas en perspectiva}
\begin{teo}
Sean $l$ y $l'$ rectas tales que $l\neq l'$ con $\{A,B,C,D\}\subset l$ y $\{A,B',C',D'\}\subset l'$, consideremos las rectas $b,c,d$ tales que $\{B,B'\}\subset b$, $\{C,C'\}\subset c$, $\{D,D'\}\subset d$. Si $l\{A,B;C,D\}=l'\{A,B';C',D'\}=-1$, entonces $b\cap c\cap d\neq\emptyset$.
\end{teo}
\begin{dem}
Supongamos que $l\{A,B;C,D\}=l'\{A,B';C',D'\}=-1$ y que $b\cap c=\{P\}$. 
Consideremos lo siguiente, sea $m$ la recta tal que $\{A,P\}\subset m$, $n$ la recta tal que $\{D,P\}\subset n$ y $n\cap l'=\{D''\}$. Entonces, por el Teorema~\ref{T1TDHA} $P\{A,B;C,D\}=-1$ pues $l$ es transversal a este haz. Así, por el Corolario~\ref{CTDHA} $l'\{A,B';C',D''\}=-1$ pues $l'$ es transversal al haz $P\{A,B;C,D\}=-1$. 

Por lo tanto, $l'\{A,B';C',D''\}=l'\{A,B';C',D'\}$, entonces $D''=D'$ y así concluimos que $b\cap c\cap d=\{P\}$.
\end{dem}
\section{Líneas conjugadas ortogonales}
\begin{teo}
Sean $a,b,c,d$ rectas en el plano, $A\in a, B\in b, C\in c, D\in d$, tales que $O(A,B;C,D)$ y $O\notin\{A,B,C,D\}$. Entonces $c$ es ortogonal a $d$ si y solamente si $|\angle AOC|=|\angle COB|$ y $|\angle BOD|=|\angle DOA|$.
\end{teo}
\begin{dem}
\begin{enumerate}
\item [($\Rightarrow$)] Supongamos que $c$ es ortogonal a $d$. Sea $l$ una recta ($l\neq d$) tal que $l\parallel d$ y sean $l\cap a=\{A'\}, l\cap b=\{B'\}, l\cap c=\{C'\}, l\cap d=\{D'\}$ donde $D'$ es el punto al infinito en $l$, así resulta que el conjugado armónico de $C'$ es un punto al infinito, entonces por la Observación~\ref{OBPAI} tenemos que $|A'C'|=|C'B'|$. Ahora, notemos que $\triangle OA`C`\equiv\triangle OB'C'$ \textbf{cc(LAL)} ya que ademàs de que $|A'C'|=|C'B'|$, $OC'$ es común y  como $c$ es ortogonal a $d$ y $l$ es paralela a $d$, $c$ es ortogonal a $l$, entonces $|\angle OC'A'|=|\angle OC'B'|=\perp$. Así que como consecuencia se tiene que $|\angle A'OC'|=|\angle C'OB'|$ y así que $|\angle AOC|=|\angle COB|$ (por colinealidad) , por ello $OC'$ es bisectriz interna del $\angle A'OB'$. Y como las bisectrices de un ángulo se cortan ortogonalmente, entonces $d$ es bisectriz externa del $\angle A'OB'$ y así $|\angle BOD|=|\angle DOA|$.
\item [($\Leftarrow$)] Supongamos que $|\angle AOC|=|\angle COB|$ y $|\angle BOD|=|\angle DOA|$. Sea $\{A,A''\}\subset a$ tal que $0<\frac{AO}{OA''}$. Sean $B\in b, C\in c, D\in d$, entonces $\angle AOC=\angle COB$ y $\angle BOD=\angle DOA''$, por ésto $\sen(\angle AOC)=\sen(\angle COB)$ y como $0<\frac{AO}{OA''}$, entonces $\angle AOD+\angle DOA''=2\perp$. Por tanto, $\sen(\angle AOD)=\sen(\angle DOA'')=\sen(\angle BOD)=-\sen(\angle DOB)$, entonces $\frac{\sen(\angle AOC)}{\sen(\angle AOD)}=\frac{\sen(\angle COB)}{-\sen(\angle DOB)}$ si y sólo si $\frac{\sen(\angle AOC)}{\sen(\angle COB)}=-\frac{\sen(\angle AOD)}{\sen(\angle DOB)}$.
\end{enumerate}
\end{dem}


\section{Curvas ortogonales}
\begin{df}
Sean $\mathcal{C}(O,r)$ y $\mathcal{C'}(O',r')$ tales que ($O\neq O'$) y $\mathcal{C}\cap\mathcal{C'}\neq\emptyset$. Si $T\in\mathcal{C}\cap\mathcal{C'}$ y $t$ es la recta tangente a $\mathcal{C}$ por $T$ y $t'$ es la recta tangente a $\mathcal{C'}$ por $T$. Definimos el \textcolor{red}{\bf ángulo entre $\mathcal{C}$ y $\mathcal{C'}$}\index{ángulo ! entre $\mathcal{C}$ y $\mathcal{C'}$} como: $\angle(\mathcal{C}\longrightarrow^{T}\mathcal{C'})=\angle (t\longrightarrow^{T} t')$.
\end{df}
\begin{prop}
Sean $\mathcal{C}(O,r)$ y $\mathcal{C'}(O',r')$ tales que $\mathcal{C}\cap\mathcal{C'}=\{A,B\}$, entonces $|\angle(\mathcal{C}\longrightarrow^{A}\mathcal{C'})|=|\angle(\mathcal{C}\longrightarrow^{B}\mathcal{C'})|$.
\end{prop}
\begin{pba}
Supongamos que $\mathcal{C}\cap\mathcal{C'}=\{A,B\}$ y sean $a$ la recta tangente a $\mathcal{C}$ por $A$ y $b$ la recta tangente a $\mathcal{C}$ por $B$, tales que $a\cap b=\{P\}$. Consideremos $\triangle OAB$, entonces tenemos $\triangle OAB$ es isósceles pues $|OA|=|OB|=r$. Ahora consideremos $\triangle OAP$ y $\triangle OBP$, como $OA$ es ortogonal a $AP$ pues $A,P\in a$, entonces $|\angle OAP|=\perp$ del mismo modo se tiene que $|\angle PBO|=\perp$, también tenemos que $|OA|=|OB|=r$ y $OP$ es común, entonces podemos aplicar el Lema~\ref{lb} con lo que concluimos que $\triangle OAP\equiv\triangle OBP$, por tanto $|PA|=|PB|$ y así concluimos que $P$ y $O$ pertenecen a la mediatriz de $A$ y $B$.

Análogamente, si $a'$ es la recta tangente a $\mathcal{C'}$ por $A$ y $b'$ es la recta tangente a $\mathcal{C'}$ por $B$ tales que $a'\cap b'=\{P'\}$, se tiene que $P$ y $O'$ están en la mediatriz de $AB$ y que $|P'A|=|P'B|$.

Finalmente, también tenemos que $\triangle PAP'\equiv\triangle PBP'$ \textbf{LLL} pues $|PA|=|PB|$, $|P'A|=|P'B|$ y $|PP'|=|PP'|$, entonces $|\angle PAP'|=|PBP'|$. Por lo tanto, $| \angle(\mathcal{C}\longrightarrow^{A}\mathcal{C'}|
)|=|\angle(\mathcal{C}\longrightarrow^{B}\mathcal{C'})|$.
\end{pba}
\begin{df}
Sean $\mathcal{C}(O,r)$ y $\mathcal{C'}(O',r')$ con $O\neq O'$. Decimos que $\mathcal{C}$ y $\mathcal{C'}$ son \textcolor{red}{\bf circunferencias ortogonales}\index{circunferencias ortogonales} si y solamente si $|\angle(\mathcal{C}\longrightarrow\mathcal{C'}|=\perp$.
\end{df}

\begin{prop}\label{PCO}
Si $\mathcal{C}(O,r)$ y $\mathcal{C'}(O',r')$ son ortogonales, entonces la tangente a cualquiera de estas circunferencias contiene al centro de la otra circunferencia. 
\end{prop}
\begin{pba}
Ver sección de ejercicios, Ejercicio~\ref{EPCO}.
\end{pba}
\section{Una propiedad armónica en relación con circunferencias ortogonales}
\begin{teo}
Sean $\mathcal{C}(O,r)$ y $\mathcal{C'}(O',r')$ tal que $\mathcal{C}\cap\mathcal{C'}=\{P,Q\}$. Cosideremos $l$ la recta tal que $\{O,O'\}\subset l$ y sean $l\cap\mathcal{C}=\{A,B\}$, $l\cap\mathcal{C'}=\{C,D\}$. Entonces $l(A,B;C,D)$ si solamente si $\mathcal{C}(O,r)$ es ortogonal a $\mathcal{C'}(O',r')$.
\end{teo}
\begin{dem}
\begin{enumerate}
\item [($\Rightarrow$)] Supongamos que $l(A,B;C,D)$, como $AB$ es diámetro $\mathcal{C}$, $|AO|=|OB|~$, entonces por la Proposición~\ref{P2PPA} tenemos que $OB^{2}=OC\cdot OD$ pero como $|OB|=|OP|=r$, entonces $OP^{2}=OC\cdot OD$ (la potencia de $O$ con respecto a $\mathcal{C'}$), entonces $\overline{OP}$ es tangente a $\mathcal{C'}$ en $P$ y por tanto $\mathcal{C}(O,r)$ es ortogonal a $\mathcal{C'}(O',r')$.
\item [($\Leftarrow$)] Ahora supogamos que $\mathcal{C}(O,r)$ es ortogonal a $\mathcal{C'}(O',r')$, como $P\in\mathcal{C'}$, entonces la tangente por $P$ a $\mathcal{C'}$ contiene a $O$ (Proposición~\ref{PCO}). Además como el diámetro $AB$ de $\mathcal{C}$ es tal que $AB\cap\mathcal{C'}=\{C,D\}$, entonces al considerar la potencia de $O$ con respecto de $\mathcal{C'}$ se tiene que $OP^{2}=OC\cdot OD$ (pues $\overline{OP}$ es tangente a $\mathcal{C'}$). Ahora como $|OB|=|OP|=r$, tenemos que $OB^{2}=OP^{2}=OC\cdot OD$, entonces por la Proposición~\ref{P2PPA} concluimos que $l(A,B;C,D)$.
\end{enumerate}
\end{dem}
\section{Cuadrángulo completo}
\begin{df}\label{CNCD}
Sean $A,B,C,D$ cuatro puntos (vértices) en posición general, es decir, cualesquiera 3 puntos no son colineales. Definimos el \textcolor{red}{\bf cuadrángulo completo}\index{cuadrángulo completo} como la figura determinada por los cuatro puntos y las seis rectas (lados) que determinan esos cuatro puntos. 
\end{df}
\begin{df}
Sea $ABCD$ un cuadrángulo completo. Decimos que dos lados son opuestos si y solamente si no tienen vèrtices en común. A los vértices determinados por los lados opuestos les llamamos \textcolor{red}{\bf puntos diagonales}\index{puntos diagonales} . Y al triángulo formado por los puntos diagonales se le llamará \textcolor{red}{\bf triángulo diagonal}\index{triángulo diagonal} del cuadrángulo $ABCD$.
\end{df}

\section{Cuadrilátero completo}
\begin{df}\label{CLCD}
Sean $a,b,c,d$ cuatro rectas (lados) en posición general, es decir, cualesquiera 3 rectas no son concurrentes. Definimos el \textcolor{red}{\bf cuadrilátero completo}\index{cuadrilátero completo} como la figura comprendida por las cuatro rectas y los seis puntos (vértices) que determinan esas cuatro rectas. 
\end{df}
\begin{df}
Sea el cuadrilátero completo $abcd$. Decimos que dos vértices son opuestos si y solamente si no estàn en la misma recta. La recta determinada por los vértices opuestos se le llamará \textcolor{red}{\bf recta diagonal}\index{recta diagonal}. Y al trilátero (ver Definición~\ref{trilátero}) formado por las rectas diagonales se le llamará \textcolor{red}{\bf trilátero diagonal}\index{trilátero diagonal} del cuadrilátero completo $abcd$.
\end{df}
\section{Principio de dualidad}
La Definición~\ref{CLCD} y la Definición~\ref{CNCD} son un ejemplo del principio de dualidad. Si observemos que si en la Definición~\ref{CLCD} intercambiamos rectas por puntos, concurrencias por colinealidades y otros mínimos cambios de escritura, podemos obtener la Definición~\ref{CNCD}. Este principio de dualidad es visto a mayo profundidad en un curso de geometría proyectiva en donde es de gran ayuda ya que dado un  teorema se puede probar que al dualizarlo el teorema dual es verdadero si el teorema inicial es también cierto. 

Otro ejemplo en donde se puede apreciar el principio de dualidad lo podemos encontrar en la Definición~\ref{PDUP} y la Definición~\ref{PDUR} que vimos en el capítulo 7.

Un último ejemplo se da en las siguientes definiciones, que se usaron previamente.

\begin{df}\label{trilátero}
Un \textcolor{red}{\bf trilátero}\index{trilátero} es una figura que consiste en tres rectas no concurrentes y los tres puntos que éstas determinan. 
\end{df}
\begin{df}
Un \textcolor{red}{\bf triángulo}\index{triángulo} es la figura que consiste en tres puntos no colineales y las tres rectas que éstos determinan.   
\end{df}
\section{Propiedades armónicas de cuadrángulos y cuadriláteros}
\begin{teo}
En cada lado del trilátero diagonal de un cuadrilátero completo hay una hilera de puntos armónicos conformada por los dos vértices en la diagonal y los puntos por los que es intersecada por las diagonales restantes.
\end{teo}
\begin{pba}
Sean $a,b,c,d$ rectas en el plano en posición general y sean $a\cap b=\{E\}, a\cap c=\{F\}, a\cap d=\{G\}, b\cap c=\{H\}, b\cap d=\{I\}, c\cap d=\{J\}$. 

Sea $p$ la recta determinada por $E,J$, $q$ la recta determinada por $F,I$ y $r$ la recta determinada por $G,H$, tales que $p\cap q=\{R\}, q\cap r=\{P\}, r\cap p=\{Q\}$. 

Consideremos $\triangle JEH$, tenemos que $\{J,I\}\subset d, \{E,F\}\subset a, \{H,Q\}\subset r$ y $a\cap d\cap r=\{G\}$. 

Entonces por el Teorema~\ref{Teo de Ceva} tenemos que 
$$\frac{JQ}{QE}\cdot\frac{EI}{IH}\cdot\frac{HF}{FJ}=1$$
También como $\{F,I,R\}\subset q$ por el Teorema~\ref{Teo de Menelao}
$$\frac{JR}{RE}\cdot\frac{EI}{IH}\cdot\frac{HF}{FJ}=-1$$
Por tanto $\frac{JQ}{QE}=-\frac{JR}{RE}$, entonces $p(J,E;Q,R)$ así que por el Teorema~\ref{T1RRC} para cualquier $X\notin p$ se tiene que $X(J,E;Q,R)$, entonces $H(J,E;Q,R)$ y aplicando nuevamente el Teorema~\ref{T1RRC} tenemos que $q(F,I;P,R)$.

Análogamente se tiene que como $p(J,E;Q,R)$, entonces $I(J,E;Q,R)$ y por tanto $r(G,H;Q,P)$.
\end{pba}

\begin{teo}
En cada vértice del triángulo diagonal de un cuadrángulo completo hay un haz de rectas armónicas, conformado por las dos rectas que pasan por el punto diagonal y las rectas que lo unen con los puntos diagonales restantes. 
\end{teo}
\begin{dem}
Sean $A,B,C,D$ puntos en el plano en posición general y sean $\overline{AB}=e$, $\overline{AC}=f$, $\overline{AD}=g$, $\overline{BC}=h$, $\overline{BD}=i$, $\overline{CD}=j$. 

Sea $\{P\}=e\cap j$, $\{Q\}=f\cap i$ y $\{R\}=g\cap h$ tales que $\overline{PQ}=r$, $\overline{QR}=p$, $\overline{RP}=q$. 

Consideremos $\triangle jeh=\triangle PCB$, tenemos que $j\cap i=\{D\}$, $e\cap f=\{A\}$, $h\cap q=\{R\}$ y $\{A,D,R\}\subset g$. 

Entonces por el Teorema~\ref{Teo de Menelao} tenemos que 
$$\frac{PD}{DC}\cdot\frac{CR}{RB}\cdot\frac{BA}{AP}=-1$$
Sean $r\cap h=\{S\}$, $r\cap g=\{T\}$ y $q\cap i=\{U\}$. Como $f\cap i\cap r=\{Q\}$ por el Teorema~\ref{Teo de Ceva}
$$\frac{PD}{DC}\cdot\frac{CS}{SB}\cdot\frac{BA}{AP}=1$$
Por tanto $\frac{CR}{RB}=-\frac{CS}{SB}$, entonces $h(C,B;R,S)$ así que por el Teorema~\ref{T1RRC} para cualquier $X\notin h$ se tiene que $X(C,B;R,S)$, entonces $P(C,B;R,S)$ y por tanto $P(j,e;q,r)$. Y aplicando nuevamente el Teorema~\ref{T1RRC} tenemos que $Q(A,D;R,T)$ y así $Q(f,i;p,r)$.

Análogamente se tiene que como $P(j,e;q,r)$, entonces $i(D,B;U,Q)$ y por tanto $R(g,h;q,p)$.

\end{dem}

\section{Cuadrángulo y cuadrilátero con triángulo diagonal común}
En esta sección veremos cómo a partir de un cuadrilátero completo es posible encontrar un cuadrángulo completo que tenga el mismo triángulo diagonal que el cuadrilátero completo dado.

Sean $a,b,c,d$ cuatro rectas en posición general tales que $a\cap b=\{E\}$, $a\cap c=\{F\}$, $a\cap d=\{G\}$, $b\cap c=\{H\}$, $b\cap d=\{I\}$, $c\cap d=\{J\}$. Así tenemos el cuadrilátero completo $abcd$ cuyo trilátero diagonal está determinado por $\overline{EJ}=p$, $\overline{FI}=q$ y $\overline{GH}=r$, tales que $p\cap q=\{R\}$, $q\cap r=\{P\}$, $p\cap r=\{Q\}$.

Nuestra tarea ahora es encontrar un cuadrángulo completo que tenga como triángulo diagonal al $\triangle PQR$. 

Construcción:

Construir lar rectas por cada vértice del trilátero diagonal con los vértices del cuadrilátero que determinan la diagonal opuesta a dicho vértice. 

Para $P$ la diagonal opuesta es $p$, para $Q$ la diagonal opuesta es $q$ y para $R$ la diagonal opuesta es $r$. Construir $p_{E}=\overline{PE}$, $p_{J}=\overline{PJ}$, $q_{F}=\overline{QF}$, $q_{I}=\overline{QI}$, $r_{G}=\overline{RG}$ y $r_{H}=\overline{RH}$.

Consideremos $\triangle PQR$ y $\triangle JIH$, entonces:

$\overline{PQ}\cap\overline{JI}=\{G\}$,
$\overline{QR}\cap\overline{IH}=\{E\}$,
$\overline{PR}\cap\overline{JH}=\{F\}$,
$r\cap d=\overline{GH}\cap d=\{G\}$,
$P\cap b=\overline{EJ}\cap b=\{E\}$,
$q\cap c=\overline{FI}\cap c=\{F\}$. 
Como $\{E,F,G\}\subset a$, entonces $\triangle PQR$ y $\triangle JIH$ están en perspectiva desde $\overline{PJ}\cap\overline{QI}\cap\overline{RH}$.

Consideremos $\triangle PQR$ y $\triangle EFH$, entonces:

$\overline{PQ}\cap\overline{EF}=r\cap a=\overline{GH}\cap a=\{G\}$,
$\overline{QR}\cap\overline{FH}=p\cap c=\overline{EJ}\cap c=\{J\}$,
$\overline{PR}\cap\overline{EH}=q\cap b=\overline{FI}\cap b=\{I\}$.

Como $\{G,J,I\}\subset d$, entonces $\triangle PQR$ y $\triangle EFH$ están en perspectiva desde $\overline{PE}\cap\overline{QF}\cap\overline{RH}$.

Consideremos $\triangle PQR$ y $\triangle JFE$, entonces:

$\overline{PQ}\cap\overline{JF}=r\cap c=\overline{GH}\cap c=\{H\}$,
$\overline{QR}\cap\overline{FG}=p\cap a=\overline{EJ}\cap a=\{E\}$,
$\overline{PR}\cap\overline{GJ}=q\cap d=\overline{FI}\cap d=\{I\}$.

Como $\{H,E,I\}\subset b$, entonces $\triangle PQR$ y $\triangle JFG$ están en perspectiva desde $\overline{PJ}\cap\overline{QF}\cap\overline{RG}$.


Consideremos $\triangle PQR$ y $\triangle EIG$, entonces:

$\overline{PQ}\cap\overline{EI}=r\cap b=\overline{GH}\cap b=\{H\}$,
$\overline{QR}\cap\overline{IG}=p\cap d=\overline{EJ}\cap d=\{J\}$,
$\overline{PR}\cap\overline{GE}=q\cap a=\overline{FI}\cap a=\{F\}$.

Como $\{H,J,F\}\subset c$, entonces $\triangle PQR$ y $\triangle EIG$ están en perspectiva desde $\overline{PE}\cap\overline{QI}\cap\overline{RG}$.

Ahora nombremos a:
$\{Z\}=\overline{PJ}\cap\overline{QI}\cap\overline{RH}$,
$\{Y\}=\overline{PE}\cap\overline{QF}\cap\overline{RH}$,
$\{X\}=\overline{PJ}\cap\overline{QF}\cap\overline{RG}$ y 
$\{W\}=\overline{PE}\cap\overline{QI}\cap\overline{RG}$. 

Considerando al cuadrángulo $WXYZ$ tenemos que los vértices de su triángulo diagonal son:
$\overline{WX}\cap\overline{YZ}=\{R\}$, $\overline{WZ}\cap\overline{XY}=\{Q\}$ y $\overline{WY}\cap\overline{XZ}=\{P\}$.

De manera análoga, a partir de una cuadrángulo completo se puede construir un cuadrilátero completo cuyo triángulo diagonal sea el mismo que el del cuadrángulo completo dado. 

\subsection*{Ejercicios}
\begin{enumerate}
\item Demostrar el Teorema~\ref{TRA}, página \pageref{TRA}. \label{ETRA}
\item Probar la Proposición~\ref{PCO}, página \pageref{PCO}. \label{EPCO}
\end{enumerate}




\chapter{Triángulo}

\section{Triángulo pedal}
\begin{df}
Sean $\triangle ABC$, $h_{A}\cap BC=\{D\}$, $h_{B}\cap CA=\{E\}$, $h_{C}\cap AB=\{F\}$, entonces consideremos $\triangle DEF$, a este triángulo se le conoce como el \textcolor{red}{\bf triángulo pedal}\index{triángulo ! pedal} del $\triangle ABC$.
\end{df}
\section{Propiedades que se refieren al incírculo y excírculos}
Primero haremos una construcción que nos servirá de apoyo para observar algunas propiedades sobre el incírculo y los excírculos de un triángulo. 

Sea $\triangle ABC$, construir las 6 bisectrices. Por $A$, sean $a_{I}$ la bisectriz interna y $a_{E}$ la bisectriz externa. 
Por $B$, sean $b_{I}$ la bisectriz interna y $b_{E}$ la bisectriz externa. 
Por $C$, sean $c_{I}$ la bisectriz interna y $c_{E}$ la bisectriz externa. 

Sabemos que $a_{I}\cap b_{I}\cap c_{I}=\{I\}$, donde $I$ es el incentro del $\triangle ABC$. 

También se tiene lo siguiente:
\begin{itemize}
\item $a_{I}\cap b_{E}\cap c_{E}=\{I_{A}\}$, excentro opuesto a $A$.
\item $a_{E}\cap b_{I}\cap c_{E}=\{I_{B}\}$, excentro opuesto a $B$.
\item $a_{E}\cap b_{E}\cap c_{I}=\{I_{C}\}$, excentro opuesto a $C$.
\end{itemize}

Ahora constrúyase la circunferencia inscrita añ $\triangle ABC$, ($\mathcal{C}(I,r)$) así como la excrita al $\triangle ABC$ por el lado opuesto a $K\in\{A,B,C\}$ ($\mathcal{C}(I_{K},r_{K})$).

Sean $\mathcal{C}(I,r)\cap\overline{AB}=\{Z\}$, $\mathcal{C}(I,r)\cap\overline{BC}=\{X\}$, $\mathcal{C}(I,r)\cap\overline{CA}=\{Y\}$.
 
Y para toda $K\in\{A,B,C\}$ sean:
$\mathcal{C}(I_{K},r_{K})\cap\overline{AB}=\{Z_{K}\}$, $\mathcal{C}(I_{K},r_{K})\cap\overline{BC}=\{X_{K}\}$, $\mathcal{C}(I_{K},r_{K})\cap\overline{CA}=\{Y_{K}\}$.
\begin{df}
$s$ es el \textcolor{red}{\bf semiperímetro}\index{semiperímetro} del  $\triangle ABC$ si y solamente si $s=\frac{AB+BC+CA}{2}$ así que $AZ_{A}=Y_{A}A$. 
\end{df}
Veamos ahora algunas propiedades que se cumplen en la construcción hecha. 
\begin{itemize}
\item $Pot_{\mathcal{C}(I_{A},r_{A})}(A)=(AZ_{A})^{2}=(AY_{A})^{2}$. Entonces $(AZ_{A})^{2}=(AY_{A})^{2}$.

Análogamente, $BZ_{A}=BX_{A}$, $AZ=YA$, $ZB=BX$, $XC=CY$, $X_{A}C=Y_{A}C$. 
\item Como el perímetro del $\triangle ABC$ está dado por $P=AB+BC+CA$, tenemos que:
\begin{eqnarray*}
AB+BC+CA
&=& AB+BX_{A}+X_{A}C+CA\\
&=& AB+BZ_{A}+Y_{A}C+CA\\
&=& AZ_{A}+Y_{A}A\\
&=& 2(AZ_{A})\\
\end{eqnarray*}

Por tanto, $AZ_{A}=S=Y_{A}A$.
\item 
\begin{eqnarray*}
AB+BC+CA
&=& AZ+ZB+BX+XC+CY+YA\\
&=& 2AZ+2ZB+2XC\\
&=& 2(AZ+ZB+XC)\\
\end{eqnarray*}
Por tanto, $s=AB+XC$. Entonces $AZ_{A}=AB+XC$, así que $AB+BZ_{A}=AB+XC$, entonces $BZ_{A}=XC$, $BX_{A}=XC$, $BX_{A}+X_{A}X=X_{A}X+XC$, luego $BX=X_{A}C$. 

\item Si $L$ es el punto medio de $BC$, entonces $L$ es punto medio de los puntos de tangencia. 

Como $L$ es punto medio de $BC$, tenemos que $BL=LC$, entonces $BX+XL=LX_{A}+X_{A}C$ y como $BX=X_{A}C$ se tiene que $XL=LX_{A}$. 

\item Tenemos que 
\begin{eqnarray*}
\A(\triangle ABC)
&=& \A(\triangle ABI)+\A(\triangle BCI)+\A(\triangle CAI)\\
&=& \frac{AB\cdot r}{2}+\frac{BC\cdot r}{2}+\frac{CA\cdot r}{2}\\
&=& r\left(\frac{AB+BC+CA}{2}\right)\\
&=& r\cdot s\\
\end{eqnarray*}
\begin{eqnarray*}
\A(\triangle ABC)
&=& \A(\triangle ABI_{A})+\A(\triangle CAI_{A})-\A(\triangle BCI_{A})\\
&=& \frac{AB\cdot r_{A}}{2}+\frac{CA\cdot r_{A}}{2}-\frac{BC\cdot r_{A}}{2}\\
&=& r_{A}\left(\frac{AB+CA-BC}{2}\right)\\
&=& r_{A}\left(\frac{AB+BC+CA-2BC}{2}\right)\\
&=& r_{A}(s-BC)\\
\end{eqnarray*}

Análogamente se tiene que:

$\A(\triangle ABC)=r_{B}(s-CA)=r_{C}(s-AB)$.

Así, $\frac{1}{r_{A}}=\frac{s-BC}{\A(\triangle ABC)}$, $\frac{1}{r_{B}}=\frac{s-CA}{\A(\triangle ABC)}$ y $\frac{1}{r_{C}}=\frac{s-AB}{\A(\triangle ABC)}$. 

Y por tanto 
\begin{eqnarray*}
\frac{1}{r_{A}}+\frac{1}{r_{B}}+\frac{1}{r_{C}}
&=& \frac{s-BC+s-CA+s-AB}{\A(\triangle ABC)}\\
&=& \frac{3s-2s}{\A(\triangle ABC)}\\
&=& \frac{s}{\A(\triangle ABC)}\\
&=& \frac{1}{r}\\
\end{eqnarray*}
\end{itemize}


\section{Cuadrángulo ortocéntrico}
Sea $\triangle ABC$ y sean $a_{I}$ la bisectriz interna y $a_{E}$ la bisectriz externa por $A$, $b_{I}$ la bisectriz interna y $b_{E}$ la bisectriz externa por $B$, $c_{I}$ la bisectriz interna y $c_{E}$ la bisectriz externa por $C$. 

Y consideremos $I$ el incentro del $\triangle ABC$, $I_{A}$ el excentro opuesto a $A$, $I_{B}$ el excentro opuesto a $B$, $I_{C}$ el excentro opuesto a $C$.

Afirmación: $I,I_{A},I_{B},I_{C}$ son tales que cuando se elijen tres de estos puntos el restante es el ortocentro del triángulo que ellos determinan.
Consideremos
\begin{itemize}
\item $\triangle II_{A}I_{C}$

Tenemos que $h_{I}=b_{I}$, $h_{I_{A}}=c_{E}$ y $h_{I_{C}}=a_{E}$, entonces $h_{I}\cap h_{I_{A}}\cap h_{I_{C}}=b_{I}\cap c_{E}\cap a_{E}=\{I_{B}\}$. Por tanto, el ortocentro del $\triangle II_{A}I_{C}$ es $I_{B}$.
\item $\triangle I_{A}I_{B}I_{C}$

Tenemos que $h_{I_{A}}=a_{I}$, $h_{I_{B}}=b_{I}$ y $h_{I_{C}}=c_{I}$, entonces $h_{I_{A}}\cap h_{I_{B}}\cap h_{I_{C}}=a_{I}\cap b_{I}\cap c_{I}=\{I\}$. Por tanto, el ortocentro del $\triangle I_{A}I_{B}I_{C}$ es $I$.
\item $\triangle II_{A}I_{B}$

Tenemos que $h_{I}=c_{I}$, $h_{I_{A}}=b_{E}$ y $h_{I_{B}}=a_{E}$, entonces $h_{I}\cap h_{I_{A}}\cap h_{I_{B}}=c_{I}\cap b_{E}\cap a_{E}=\{I_{C}\}$. Por tanto, el ortocentro del $\triangle II_{A}I_{B}$ es $I_{C}$.
\item $\triangle II_{B}I_{C}$

Tenemos que $h_{I}=a_{I}$, $h_{I_{B}}=c_{E}$ y $h_{I_{C}}=b_{E}$, entonces $h_{I}\cap h_{I_{B}}\cap h_{I_{C}}=a_{I}\cap c_{E}\cap b_{E}=\{I_{A}\}$. Por tanto, el ortocentro del $\triangle II_{B}I_{C}$ es $I_{A}$.
\end{itemize}

\begin{obs}
\begin{itemize}
\item El triángulo pedal del $\triangle II_{A}I_{C}$ es el $\triangle BCA$.
\item El triángulo pedal del $\triangle I_{A}I_{B}I_{C}$ es el $\triangle ABC$. 
\item El triángulo pedal del $\triangle II_{A}I_{B}$ es el $\triangle CBA$.
\item El triángulo pedal del $\triangle II_{B}I_{C}$ es el $\triangle ACB$.
\end{itemize}
Por tanto el triángulo pedal de los 4 triángulos obtenidos a partir de $I,I_{A},I_{B},I_{C}$ es el $\triangle ABC$. 
\end{obs}

Recordemos que la circunferencia de los nueve puntos de un triángulo pasa por los puntos medios de los lados del triángulo, los pies de las alturas y los puntos medios de los segmentos determinados por el ortocentro y cada vértice de triángulo. 

Así, si $\mathcal{C}(O,r)$ es la circunferencia que inscribe al $\triangle ABC$ tenemos que $\mathcal{C}$ es la circunferencia de los nueve puntos de $\triangle II_{A}I_{C}$, $\triangle I_{A}I_{B}I_{C}$, $\triangle II_{A}I_{B}$ y $\triangle II_{B}I_{C}$.

\begin{df}
Sean $\{A,B,C,D\}$ un conjunto de puntos de tal forma que cada que elegimos tres de ellos el cuarto es el ortocentro del triángulo que ellos determinan. Llamaremos a $\{A,B,C,D\}$ un \textcolor{red}{\bf grupo ortocéntrico}\index{grupo ortocéntrico} de puntos. 
\end{df}

\section{Triángulos referidos a un grupo ortocéntrico de puntos}
\begin{teo}
En un grupo ortocéntrico de puntos, los cuatro triángulos que ellos determinan tienen la mismca circunferencia de los nueve puntos. 
\end{teo}
\begin{dem}
Sean $\{A,B,C,D\}$ un grupo ortocéntrico de puntos y sean $\triangle ABC$, $\triangle ABD$, $\triangle BCD$ y $\triangle ACD$ los cuatro triángulo que estos determinan. 

Consideremos $\mathcal{C_{A}}(O_{A},r)$, $\mathcal{C_{B}}(O_{B},r)$, $\mathcal{C_{C}}(O_{C},r)$ y $\mathcal{C_{D}}(O_{D},r)$ las circunferencias que inscriben al $\triangle BCD$, $\triangle ACD$, $\triangle ABD$ y al $\triangle ABC$ respectivamente, entonces $\overline{O_{B}O_{C}}$ es mediatriz de $AD$ pues $|O_{B}A|=|O_{B}D|=|O_{C}A|=|O_{C}D|=r$. Análogamente $\overline{O_{A}O_{D}}$ es mediatriz de $BC$, $\overline{O_{B}O_{D}}$ es mediatriz de $AC$ y $\overline{O_{C}O_{D}}$ es mediatriz de $AB$. 

Sean $\overline{O_{A}O_{D}}\cap\overline{BC}=\{L\}$, $\overline{O_{B}O_{D}}\cap\overline{AC}=\{M\}$ y $\overline{O_{C}O_{D}}\cap\overline{AB}=\{N\}$.

Observemos además que $\overline{BC}$ es mediatriz de $O_{A}O_{D}$, $\overline{AC}$ es mediatiz de $O_{B}O_{D}$ y $\overline{AB}$ es mediatriz de $O_{C}O_{D}$.

De lo anterior tenemos que $|BL|=|LC|$, $|AM|=|MC|$, $|AN|=|NB|$, $|O_{D}L|=|LO_{A}|$, $|O_{B}M|=|O_{D}M|$ y $|O_{C}N|=|NO_{D}|$. 

Entonces $|O_{D}O_{A}|=2|O_{D}L|$, $|O_{D}O_{B}|=2|O_{D}M|$ y $|O_{D}O_{C}|=2|O_{D}N|$.
por lo tanto $\triangle LMN$ y $\triangle O_{A}O_{B}O_{C}$ son homotéticos desde $O_{D}$ con razón $\frac{1}{2}$. 

Recordemos que $\triangle LMN$ es homotéticos al $\triangle ABC$ desde eñ centroide $G$ con razón $\frac{1}{2}$. Así que $\triangle ABC\equiv\triangle O_{A}O_{B}O_{C}$, entonces $\{O_{A},O_{B},O_{C},O_{D}\}$ es un grupo ortocéntrico de puntos.

Puesto que $\mathcal{C}_{9}(\triangle ABC)$ contiene a $L,M,N$ por ser los puntos medios de sus lados y $\mathcal{C}_{9}(\triangle O_{A}O_{B}O_{C})$ contiene a $L,M,N$ por ser los puntos medios de los segmentos del vértice al ortocentro, concluimos que $\triangle BCD$, $\triangle ACD$, $\triangle ABD$, $\triangle ABC$, $\triangle O_{A}O_{B}O_{C}$, $\triangle O_{A}O_{B}O_{D}$, $\triangle O_{A}O_{C}O_{D}$ y $\triangle O_{B}O_{C}O_{D}$ tienen la misma circunferencia de los nueve puntos.
\end{dem}


\section{La línea de Simson}
\begin{teo}\label{TLS}
Sean $\triangle ABC$ y $P$ un punto en el plano. Consideremos la recta $a$ la ortogonal a $\overline{BC}$ por $P$, $b$ la ortogonal a $\overline{CA}$ por $P$ y $c$ la ortogonal a $\overline{AB}$ por $P$ tales que $a\cap\overline{BC}=\{X\}$, $b\cap\overline{CA}=\{Y\}$ y $c\overline{AB}=\{Z\}$. Entonces si $\mathcal{C}(O,r)$ es la circunferencia que inscribe al $\triangle ABC$, $P\in\mathcal{C}$ si y solamente si $X,Y\;\&\;Z$ son colineales. (Consideramos $\triangle ABC$ ordenado levógiramente).
\end{teo}
\begin{dem}
\begin{enumerate}
\item [($\Rightarrow$)]
Supongamos que $P\in\mathcal{C}$ y sin perder generalidad supongamos $P\in\widehat{CA}$, entonces $\square ABCP$ es convexo y cíclico. 

Observemos lo siguiente:
\begin{itemize}
\item $\angle CBA=2\perp-\angle APC$.
\item $|\angle PZA|=|\angle PYA|=\perp$, entonces $\{P,Y,Z,A\}$ es concíclico.
\item $|\angle PYC|=|\angle PXC|=\perp$, entonces $\{P,X,Y,C\}$ es concíclico. 
\end{itemize}

Por la segunda observación, tenemos dos casos:
\begin{enumerate}
\item $\angle PAZ=\angle PYZ$.
\item $\angle PAZ=2\perp -\angle PYZ$. 
\end{enumerate}
Supongamos que $\angle PAZ=\angle PYZ$, entonces $\{A,Y,\}\subset\widehat{ZP}$, entonces $Z$ divide externamente a $AB$ y $X$ divide internamente a $BC$. Como $Z$ divide externamente, $\angle BAZ=2\perp$ además $\angle BAZ=\angle BAP+\angle PAZ$, entonces $\angle BAP=2\perp -\angle PAZ$, así $\angle BAP=2\perp - \angle PYZ$. 
Como $\square ABCP$ es cíclico, $\angle BAP=2\perp - \angle PCB$, entonces $2\perp -\angle PCB=2\perp - \angle PYZ$, por tanto $\angle PCB=\angle PYZ$.

Ahora como $X$ divide internamente a $BC$, entonces $\{X,Y,\}\subset\widehat{PC}$, $\angle XYP=2\perp -\angle PCY$ pero $X\in BC$, entonces $\angle PCX=\angle PCB$. Entonces $\angle XYP=2\perp -\angle PCB=\angle BAP$ por tanto $\angle XYP=\angle BAP$. 

El caso 2 es análogo pues $X$ divide externamente a $BC$.

Como $\angle BAP=2\perp$, tenemos que $\angle BAP+\angle PAZ=2\perp$, entonces $\angle XYP+\angle PYZ=2\perp$, por lo que $\angle XYZ=2\perp$ y por lo tanto $X$, $Y$ y $Z$ son colineales. 

\item [($\Leftarrow$)]
Supongamos que $X,Y,Z$ son colineales, entonces por el Teorema~\ref{Teo de Menelao} tenemos que 
$$\frac{AZ}{ZB}\cdot\frac{BX}{XC}\cdot\frac{CY}{YA}=-1$$
Supongamos que $0<\frac{CY}{YA}$, $0<\frac{AZ}{ZB}$, $0<\frac{BX}{XC}$. Como $Z$ divide externamente a $AB$, $\angle PAZ=\angle PYZ$ y como $X$ divide internamente a $BC$, $\angle XYP=2\perp -\angle PCX$ pero $X$ y $B$ son colineales, entonces $\angle XYP=2\perp - \angle PCB$. 

Además como $X,Y,Z$ son colineales, entonces $\angle XYZ=2\perp$ y $\angle XYZ=\angle XYP+\angle PYZ$, entonces $\angle XYP+\angle PYZ==2\perp$, así $2\perp -\angle PCB+\angle PAZ=2\perp$ por tanto $\angle PAZ=\angle PCB$. 

Por otra parte, como $\angle BAZ=2\perp$, entonces $\angle BAP+\angle PAZ=2\perp$, entonces $\angle BAP+\angle PCB=2\perp$.
Por lo tanto $\square ABCP$ es concíclico, así que $P\in\mathcal{C}$.
\end{enumerate}
\end{dem}

La línea que contiene a los puntos $X,Y,Z$ del Teorema~\ref{TLS} es llamada la \textcolor{red}{\bf linea de Simson}\index{línea de Simson} de $P$ con respexto al $\triangle ABC$.
\section{Ángulo de intersección de líneas de Simson}

Sea $\triangle ABC$ y $\mathcal{C}(O,r)$ la circunferencia que inscribe al $\triangle ABC$. Consideremos $P,P'\in\mathcal{C}(O,r)$ y sean $l,l'$ las líneas de Simson de $P,P'$ con respecto al $\triangle ABC$ respectivamente, tales que $l\cap BC=\{X\}$, $l\cap CA=\{Y\}$, $l\cap AB=\{Z\}$, $l'\cap BC=\{X'\}$, $l'\cap CA=\{Y'\}$ y $l'\cap AB=\{Z'\}$. 

Sean $PX\cap\mathcal{C}\backslash\{P\}=\{U\}$ y $P'X'\cap\mathcal{C}\backslash\{P'\}=\{U'\}$.

Afirmación: $AU$ es paralela a $l$.

En efecto, como $\{P,C,X,Y\}$ es concíclico, $\angle PXY= \angle PCY=\angle PCA=\angle PUA$ por tanto $XY$ es paralela a $AU$.

Ahora, notemos que $\angle (P\longrightarrow P')=\angle UAU'$, entonces $PU$ es paralela a $P'U'$ pues son ortogonales a $BC$, entonces $|\widehat{UU'}|=|\widehat{PP'}|$, luego $\angle UAU'=\frac{1}{2}\angle UOU'=\frac{1}{2}\angle POP'$.

Por lo tanto, el $\angle (P\longrightarrow P')$ es la mitad del ángulo central que determina el arco $\widehat{PP'}$.


\section{La línea de Simson y la circunferencia de los nueve puntos}
\begin{teo}
Sean $\triangle ABC$ y $H$ su ortocentro, entonces la línea de Simson de un punto $P$ con respecto al $\triangle ABC$ biseca al segmento $PH$ y también $\mathcal{C}_{9}(\triangle ABC)$ biseca a $PH$. 
\end{teo}

\begin{teo}
Sea $\triangle ABC$ y $\mathcal{C}(O,r)$ la circunferencia que inscribe al $\triangle ABC$. Sean $P,Q\in\mathcal{C}$ tales que $O\in\overline{PQ}$, $l$ y $m$ la líneas de Simson con respecto al $\triangle ABC$ de $P$ y $Q$ respectivamente, entonces $l$ y $m$ se intersecan en ángulos rectos sobre $\mathcal{C}_{9}(\triangle ABC)$. 
\end{teo}



\addcontentsline{toc}{chapter}{Bibliografía}
\begin{thebibliography}{99}

\bibitem{Bulajich-Gómez} Bulajich Manfrino, R., Gómez Ortega, J.A. {\it Geometría}. Cuadernos de olimpiadas matemáticas. 1 ed. 2002, México: Instituto de Matemáticas, UNAM. 188.

\bibitem{Euclides} Euclides {\it Elementos}. Libros I-IV. 1 ed. 1991, España: Editorial Gredos, S.A. 368.

\bibitem{Euclides} Euclides {\it Elementos}. Libros V-IX. 1 ed. 1994, España: Editorial Gredos, S.A. 240.

\bibitem{Euclides} Euclides {\it Elementos}. Libros X-XIII. 1 ed. 1996, España: Editorial Gredos, S.A. 358.

\bibitem{Shively} Shively, L.S., {\it Introducción a la Geometría Moderna}. 1 ed. 1979, México: Compañía Editorial Continental, S.A. 172.

\end{thebibliography}




\printindex
%\listoffigures
%\listoftables

\end{document}
